\section{Threat model}
In this section the threat model will give an overview of what kind of threats there are to the system. Threat in this case targets a vulnerability that exists in the IT infrastructure, network or applications. This can be internal factors, external factors, and physical threats. The section is divided into four sub categories which are threats relevant to the pre-election, election, post-election phase, and threats which are always present no matter which election phase the system is currently in. Additionally, some threats are exclusive to which 'voter registration mode' the election is configured to run so this notation can be seen where applicable. 

\subsection{Threat types}
This threat model roughly categorizes the different threats into three different types.

\begin{itemize}
    \item Internal
    \item External
    \item Physical
\end{itemize}

\textbf{Internal} are threats which come from inside the organization. It could be malicious employees or genuine human errors which can still have dire consequences.

\textbf{External} are threats which come from outside the organization. It could be a malicious outsider trying to gain access to the systems or malware infecting employees computers.

\textbf{Physical} are threats which stem from physical hardware. Servers could break down or the organization be victim to a natural disaster.


\subsection{Threat examples}

Each threat in the following subsections is described by a title, a deeper description of the vulnerability and what asset of the system is at risk or affected by this threat.

\subsubsection{Pre-election phase threats}
These are threats which are relevant in the pre-election phase of the system.
%\subsubsection{Malicious election administrator}
%\subsubsection{Opening ceremony}
%\subsubsection{OPT provider}

\subsubsection{Election phase threats}
These are threats which are relevant in the election phase of the system.


\subsubsection{Post-election phase threats}
These are threats which are relevant in the post-election phase of the system.

\subsubsection{Generic threats}
These are threats that are always a threat to the system no matter which election phase the system is in.
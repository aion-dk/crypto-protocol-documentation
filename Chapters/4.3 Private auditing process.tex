\subsection{Private auditing process} \label{sec: private auditing process}
This process is only relevant if the voter registration mode 'On-demand' described in \cref{sec: on-demand mode} is used. This auditing is private since the identification token contains the voter's identity information such an email address which is sensitive information that cannot be publicly accessible and must be audited privately within the system's borders.


\subsubsection{Eligibility verifiability}
During the post-election the entire bulletin board is audited. The auditor needs to be provided with the list of eligible voters $\boldsymbol{\mathcal{V}}$ from the \textit{Election Administrator} $\mathcal{E}$. The auditor also need to be provided with all identity tokens from the \textit{Identity Provider} $\mathcal{I}$. All authentication tokens and public keys are provided through the \textit{Digital Ballot Box} $\mathcal{D}$. The Auditor is also required to be provided the list that links the identity and authentication tokens with a voter that the \textit{Voter Authorization Service} $\mathcal{A}$ stores. 

The private auditor validates the signature of the identity token, authentication token, and public key by \( \mathbf{VerifySignature}_{Y_i} (\sigma, h_\mathrm{v}) \) (\cref{alg: sig ver}). The auditor checks for each authentication token that their individual link to a identity token matches the link the \textit{Voter Authorization Service} $\mathcal{A}$ has. It is also checked that the link in the public key token of the \textit{Voter} $\mathcal{V}$ is linked to the same identity token. The private auditor also verifies that the identity of \textit{Voter} $\mathcal{V}_i$ that is in the identity token is also on the list of eligible voters $\mathcal{V}_i \in \boldsymbol{\mathcal{V}}$. 

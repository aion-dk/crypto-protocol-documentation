\subsection{Requirements} \label{sec: requirements}
The requirements that Assembly Voting X must fulfill are split into the following three categories: functional, non-functional, and security requirements.


\subsubsection{Functional Requirements} \label{sec: functional requirements}
Functional requirements relate to properties of the election system that voters, or users in general (including election officials, candidates or auditors), can actively choose to perform. These properties are in some way measurable. Assembly Voting X has the following functional requirements:

\begin{itemize}
    \item election types supported:
    \begin{itemize}
        \item referendum: direct vote on a proposal or issue,
        \item candidate election: one vote on a candidate from a predefined list,
        \item multiple choice: a selection of multiple vote options,
        \item ranked election: an ordered selection of multiple vote options,
        \item write-in vote: a free-form text of a maximum size,
    \end{itemize}
    \item verification mechanisms for voters to check that the encrypted ballot contains what they expect,
    \item possibility to confirm selected options after voting by an overview of the complete ballot,
    \item possibility to correct mistakes before submitting an encrypted ballot,
    \item ability for overwrite your vote, i.e. a voter can vote multiple times while only the latest submitted vote will be counted in the final tally,
    \item ability to check the status of your ballot after submission,
    \item public auditability of the election process throughout the election period.
\end{itemize}


\subsubsection{Non-functional Requirements} \label{sec: non-functional requirements}
Non-functional requirements describe properties of the election system that impact the user experience while interacting with the system. 

\paragraph{Mobility} is the property that enables voters to use any device (PC, laptop, tablet, smart phone) in their possesion, to connect to the election system. They do not need to be in a special location (e.g. polling station) in order to vote. Instead, they can participate in the voting process being located in any place, that they consider private, and that has an internet connection.

\paragraph{Vote \& go} entails that voters are only required to be present during voting phase. Results can be computed without the presence of voters.

\paragraph{Transparency} implies that election data is available for auditing through a public bulletin board.


\subsubsection{Security Requirements} \label{sec: security requirements}
Security requirements describe properties of the election system that contribute to the quality and reliability of an election result. This section briefly describes the properties, while the explanation of how these propeties are achieved is presented in \cref{sec: election properties}.

\paragraph{Eligibility} property is defined as the fact that only a limited number of predefined voters are allowed to cast a valid vote. 

\paragraph{Privacy} property implies that no entity can read a partial result or any votes before it is intended. This is to prevent influencing of the subsequent voters throughout the election period. Voters could be swayed to vote differently than their initial decision if the current results were publicized.

\paragraph{Anonymity} property implies that no single entity can determine the way a particular voter voted.

\paragraph{Integrity of voting data} is the property that implies detection mechanisms of whether any votes recorded on the bulletin board during the election phase have been modified or deleted.

\paragraph{Verifiability} property describes that all steps of the election protocol are verifiable by following some auditing process.

\paragraph{Receipt-freeness} property is defined as the fact that voters are not able to prove to a third party the way they voted, after they submitted the encrypted ballot.

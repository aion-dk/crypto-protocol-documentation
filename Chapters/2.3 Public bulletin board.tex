\subsection{Public bulletin board} \label{sec: public bulletin board}
All events that happen during an election are published by the \textit{Digital Ballot Box} $\mathcal{D}$ as items on a publicly available bulletin board. Each item from the board is owned (or written) by a relevant actor. Each item posted on the bulletin board describes a specific event and it is uniquely identifiable by its \textit{hash value}. The \textit{hash value} of the last item on the board represents the \textit{board hash value} at that specific point in time. All events are stored as an \textit{append only list}, that means, no event can be removed or replaced and each new event is appended at the end of the list. The structure of the bulletin board has been inspired from \cite{Heather09}.

The way we deviate from \cite{Heather09} is that, to append a new item on the board, the writer needs to include as part of the new item a reference to any existing item from the board (i.e. include the \textit{hash value} of that item), instead of referencing exactly the previous item. We call this reference, the \textit{parent item}. Finally, the \textit{hash value} of the new item is computed by the \textit{Digital Ballot Box} $\mathcal{D}$ by hashing the content of the item (including the reference to the \textit{parent item}) concatenated with the current \textit{board hash value} and a registration timestamp. $\mathcal{D}$ signs the \textit{hash value} of the new item and delivers it to the writer as proof of acceptance of the new item on the board. Note that the \textit{Digital Ballot Box} ensures that the new item is linked directly to the previous item on the board.

As a result of this modification, each item from the bulletin board references 2 other items:
\begin{enumerate}
    \item An existing item that the writer chooses as the \textit{parent item}
    \item The previous item of the board
\end{enumerate}

This modification to the bulletin board structure implies that the \textit{history} property described in \cite{Heather09} is protected in this case by the \textit{Digital Ballot Box}. Furthermore, we introduce a new property to the bulletin board called \textit{ancestry} which is defined by items being related to each other in a meaningful way. As a result, when traversed on the \textit{ancestry} line, the structure of the bulletin board looks like a tree, while, when traversed on the \textit{history} line, the structure looks linear. 

In addition, we introduce a new concept to the bulletin board structure, that we call a \emph{hidden verification track}, that is used for performing the \textit{ballot checking process} described in \Cref{sec: challenging a vote cryptogram}. It is called:
\begin{itemize}
    \item \emph{hidden} because this track is not publicly available as part of the bulletin board, instead, it is available on request based on the \textit{hash value} of a specific event.
    \item \textit{verification} because this track is used just for the purpose of the \textit{ballot checking process}.
    \item \emph{track} because it spawns an extra \textit{history} of events that is injected under a specific item from the main \textit{history}.
\end{itemize}

As a consequence of these modifications, any $i^\mathrm{th}$ item from the bulletin board consists of the following tuple $(m_i, c_i, \mathcal{W}, \sigma_i, p_i, h_{i-1}, t_i, h_i)$, where $m_i$ is the type of the item, $c_i$ is the description of the event, $\mathcal{W}$ is a reference to a writer of the item, $\sigma_i$ is the writer's signature, $p_i$ is the \textit{hash value} of the parent item with $p_i \in \{h_1, ..., h_{i-1}\}$, $h_{i-1}$ is the \textit{hash value} of the previous item in the \textit{history}, $t_i$ is the registration timestamp and $h_i$ is the item's \textit{hash value}.

The different kinds of items (i.e. the values that $m_i$ can have) and the events they support are described in \Cref{sec: bulletin board event types}. The rules about how items can reference a parent item are also described in this section.

In order to write a new item on the bulletin board, a writer needs to follow the protocol described in \Cref{sec: writing on the bulletin board}. The following actors are allowed to write on the bulletin board:
\begin{itemize}
    \item \textit{Election Administrator} $\mathcal{E}$ is the actor that writes all the configuration events of an election.
    \item \textit{Voters} $\mathcal{V}_i$, with $i \in \{1, ..., n_\mathrm{v}\}$, are the actors that write all the vote related events of an election.
    \item \textit{Digital Ballot Box} $\mathcal{D}$ is the actor that ultimately accepts all the events that are published on the bulletin board. In addition, $\mathcal{D}$ also writes on the board all the auxiliary events that support the voting process.
    \item \textit{External Verifier} $\mathcal{X}$ is the actor that writes events related to ballot checking process. These events are not part of the public track of the bulletin board.
\end{itemize}


\subsubsection{Writing on the bulletin board} \label{sec: writing on the bulletin board}
This section describes the protocol that any writer needs to follow in order to write an event on the bulletin board. The election protocol allows a predefined set of actors ($\mathcal{E}$, $\mathcal{V}_i$, $\mathcal{D}$ or $\mathcal{X}$) to write events on the bulletin board, but for the sake of generalization, \Cref{fig: protocol for writing an item on the bulletin board} presents the interaction between a generic writer $\mathcal{W}$ and the \textit{Digital Ballot Box} $\mathcal{D}$ necessary for publishing the $i$\textsuperscript{th} event on the bulletin board.

\begin{figure}[ht]
    \centering
    \begin{tikzpicture}[framed, node distance=0,
            % every node/.style={draw}
            ]{
            
        % Actors
        \node[title, above, anchor=north east] (w) {
            \textbf{Writer $\mathcal{W}$}};
        \node[title, right=of w] (dbb) {
            \textbf{Digital Ballot Box} $\mathcal{D}$};
        
        % internal knowledge
        \node[block, below=of w] (w1) {
            /* internal knowledge: \\
            \phantom{/}* $x_\mathcal{W}$, $Y_\mathcal{D}$, $m_i$, $c_i$, $p_i$ \\
            \phantom{/}*/
            };
        \node[block, below=of dbb] (dbb1) {
            /* internal knowledge: \\
            \phantom{/}* $x_\mathcal{D}$, $Y_\mathcal{W}$, $\boldsymbol{h} = \{h_1, ... h_{i-1}\}$ \\
            \phantom{/}*/
            };
        
        % All content
        \node[block, below=of w1] (w2) {
            $\sigma_i \gets \mathsf{Sign}(x_\mathcal{W}; m_i || c_i || p_i)$
            };
        \node[arrow, below=of w2.south, anchor=north west] (a1) {
            $\sigma_i$, $m_i$, $c_i$, $p_i$
            };
        \node[block, anchor=north west] at (a1.south -| dbb1.west) (dbb2) {
            verify that $m_i$, $c_i$, and $p_i$ are compatible and $\mathsf{SigVer} (Y_\mathcal{W}, \sigma_i; m_i || c_i || p_i)$ then: \\ [9pt]
            $t_i \gets$ current timestamp \\
            $h_i \gets \mathcal{H}(m_i || c_i || p_i || h_{i-1} || t_i)$ \\
            $\rho_i \gets \mathsf{Sign}(x_\mathcal{D}; \sigma_i || h_i)$ \\ [9pt]
            /* store on the Bulletin Board \\
            \phantom{/}* as the $i^\mathrm{th}$ item the tuple: \\
            \phantom{/}* $(m_i, c_i, \mathcal{W}, \sigma_i, t_i, p_i, h_i)$ \\
            \phantom{/}*/
            };
        \node[arrow, anchor=north west] at (a1.west |- dbb2.south) (a2) {
            $\rho_i$, $t_i$, $h_i$, $h_{i-1}$
            };
        \node[block, anchor=north west] at (a2.south -| w2.west) (w3) {
            verify that \\
            $h_i = \mathcal{H}(m_i || c_i || p_i || h_{i-1} || t_i)$ and \\
            $\mathsf{SigVer} (Y_\mathcal{D}, \rho_i; \sigma_i || h_i)$
            };
        
        % Arrows and lines
        \draw[dashed] (w.south west)--(dbb.south east);    
        \draw[dashed] (w1.south west)--(dbb1.south east);
        \draw[->] (a1.south west)--(a1.south east);
        \draw[->] (a2.south east)--(a2.south west);
        }
    \end{tikzpicture}
    \caption{Protocol for writing an item on the bulletin board}
    \label{fig: protocol for writing an item on the bulletin board}
\end{figure}

The publicly available information consists of: the public key of the writer $Y_\mathcal{W}$, the public key of the \textit{Digital Ballot Box} $Y_\mathcal{D}$ and all the already existing items on the bulletin board with their respective hash values $\boldsymbol{h} = \{h_1, ... h_{i-1}\}$. The writer alone is in possession of his private key $x_\mathcal{W}$, while the \textit{Digital Ballot Box} is in possession of its private key $x_\mathcal{D}$.

The protocol starts by the writer picking the event type $m_i$ and its content $c_i$ to be appended on the bulletin board as the $i^\mathrm{th}$ item. All the event types are described in the \Cref{sec: bulletin board event types}. The content of the item is a text blob describing a specific event. The structure of the message has to follow the rules described in \Cref{sec: bulletin board event types} depending on the type of item chosen. Then, the writer chooses a pre-existing item on the bulletin board as the parent of the new item. The parent item is referenced by its hash value $p_i \in \boldsymbol{h}$. The choice of parent item is done according to the rules described in \Cref{sec: bulletin board event types} depending on the type of item chosen.

The writer signs with his private key $x_\mathcal{W}$ the new item concatenated with its parent hash value $m_i || c_i || p_i$. The resulting signature $\sigma_i$ is sent together with the item type $m_i$, the content $c_i$ and the parent hash value $p_i$ to the \textit{Digital Ballot Box} as a request to append the new item on the board.

\textit{Digital Ballot Box} verifies whether $m_i$, $c_i$ and $p_i$ are chosen according to the rules specified in \Cref{sec: bulletin board event types} and whether the request has a valid signature. If all validations succeed, \textit{Digital Ballot Box} generates the hash value of the new item $h_i$ by hashing a concatenation of the new item type $m_i$, its content $c_i$, its parent hash value $p_i$, the current board hash value $h_{i-1}$ and the registration timestamp $t_i$. It then stores the new item on the bulletin board as the tuple $(m_i, c_i, \mathcal{W}, \sigma_i, t_i, p_i, h_i)$, where $\mathcal{W}$ is a reference to the writer.

The \textit{Digital Ballot Box} signs with its private key $x_\mathcal{D}$ the concatenation of the writer's signature $\sigma_i$ and the hash value of the new item $h_i$. The resulting signature $\rho_i$ is sent together with the registration timestamp $t_i$, the new board hash value $h_i$ and the exactly previous board hash value $h_{i-1}$ to the writer as proof that the item has been appended on the board.

Finally, the writer verifies whether the hash value of the new item is computed correctly and whether the response has a valid signature.

Note that, when the protocol is performed by a specific writer, for example, the \textit{Voter} $\mathcal{V}_i$, the writer's key pair $(x_\mathcal{W}, Y_\mathcal{W})$ will be replaced by the voter's key pair $(x_i, Y_i)$.

% % \clearpage

% \color{red}
% no

% The process of having a vote submitted to the bulletin board starts by the \textit{Voter} $\mathcal{V}_i$ asking for the latest hash value of the board. The \textit{Digital Ballot Box} $\mathcal{D}$ returns the current hash value of the board $h_{\mathrm{a},i}$ and the current time stamp $t_{\mathrm{a},i}$, which will be used as parameters in the generation of the vote submission. The entire process can be seen in figure \ref{fig:vote_submission_protocol}.

% no

% Each time a new submission of a vote cryptogram \( (e_i, h_{\mathrm{v},i}, \sigma_i, PK_i) \) is received, the \textit{Digital Ballot Box} $\mathcal{D}$ validates the following:
% \begin{itemize}
%     \item the vote submission is not too old, i.e. \( t_{\mathrm{a},i} < t_{\mathrm{r},i} < t_{\mathrm{a},i} + \epsilon \), where $t_{\mathrm{r},i}$ is the registration timestamp and $\epsilon$ is the \textit{latency parameter}, which represents the maximum time the vote submission process can take,
%     \item the authenticity of the vote submission, i.e. $\sigma_i$ is a valid, well-formatted signature,
%     \item the correctness of the vote cryptogram, i.e. validate that $e_i$ is constructed based on the empty cryptogram $e_{0,i}$ by checking the proof of correct encryption $PK_i$.
% \end{itemize}

% no

% If all validations succeed, the vote submission is registered (as the $i^{th}$ item on the list) and a new board hash value is calculated \( h_{\mathrm{b}, i} = \mathcal{H} (h_{\mathrm{v},i} || h_{\mathrm{b}, i-1} || t_{\mathrm{r},i}) \), where $e_i$ is the vote cryptogram, $h_{\mathrm{v},i}$ is the vote's content hash, $\sigma_i$ is the \textit{Voter} $\mathcal{V}_i$'s signature on the vote submission and $t_{\mathrm{r},i}$ is the registration time stamp.

% no

% It is visible that the board hash value is calculated based on the previous board hash value in a \textit{blockchain} like manner where each vote received is a block in the chain. The first element of the bulletin board will compute its board hash value by using the previous board hash the value \( h_{\mathrm{b},0} = 0 \) (\textit{genesis hash}).

% \begin{itemize}
%     \item If the \textit{Voter} $\mathcal{V}_i$ chooses to challenge the encryption, the browser will print on the screen all information about the vote cryptogram that is necessary to revert the encryption process (i.e. the vote cryptogram $e$, the encryption key $Y_\mathrm{enc}$ and the randomizer $r$). The \textit{Voter} $\mathcal{V}_i$ must use a second, \textbf{trusted} device to perform the decryption process, in order to output the plain text $m$. If the vote $m$ corresponds to the selected candidate, the \textit{Voter} $\mathcal{V}_i$ gains confidence that his browser behaved correctly, otherwise, there is a clear evidence of an attack to the \textit{Voter} $\mathcal{V}_i$'s machine.
    
%     The \textit{Voter} $\mathcal{V}_i$ has to recast/regenerate his vote (return to previous bullet point 2) and he can repeat this process as many times as needed until he gains enough trust in the voting application.
    
%     \item The browser asks the \textit{Digital Ballot Box} $\mathcal{D}$ for the latest hash value of the board $h_\mathrm{a}$, which we call \textit{the acknowledged hash}. The browser computes the vote's content hash $h_\mathrm{v} = \mathcal{H}(s)$, where $s$ is a message containing the following information: the voter id, the election id, the vote cryptogram $e$, the acknowledged hash $h_\mathrm{a}$ and the acknowledged time stamp. 
    
%     \item The browser certifies the authenticity of the vote cryptogram by generating a response (signature) of the \textit{Voter} $\mathcal{V}_i$ on the vote's content hash as a \textit{Schnorr signature} \( \sigma \leftarrow \mathbf{Sign}_{x_i} (h_\mathrm{v}) \) (algorithm \ref{alg: sign}). The browser submits to the \textit{Digital Ballot Box} $\mathcal{D}$ the following: the vote cryptogram $e$, the proof of correct encryption $PK$, the vote content hash $h_\mathrm{v}$ and the signature $\sigma$.
    
%     % \item The browser certifies the authenticity of the vote cryptogram by generating a signature of the voter $\mathcal{V}_j$ on a message $s$ containing the following information: voter id, election id, generation time stamp and the vote cryptogram $e$. The resulting \textit{Schnorr signature} is \( \sigma \leftarrow \mathbf{Sign}_{x_j} (s) \) (algorithm \ref{alg: sign}). The browser submits to the bulletin board the following: the vote cryptogram $e$, the proof of correct encryption $PK$, the generation time stamp and the signature $\sigma$.
    
%     \item The \textit{Digital Ballot Box} $\mathcal{D}$ receives this information and accepts the new vote cryptogram if all the following are valid: the proof of correct encryption, the vote content hash, the voter's signature and the acknowledged time stamp is not too old (described in section \ref{bulletin board}).
    
%     % \item The bulletin board $\mathcal{D}$ receives this information and accepts the new vote cryptogram if both the proof of correct encryption and the voter's signature are valid. The voter's signature is verified by reconstructing the message of the signature $s$ (voter id, election id, generation time stamp and the vote cryptogram $e$) and running the \textit{Schnorr signature} verification algorithm \( \mathbf{Verify}_{Y_j} (\sigma, s) \) (algorithm \ref{alg: verify sig}).
    
%     \item If the encrypted vote is accepted, the \textit{Digital Ballot Box} $\mathcal{D}$ appends it to the bulletin board at the end of the list of vote cryptograms. Next, the \textit{Digital Ballot Box} $\mathcal{D}$ calculates the new hash value of the list (see section \ref{bulletin board}) $h_\mathrm{b} = \mathcal{H}(b)$, where $b$ is a message that contains the following information: the vote content hash $h_\mathrm{v}$, the previous hash value of the board $h_{\mathrm{b}-1}$ and the registration time stamp $t_{\mathrm{r}}$. Afterwards, the \textit{Digital Ballot Box} $\mathcal{D}$ sends back to the \textit{Voter} $\mathcal{V}_i$ a confirmation receipt $\rho$ in form of a \textit{Schnorr signature} on the following message $r$: \textit{Voter} $\mathcal{V}_i$'s signature $\sigma$ and the new hash value of the board $h_\mathrm{b}$. The confirmation receipt is computed \( \rho \leftarrow \mathbf{Sign}_{x_\mathrm{sign}} (r) \) (algorithm \ref{alg: sign}).
% \end{itemize}

% \begin{figure}[ht]
%     \centering
%     \begin{tikzpicture}[framed]
%         \matrix (m) [matrix of nodes, nodes = {draw = none, anchor = base west, align = left, text depth = 0pt} ]{
%             \textbf{Voter $\mathcal{V}_i$} & & \textbf{Digital Ballot Box} $\mathcal{D}$ \\ [2mm]
%             /*knows $x_j$*/ & & /*knows \( \boldsymbol{Y} = (Y_1, ..., Y_{n_\mathrm{v}}) \)*/ \\ [2mm]
%             \( PK \gets \mathbf{Prove}_G (x_j) \) & & \\
%             & send $j$, $PK$ & \\
%             & & \scriptsize /*validate proof*/ \\ [-1mm]
%             & & if \( \mathbf{Verify}_G (PK, Y_j) \) \\
%             & & -- successful authentication \\
%         };
        
%         \draw[shorten <= -0.5cm] (m-1-1.south east)--(m-1-1.south west);
%         \draw[shorten <= -0.5cm] (m-1-3.south east)--(m-1-3.south west);
%         \draw[-latex] (m-4-2.south west)--(m-4-2.south east);
%     \end{tikzpicture}
%     \caption{Protocol for voter authentication}
%     \label{fig: authentication protocol}
% \end{figure}
% \color{black}



\subsubsection{Bulletin board event types} \label{sec: bulletin board event types}
The bulletin board has been designed as a self-documented event log. In order to support that, it needs to contain many kinds of items that are documenting different events throughout the election, such as events related to election configuration, events related to the voting process or events related to challenging a vote cryptogram process (described in \Cref{sec: challenging a vote cryptogram}). Each event is documented as an item on the bulletin board.

All bulletin board items have many of the same attributes in their structure, that is listed below which is to be called \textbf{metadata}:
\begin{itemize}
    \item \textit{type} is a string that defines the type of the bulletin board item.
    \item \textit{content} is data that describes the details of the event.
    \item \textit{writer} is a reference to the writer of the item (e.g. a \textit{Voter} or the \textit{Digital Ballot Box}).
    \item \textit{signature} is a digital signature of the writer on the item.
    \item \textit{registration time} is a timestamp documenting when the event happened.
    \item \textit{parent} is a reference to a previously existing item on the bulletin board.
    \item \textit{address} is a unique identifier for each item on the board.
\end{itemize}

The bulletin board items will have additional attributes which will differ in addition to some rules that are also varying depending on the item type.

All item types are described in the following list and are grouped in the following 3 categories:

\paragraph{Configuration items}
\begin{enumerate}
    \item 
        The \textit{genesis} is the initial item of the bulletin board and it describes some metadata of the election. Basically, this is the item that spawns a new bulletin board. The structure of it consists of the \textbf{metadata} and the following attributes:
        \begin{itemize}
            \item the elliptic curve domain parameters $(p, a, b, G, q, h)$,
            \item the public key of the \textit{Digital Ballot Box} $Y_\mathcal{D}$,
            \item the URL of the bulletin board,
            \item the voter authorization mode.
        \end{itemize}
        The following rules apply to the \textit{genesis} item:
        \begin{itemize}
            \item It is written by the \textit{Digital Ballot Box} $\mathcal{D}$.
            \item It has no \textit{parent}.
            \item The \textit{genesis} item is the first item of the board.
            \item There is one single \textit{genesis} item on the entire bulletin board (i.e. the data in the \textit{genesis} item does not support updates). 
        \end{itemize}
    \item
        The \textit{election configuration} is an item specifying some configuration on election level. The structure of it consists of the \textbf{metadata} and the following attributes:
        \begin{itemize}
            \item the public key of the \textit{Election Administrator} $Y_\mathcal{E}$,
            \item the start and end time of the election.
        \end{itemize}
        The following rules apply to the \textit{election configuration} item:
        \begin{itemize}
            \item It is written by the \textit{Election Administrator} $\mathcal{E}$.
            \item It has as \textit{parent} the previous configuration item.
            \item The first \textit{election configuration} item defines the initial configuration of the election.
            \item The following \textit{election configuration} items define updates that are made to the election configuration.
            \item \textcolor{orange}{How does an update affect a running election?}
        \end{itemize}
    
    \item 
        The \textit{contest configuration} is an item defining the configuration of a contest. The structure of it consists of the \textbf{metadata} and the following attributes:
        \begin{itemize}
            \item a unique identifier of the contest,
            \item the marking rules of the contest,
            \item the question type of the contest,
            \item the list of candidates with their unique labels $\{m_1, ..., m_{n_\mathrm{c}}\}$, where $n_\mathrm{c}$ is the number of candidates,
            \item the result rules of the contest.
        \end{itemize}
        The following rules apply to the \textit{contest configuration} item:
        \begin{itemize}
            \item It is written by the \textit{Election Administrator} $\mathcal{E}$.
            \item It has as \textit{parent} the previous configuration item.
            \item The first \textit{contest configuration} item with a specific contest identifier defines the initial configuration of that contest.
            \item The following \textit{contest configuration} items with a specific contest identifier define updates that are made to that contest configuration.
            \item \textcolor{orange}{What attributes are allowed to be updated?}
            \item \textcolor{orange}{How does an update affect a running election?}
        \end{itemize}
        
    \item 
        The \textit{threshold configuration} is the item defining the ballot encryption key and all data relevant to the threshold ceremony. The structure of it consists of the \textbf{metadata} and the following attributes:
        \begin{itemize}
            \item the encryption key $Y_\mathrm{enc}$,
            \item the threshold setup $t$-out-of-$n_\mathrm{t}$, where $t$ is the amount of trustees needed for decryption and $n_\mathrm{t}$ is the total number of trustees,
            \item the list of trustees, each defined by its public key $Y_{\mathcal{T}_i}$ and the list of public polynomial coefficients $\{P_{\mathcal{T}_i,1}, ..., P_{\mathcal{T}_i,t-1}\}$. This data is relevant for auditing purpose.
        \end{itemize}
        The following rules apply to the \textit{threshold configuration} item:
            \begin{itemize}
                \item It is written by the \textit{Election Administrator} $\mathcal{E}$.
                \item It has as \textit{parent} the previous configuration item.
                \item There is one single \textit{threshold configuration} item on the entire bulletin board (i.e. the threshold configuration does not support updates).
            \end{itemize}
    
    \item 
        The \textit{voters configuration} is the item defining all eligible voters in the election. This item is present when the voter registration mode is \textbf{pre-election}. The structure of it consists of the \textbf{metadata}, and a list of all voters each defined by the following attributes:
        \begin{itemize}
            \item a unique identifier $vID_i$,
            \item the public key $Y_i$,
            \item a flag defining whether the voter is enabled or not. \textcolor{orange}{This is so the election administrators is able to disable a voter if necessary. This information is public to maintain end to end verifiability.}
        \end{itemize}
        The following rules apply to the \textit{voters configuration} item:
        \begin{itemize}
            \item It is written by the \textit{Election Administrator} $\mathcal{E}$.
            \item It has as \textit{parent} the previous configuration item.
            \item There is one single \textit{voters configuration} item on the entire bulletin board.
        \end{itemize}
    
    \item
        The \textit{voter configuration} is an item that describes a change in the list of voters. It can be used either to add a new voter or to update an existing voter. This item can be present when the voter registration mode is \textbf{pre-election}. The structure of it consists of the \textbf{metadata} and the following attributes:
        \begin{itemize}
            \item the voter identifier $vID_i$,
            \item the public key $Y_i$,
            \item the enabled flag. \textcolor{orange}{This is so the election administrators is able to disable a voter if necessary. This information is public to maintain end to end verifiability.}
        \end{itemize}
        The following rules apply to the \textit{voter configuration} item:
        \begin{itemize}
            \item It is written by the \textit{Election Administrator} $\mathcal{E}$.
            \item It has as \textit{parent} the previous configuration item.
            \item If the voter identifier $vID$ has been previously defined on the bulletin board, then this item updates that voter's configuration.
            \item Otherwise, this item defines a new eligible voter.
            \item \textcolor{orange}{What attributes of a voter can be changed?}
            \item \textcolor{orange}{What if a voter is disabled after having voted?}
        \end{itemize}
    
    \item
        The \textit{voter authorization configuration} is the item describing the configuration of the \textit{Voter Authorizer} $\mathcal{A}$ and \textit{Identity Providers} $\{\mathcal{I}_1, ..., \mathcal{I}_{n_\mathrm{i}}\}$, where $n_\mathrm{i}$ is the number of providers. This item is present when the voter registration mode is \textbf{on-demand}. The structure of it consists of the \textbf{metadata} and the following attributes:
        \begin{itemize}
            \item the public key of the \textit{Voter Authorizer} $Y_\mathcal{A}$,
            \item a list that contains all \textit{Identity Providers} defined by their public keys $\{Y_{\mathcal{I}_1}, ..., Y_{\mathcal{I}_{n_\mathrm{i}}}\}$.
        \end{itemize}
        The following rules apply to the \textit{voter authorization configuration} item:
        \begin{itemize}
            \item It is written by the \textit{Election Administrator} $\mathcal{E}$.
            \item It has as \textit{parent} the previous configuration item.
            \item The first \textit{voter authorization configuration} item defines the initial configuration of the \textit{Voter Authorizer} and the \textit{Identity Providers}.
            \item The following \textit{voter authorization configuration} items define updates that are made to the configuration of the \textit{Voter Authorizer} or the \textit{Identity Providers}.
            \item \textcolor{orange}{How does an update affect a running election?}
            \item \textcolor{orange}{How does an update affect a currently open voting session authorized by the old configuration of the \textit{Voter Authorizer}?}
        \end{itemize}
\end{enumerate}

\paragraph{Voting items}
\begin{enumerate}
    \setcounter{enumi}{7}
    \item
        The \textit{voter session} is the item that documents the fact that a new voter has been authorized to cast a vote. This item is present when the voter registration mode is \textbf{on-demand}. The structure of it consists of the \textbf{metadata} and the following voter attributes:
        \begin{itemize}
            \item the voter identifier $vID_i$,
            \item the public key of the \textit{Voter} $Y_i$,
            \item the authorization token $\sigma_{\mathrm{auth}, i}$ (see \Cref{sec: voter authentication procedure} for details) generated by the \textit{Voter Authorization} $\mathcal{A}$.
        \end{itemize}
        The following rules apply to the \textit{voter authorization configuration} item:
        \begin{itemize}
            \item It is written by the \textit{Digital Ballot Box} $\mathcal{D}$.
            \item It has as \textit{parent} the previous configuration item.
            \item A new item is generated whenever a voter is authorized to vote.
            \item \textcolor{orange}{What if a voter needs to be disabled?}
        \end{itemize}
    
    \item 
        The \textit{voter encryption commitment} is the item that secretly defines the encryption parameters chosen by the \textit{Voter}. The structure of it consists of the \textbf{metadata} and the commitment values $C$ to all cryptogram randomizers chosen by the \textit{Voter}. The following rules apply to the \textit{voter encryption commitment} item:
        \begin{itemize} 
            \item It is written by the \textit{Voter} $\mathcal{V}_i$.
            \item It has as \textit{parent} the \textit{voter session} item .
            \item A new item is generated whenever the \textit{Voter} encrypts a new ballot.
        \end{itemize}
    
    \item
        The \textit{server encryption commitment} is the item that secretly defines the encryption parameters chosen by the \textit{Digital Ballot Box}. It is generated as a response to the \textit{voter encryption commitment} being posted. The structure of it consists of the \textbf{metadata} and the commitment values $C$ to all cryptogram randomizers chosen by the \textit{Digital Ballot Box}. The following rules apply to the \textit{server encryption commitment} item:
        \begin{itemize}
            \item It is written by the \textit{Digital Ballot Box} $\mathcal{D}$.
            \item It has as \textit{parent} the \textit{voter encryption commitment} item.
            \item A new item is generated whenever a \textit{voter encryption commitment} is published.
        \end{itemize}
    
    \item
        The \textit{ballot cryptograms} is the item that contains the encrypted digital ballot. The structure of it consists of the \textbf{metadata} and the cryptograms that the ballot is made out of. The following rules apply to the \textit{ballot cryptograms} item:
        \begin{itemize}
            \item It is written by the \textit{Voter} $\mathcal{V}_i$.
            \item It has as \textit{parent} the \textit{server encryption commitment} item.
            \item A new item is generated whenever the \textit{Voter} submits a new encrypted digital ballot.
        \end{itemize}
    
    \item
        The \textit{cast request} is the item that documents the action of casting a previously submitted ballot. The structure of it consists only of \textbf{metadata}. The following rules apply to the \textit{cast request} item:
        \begin{itemize}
            \item It is written by the \textit{Voter} $\mathcal{V}_i$.
            \item It has as \textit{parent} the \textit{ballot cryptograms} item.
            \item A new item is generated whenever the \textit{Voter} chooses to cast a ballot.
        \end{itemize}
    
    \item
        The \textit{spoil request} is the item that documents the action of spoiling (or challenge the encryption of) a previously submitted ballot. The structure of it consists only of \textbf{metadata}. The following rules apply to the \textit{spoil request} item:
        \begin{itemize}
            \item It is written by the \textit{Voter} $\mathcal{V}_i$.
            \item It has as \textit{parent} the \textit{ballot cryptograms} item.
            \item A new item is generated whenever the \textit{Voter} chooses to spoil a ballot.
        \end{itemize}
\end{enumerate}

\paragraph{Hidden items}
\begin{enumerate}
    \setcounter{enumi}{13}
    \item
        The \textit{verification track start} is the initial item of the hidden verification track. It, basically, spawns the verification track of each ballot cryptogram. It is generated as a response to the \textit{ballot cryptograms} item being posted.  The structure of it consists only of \textbf{metadata}. The following rules apply to the \textit{verification track start} item:
        \begin{itemize}
            \item It is written by the \textit{Digital Ballot Box} $\mathcal{D}$.
            \item It has as \textit{parent} the \textit{ballot cryptograms} item.
            \item It is appended on the hidden verification track.
            \item A new item is generated whenever a new \textit{ballot cryptograms} item is appended on the bulletin board.
        \end{itemize}
    
    \item
        The \textit{verifier} is the item that defines the details of the \textit{External Verifier} $\mathcal{X}$. The structure of it consists of the \textbf{metadata} and the public key of the \textit{External Verifier} $Y_\mathcal{X}$. The following rules apply to the \textit{verifier} item:
        \begin{itemize}
            \item It is written by the \textit{External Verifier} $\mathcal{X}$.
            \item It has as \textit{parent} the \textit{verification track start} item.
            \item It is appended on the hidden verification track.
            \item A new item is generated whenever an \textit{External Verifier} is used to challenge a vote cryptogram.
        \end{itemize}
    
    \item
        The \textit{voter commitment opening} is the item that contain a part of the data (voter side) necessary for unpacking the spoiled encrypted ballot. This data is encrypted itself, such that only the \textit{External Verifier} can read it. The structure of it consists of the \textbf{metadata} and the encrypted data blob of the openings of the \textit{Voter's} commitment from the \textit{voter encryption commitment} item. The following rules apply to the \textit{voter commitment opening} item:
        \begin{itemize}
            \item It is written by the \textit{Voter} $\mathcal{V}_i$.
            \item It has as \textit{parent} the \textit{verifier} item.
            \item It is appended on the hidden verification track.
            \item A new item is generated whenever the \textit{Voter} chooses to challenge the vote cryptograms and confirms the collaboration with the \textit{External Verifier}.
        \end{itemize}
    
    \item
        The \textit{server commitment opening} is the item that contain the second part of the data (server side) necessary for unpacking the spoiled encrypted ballot. This data is encrypted itself, such that only the \textit{External Verifier} can read it. This item is generated as a response to the \textit{voter commitment opening} item being posted.The structure of it consists of the \textbf{metadata} and the encrypted data blob of the openings of the commitment of the \textit{Digital Ballot Box} from the \textit{server encryption commitment} item. The following rules apply to the \textit{server commitment opening} item:
        \begin{itemize}
            \item It is written by the \textit{Digital Ballot Box} $\mathcal{D}$.
            \item It has as \textit{parent} the \textit{voter commitment opening} item.
            \item It is appended on the hidden verification track.
            \item A new item is generated whenever the \textit{Voter} chooses to challenge the vote cryptograms and confirms the collaboration with the \textit{External Verifier}.
        \end{itemize}
\end{enumerate}

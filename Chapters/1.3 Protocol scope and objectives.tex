\subsection{Protocol scope and objectives}
Some of the core features of the election protocol include: voters vote remotely, votes are encrypted, the system uses threshold cryptography, votes are cryptographically shuffled to ensure anonymity, all essential processes are verifiable, and auditing can be performed on all system components throughout the election event. The system cannot prevent but detects fraud or unauthorized access.

Multiple election types are supported, such as a referendum, candidate, multiple choice, or ranked elections. Multiple result types are also supported. The protocol has support for write-in votes. Additionally, the system provides continuous turnout statistics.

The scope of the protocol covers an entire election event, starting from election configuration, voter authorization, vote casting, tallying, and auditing. Cryptographic algorithms are crucial in terms of the security and auditing features of the system, but there are many non-cryptographic processes necessary to conduct a safe election. This document describes an online election system. Users, i.e., election officials and voters, access the system through a web browser or a native app on an internet-connected device such as a PC, laptop, tablet, smartphone, etc.

The overall objective of the document is to describe, claim and argue the achievement of the following objectives of our protocol, which are described in \cref{sec: requirements}:
\begin{multicols}{2}
\begin{itemize}
    \item Mobility
    \item Vote \& go
    \item Transparency
    \item Multiple voting rounds
    \item Multiple election types
    \item Confirming selected options
    \item Support correcting mistakes
    \item Vote overwrites
    \item Individual verification
    \item Universal verification    
    \item Full audit
    \item Eligibility
    \item Privacy
    \item Anonymity
    \item Integrity
    \item Ent-to-end verifiability
    \item Receipt-freeness.
\end{itemize}
\end{multicols}

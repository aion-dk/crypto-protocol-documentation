\subsection{Voter authentication modes} \label{sec: voter authentication modes}
During the pre-election phase, the voter authorizer service $\mathcal{A}$ is loaded with a list of eligible voters $\boldsymbol{\mathcal{V}} = \{ \mathcal{V}_1, ..., \mathcal{V}_{n_\mathrm{v}} \}$, where $n_\mathrm{v}$ is the total number of voters. In order to be authorized to cast a vote on the bulletin board, a user has to authenticate to the voter authorizer as voter $\mathcal{V}_i$, with $i \in \{ 1, ..., n_\mathrm{v} \}$. Once authenticated and authorized (\cref{sec: voter authorization procedure}), voter $\mathcal{V}_i$ can interact directly with the digital ballot box in the voting protocol as described in \cref{sec: vote cryptogram generation process}.

The voting system supports two mutually exclusive voter authentication modes: \textbf{credential-based} and \textbf{identity-based}. Both names refer to the means of the authentication taking place. One involves proving possession of some credentials that have pre-established before the election starts, while the other involves proving ownership of some identity provided by a third party. Some actors mentioned in the authentication modes presented below are exclusive to that mode only (i.e. credentials authority $\mathcal{C}$ for \textbf{credential-based} mode and identity provider $\mathcal{I}$ for \textbf{identity-based} mode).


\subsubsection{Credential-based mode} \label{sec: credential-based mode}
For this mode, the voter authorizer configures a set of credentials authorities $\boldsymbol{\mathcal{C}} = \{\mathcal{C}_1, ..., \mathcal{C}_{n_\mathrm{c}}\}$, where $n_\mathrm{c}$ is the number of authorities, which are responsible for generating and distributing the voter credentials during the pre-election phase. Each credentials authority is supposed to use a distinct communication channel to distribute credentials to the voters (i.e. email, SMS or postal). Therefore, all voters have to be defined with contact information for each of the communication channels supported by all credentials authorities $\boldsymbol{\mathcal{C}}$.

All credentials are converted into private-public key pairs (as described in \cref{sec: voter credential distribution process}) by all credentials authorities, which then return to the voter authorizer all public keys of all voters. The voter authorizer aggregates all public keys for each voter to form their \textit{authentication public key}. 

When trying to authenticate to the voter authorizer, a voter must generate a \textit{proof of credentials}, which can be achieved by aggregating all credentials received from all credentials authorities. If the proof validates, the voter authorizer $\mathcal{A}$ authorizes voter $\mathcal{V}_i$ to interact with the digital ballot box for casting a ballot. The entire authorization process is decribed in \cref{sec: voter authorization procedure}.


\subsubsection{Identity-based mode} \label{sec: identity-based mode}
In this mode, the voter authorizer service lists a set of third party identity providers $\boldsymbol{\mathcal{I}} = \{ \mathcal{I}_1, ..., \mathcal{I}_{n_\mathrm{i}} \}$, where $n_\mathrm{i}$ is the number of them.

Voters have to authenticate with all identity providers $\boldsymbol{\mathcal{I}}$ and receive identity tokens from each of them. Then, in order to get authorized to cast a vote, a voter has to submit all identity tokens to the voter authorizer, which checks whether they relate to an eligible voter identity from $\boldsymbol{\mathcal{V}}$. The process is further described in \cref{sec: voter authorization procedure}.

Because the voting system has to integrate into third party identity providers, all voters have to be defined with distinct identities supported by all identitytproviders $\boldsymbol{\mathcal{I}}$.

For auditing purpose, the voter authorizer stores all identity tokens received for each successful authorization that it performed. This must be audited and validated during the administration auditing process, as described in \cref{sec: administration auditing process}.

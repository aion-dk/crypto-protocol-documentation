\subsection{Election properties} \label{sec: election properties}
\paragraph{Eligibility}
During the \textit{pre-election phase} (\Cref{sec: pre-election phase}), the \textit{election administrators} define the list of eligible voters \( \boldsymbol{\mathcal{V}} = \{\mathcal{V}_1, ..., \mathcal{V}_{n_\mathrm{v}}\} \), where $n_\mathrm{v}$ is the total number of voters, and the list of printing authorities \( \boldsymbol{\mathcal{P}} = \{\mathcal{P}_1, ..., \mathcal{P}_{n_\mathrm{p}}\} \), where $n_\mathrm{p}$ is the total number of printing authorities.

During the \textit{voter credential distribution process} (\Cref{sec: voter credential distribution process}), all \textit{printing authorities} assign a distinct public signature verification key $Y_i$ to each voter $\mathcal{V}_i$. At the same time, each \textit{printing authority} \( \mathcal{P}_j \in \boldsymbol{\mathcal{P}} \) distributes to each voter their voting credentials $x_{i,j}$, with \( i \in \{ 1, ..., n_\mathrm{v} \} \) and \( j \in \{ 1, ..., n_\mathrm{p} \} \) such that \( [\sum_{j=1}^{n_\mathrm{p}} x_{i,j}]G = Y_i \). After receiving all his election credentials $x_{i,j}$, voter $\mathcal{V}_i$ can compute his private signing key \( x_i = \sum_{j=1}^{n_\mathrm{p}} x_{i,j} \).

When submitting a vote, the voter $\mathcal{V}_i$ digitally signs the vote submission with his private signing key $x_i$ and the election system accepts the vote submission only if the digital signature matches the voter's public signature verification key $Y_i$. By following this process, we make sure that the vote submission was generated by somebody in possession of $x_i$, which could only be the voter.

One can notice that the printing authority $\mathcal{P}_j$ knows a part of the signing key of voter $\mathcal{V}_i$ but not enough. Therefore, we argue that Assembly Voting X has the \textit{eligibility property} on the assumption that there exist multiple printing authorities that do not communicate with each other during the election process.

\paragraph{No revealing of partial results // fairness}
By following the election protocol, it is guaranteed that:
\begin{itemize}
    \item the secrecy of the vote is preserved. No votes from the bulletin board are decrypted, therefore no connection between a vote and a voter can be made.
    \item no partial results are computed during the election process. A result is calculated only once, after the election phase has finished. This prevents influencing of the subsequent voters throughout the election period.
\end{itemize}

All votes, that are posted on the bulletin board, are encrypted using the ElGamal cryptosystem based on elliptic curve cryptography (more details in \Cref{app: mathematics} and \Cref{app: elgamal cryptosystem}). Moreover, using a \textit{t out of n} threshold encryption scheme (\Cref{sec: threshold ceremony}), we enforce that there is no single entity that can perform the decryption of any data from the public bulletin board, but instead, it is needed a group of minimum $t$ trustees to collaborate.

Therefore, we claim that Assembly Voting X has the \textit{confidentiality property} on the assumption that at least $t$ trustees are honest, with \( t > n / 2 \) and \( n > 2 \).

One can argue that, because the bulletin board data is public, somebody could save all the data for long enough until the elliptic curve cryptosystem will be broken, and so will be able to decrypt all the data contrarily to our protocol. This fact demonstrates that our system does not comply to the \textit{everlasting confidentiality property}. We take note of this fact and we accept it.

\paragraph{Anonymity}
This property is reached by implementing a mixnet of nodes (trustees) that sequentially shuffle the list of vote cryptograms in an indistinguishable way, before they get decrypted (section \ref{sec: mixing phase}).

Obviously, each trutsee knows the way it shuffled the list of cryptograms but it does not know how it was shuffled by the other trustees. Thus, it is important that trustees do not communicate with each other.

We claim that Assembly Voting X is an \textit{anonymous voting system} on the assumption that there is a set of multiple trustess out of which at least one is honest.

\paragraph{Integrity}
The integrity of the election is preserved in our system by publishing all events (vote submissions or system events) on the bulletin board. Moreover, the bulletin board has a \textit{blockchain-like} structure that guarantees that the history of the bulletin board never changes. Also, the voters act like miners of the blockchain whenever they submit a new vote cryptogram, by signing on the history of the blockchain.

Every time a new vote submission is appended on the bulletin board, the voter receives a vote receipt $\rho_i$ that contains a pointer to the item on the bulletin board, called \textit{the board hash value} $h_{\mathrm{b}, i}$. This value is computed based on the previous \textit{board hash value} $h_{\mathrm{b}, i-1}$, which is computed based on the one before, and so on, until it reaches the \textit{genesis hash}, which is 0. This means that every time a voter checks his voter receipt, the entire bulletin board history is validated.

We claim that Assembly Voting X achieves the \textit{integrity property} through the bulletin board construction.

\paragraph{Verifiability}
There are two levels of verifiability that can be performed by different actors. Some steps are individually verifiable (i.e. only the voter that is currently performing this step can verify that the process is happening correctly), such as:
\begin{itemize}
    \item verify that the vote is cast as intended
    \item verify that the vote is registered as cast
\end{itemize}
The rest of the steps from the election protocol are publicly verifiable:
\begin{itemize}
    \item the threshold ceremony
    \item the bulletin board history
    \item the integrity and eligibility of each vote submission
    \item the integrity of each system event
    \item the correctness of the cleansing procedure, mixing phase and decryption phase (verification that votes are counted as registered)
\end{itemize}

We claim that Assembly Voting X is a \textit{verifiable election system}.

% \subsection{Accountability}
\paragraph{Receipt-freeness}
During the \textit{vote cryptogram generation process} (described in \cref{sec: vote cryptogram generation process}), the voter receives from the \textit{Digital Ballot Box} $\mathcal{D}$ an empty cryptogram $e_0$ that he uses to generate his final vote cryptogram $e$ in order to encrypt his vote $M$. At the end of the process, the vote cryptogram $e$ would be equal to \( \mathbf{Enc}_{Y_\mathrm{enc}} (M, r_0 + r_1) \), where $r_0$ is known by the \textit{Digital Ballot Box} $\mathcal{D}$ and $r_1$ is known by the voter.

% The voter is convinced that $e_0$ is an empty cryptogram because of a interactive proof $PK_0$ generated between the voter and the \textit{Digital Ballot Box} $\mathcal{D}$.

% The empty cryptogram $e_0$ and the proof $PK_0$ are relevant only for the communication amongst the voter and the \textit{Digital Ballot Box} $\mathcal{D}$, to make sure that none of them has the entire randomness value $r_0 + r_1$ that is used to generate the cryptogram $e$. Therefore, $e_0$ and $PK_0$ are not included on the bulletin board, thus not publicly available.

After the vote cryptogram $e$ has been accepted on the bulletin board, the voter is not able to produce valid cryptographic evidence that $e$ is an encryption of $M$ referencing only the data that is publicly available, as he is not in possession of value $r_0$. More details are described in section \ref{app: proving the content of a cryptogram}.

Therefore, we claim that Assembly Voting X is a \textit{receipt-free} voting protocol.

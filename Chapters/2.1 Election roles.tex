\subsection{Election roles} \label{sec: election roles}
Multiple parties are involved in the election process. Each party represents a human with access to a computer or simply a process/software that follows a particular protocol. All these parties can be categorized into the following 9 types: 


\subsubsection{Election Administrator}
There exists one administrator that is responsible for setting up the election event and making updates to the configurations. The Election Administrator $\mathcal{E}$ is a single service that is used by all election officers to configure the election. Election officers are humans that use an internet connected device to access the Election Administrator service based on some pre-established set of credentials.

The Election Administrator $\mathcal{E}$ owns a key pair $(x_\mathcal{E}, Y_\mathcal{E})$ used for signing the election configuration and it is responsible for privately storing its private key $x_\mathcal{E}$.


\subsubsection{Voter}
There exists a list of eligible voters, each denoted $\mathcal{V}_i$, with \( i \in \{ 1, ..., n_\mathrm{v} \} \), where $n_\mathrm{v}$ is the total number of voters.  A voter is a human being that is allowed to cast a valid vote in this election. A voter needs to have access to a computer that has an internet connection. Voters are the ones to generate vote cryptograms.

In \textbf{credential-based} voter authentication mode (see \cref{sec: voter authentication modes}), voters are responsible for protecting their credentials $c_{i,j}$ received from all credentials authorities. Moreover, they need to protect their private key $x_i$ while a voting session is active.

In \textbf{identity-based} voter authentication mode voters are responsible for protecting their private key $x_i$ while a voting session is active.


\subsubsection{Credentials Authority}
The Credentials Authority role is relevant only in the \textbf{credential-based} voter authentication mode, which is described in \cref{sec: credential-based mode}.
    
There is a set of credentials authorities, each noted as $\mathcal{C}_i$, with \( i \in \{ 1, ..., n_\mathrm{c} \} \), where $n_\mathrm{c}$ is the total number of credentials authorities. A credentials authority is an institution, consisting of humans and software processes. Each of them is responsible for generating voter credentials and distribute them privately to the voters. It is recommended that each credentials authority should use a different communication channel for distributing credentials (e.g. e-mail, post, SMS). Credentials Authorities must delete the voter credentials after they have been distributed.


\subsubsection{Trustee}
There is a set of trustees, each denoted as $\mathcal{T}_i$, with \( i \in \{ 1, ..., n_\mathrm{t} \} \), where $n_\mathrm{t}$ is the total number of trustees. A trustee is a human controlling a computer that has the \textit{Trustee Application} installed. Trustees are responsible for preserving the privacy and the fairness of the election during the election phase by working together to build the election encryption key while safely storing their shares of the decryption key.

The Trustee $\mathcal{T}_i$ protects its own private key $x_{\mathcal{T}_i}$ and its share of the election decryption key $sx_i$.

Trustees are also responsible for preserving the anonymity of the election by shuffling the entire list of votes in an indistinguishable way before a result is computed.


\subsubsection{Identity Provider}
The Identity Provider role is relevant only in the \textbf{identity-based} voter authentication mode, which is described in \cref{sec: identity-based mode}.
    
There is a set of Identity Providers $\mathcal{I}_i$, with $i \in \{ 1, ..., n_\mathrm{i} \}$, where $n_\mathrm{i}$ is the total number of identity providers. An Identity Provider is a third party application responsible for authenticating a voter $\mathcal{V}$, on the fly, during the election phase. Identity Providers must follow the OIDC protocol.


\subsubsection{Voter Authorizer} 
The Voter Authorizer $\mathcal{A}$ is a service that is responsible for authorizing a Voter $\mathcal{V}$ after being authenticated. The voter authentication can be performed either by providing the correct voter credentials, or by authenticating with all Identity Providers, depending on the voter authentication mode (described in \cref{sec: voter authentication modes}).

The Voter Authorizer $\mathcal{A}$ is responsible for preserving the eligibility of the election by preventing any non-eligible voters from voting.

The Voter Authorizer $\mathcal{A}$ owns a key pair $(x_\mathcal{A}, Y_\mathcal{A})$ used for signing voter authorization and it is responsible for privately storing its private key $x_\mathcal{A}$.


\subsubsection{Digital Ballot Box} 
The Digital Ballot Box $\mathcal{D}$ is the communication central unit such that all other parties push/pull data to/from it. It is a single service, publicly accessible over the internet. The Digital Ballot Box $\mathcal{D}$ has a bulletin board that contains all the public information about an election. The data that is published on the bulletin board is thoroughly described in \cref{sec: bulletin board event types}, but is can simply be summarized under the following categories: 
\begin{itemize}
    \item election configuration data, that has to be set up during the pre-election phase (described in \cref{sec: pre-election phase}). Most of this data is generated by the Election Administrator $\mathcal{E}$.
    \item voting data, which is being populated during the election phase (described in \cref{sec: election phase}). Most of this data is generated by voters in collaboration with the Digital Ballot Box $\mathcal{D}$.
    \item result data, which is collected during the post-election phase (described in \cref{sec: post-election phase}). This includes mixing and decryption files that are generated by Trustees and enable the result to be verifiable.
\end{itemize}

The Digital Ballot Box $\mathcal{D}$ owns a key pair $(x_\mathcal{D}, Y_\mathcal{D})$ used for signing data on the bulletin board and it is responsible for privately storing its private key $x_\mathcal{D}$.


\subsubsection{External Verifier}
The External Verifier $\mathcal{X}$ is an auditing tool that is used by voters if they choose to perform the process of challenging a vote cryptogram (\cref{sec: challenging a vote cryptogram}). The External Verifier enables voters to check that their vote has been correctly encrypted and stored on the bulletin board. During this process, the External Verifier $\mathcal{X}$ will generate a new key pair $(x_\mathcal{X}, Y_\mathcal{X})$ and it has to protect its private key $x_\mathcal{X}$.

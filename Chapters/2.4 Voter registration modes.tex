\subsection{Voter registration modes} \label{sec: voter registration modes}
AVX supports two mutually exclusive voter registration modes: \textbf{pre-election} and \textbf{on-demand}. Both names refer to when voters are registered in the election in their respective mode. The mentioned actors ($\mathcal{P}$, $\mathcal{I}$, and $\mathcal{A}$) in the subsections below are exclusive to the mode mentioned in.


\subsubsection{Pre-election mode} \label{sec: pre-election mode}
For this configuration there is a set of \textit{Printing Authorities} $\{\mathcal{P}_1, ..., \mathcal{P}_{n_\mathrm{p}}\}$, where $n_\mathrm{p}$ is the number of authorities, which are responsible for generating and distributing the voter credentials. There exists a list of authorized eligible voters \( \boldsymbol{\mathcal{V}} = (\mathcal{V}_1, ..., \mathcal{V}_{n_\mathrm{v}}) \) who can be authenticated to vote by providing their credentials, where $n_\mathrm{v}$ is the total number of voters.


\subsubsection{On-demand mode} \label{sec: on-demand mode}
In this configuration there is a list of eligible voters \( \boldsymbol{\mathcal{V}} = \{\mathcal{V}_1, ..., \mathcal{V}_{n_\mathrm{v}}\} \) but when the system enters the election phase these potential voters need to authenticate $\mathcal{V}_i$ through the \textit{Identity Providers} $\boldsymbol{\mathcal{I}} = \{\mathcal{I}_1, ..., \mathcal{I}_{n_\mathrm{i}}\}$, where $n_\mathrm{i}$ is the total number of identity providers. With successful authentication, the identity is checked by the \textit{Voter Authorization Service} $\mathcal{A}$ to see if it is part of the list of pre-approved eligible voters (i.e. $\mathcal{V}_i \in \boldsymbol{\mathcal{V}}$). If eligible, then the \textit{Voter Authorization Service} $\mathcal{A}$ returns an authorization token that will give authorize the voter to vote.

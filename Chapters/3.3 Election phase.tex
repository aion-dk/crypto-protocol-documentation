\subsection{Election phase} \label{sec: election phase}
The election phase lasts from the start date until the end date of a voting round. During this time, any voter $\mathcal{V}_i \in \boldsymbol{\mathcal{V}}$ can cast a valid digital ballot by performing the following steps:
\begin{itemize}
    \item authenticate and get authorized to cast a digital ballot on the bulletin board as described in \cref{sec: voter authentication procedure},
    \item select vote choices and prepare them for encryption as described in \cref{sec: mapping vote options on the elliptic curve},
    \item encrypt the ballot following the process from \cref{sec: vote cryptogram generation process},
    \item optionally, perform an audit/verification on the encrypted ballot as described in \cref{sec: challenging a vote cryptogram} and
    \item finally, cast the encrypted ballot and obtain a vote confirmation receipt as in \cref{sec: vote confirmation receipt}.
\end{itemize}


\subsubsection{Voter authentication procedure} \label{sec: voter authentication procedure}
A voter $\mathcal{V}_i$ is considered authorized to cast a digital ballot when it is in possession of a secret signing key $x_i$ that corresponds to an eligible public key $Y_i$ from the bulletin board. This is achieved differently depending on the voter authentication mode.


\paragraph{When pre-election voter authentication mode}\mbox{}\\
Each voter $\mathcal{V}_i$ has to input in the voting application all credentials received from the Printing Authorities $x_{i, j}$, with $j \in \{ 1, ..., n_\mathrm{p} \}$. The application combines all credentials to form the voter's secret signing key $x_i = \sum_{j=1}^{n_\mathrm{p}} x_{i, j} \pmod q$. The associated public key $Y_i$ has been computed and included on the bulletin board in the pre-election phase.


\paragraph{When on-demand voter athentication mode}\mbox{}\\
Each voter $\mathcal{V}_i$ has to follow the protocol from \cref{fig: voter authentication protocol} in order to get authorized to cast a digital ballot on the bulletin board. Specifically, the voter must authenticate and get identity tokens $\sigma_{\mathrm{id}, j}$ from all of the Idenitity Providers $\mathcal{I}_j \in \boldsymbol{\mathcal{I}}$ that have been configured by the Voter Authorizer in the pre-election phase.

Then, the voting application generates a key pair $(x_i, Y_i) \gets \mathsf{KeyGen}()$ (\cref{alg: key gen}) and forwards all identity tokens $\{ \sigma_{\mathrm{id}, 1}, ..., \sigma_{\mathrm{id}, n_\mathrm{i}} \}$ and the public key $Y_i$ to the Voter Authorizer service $\mathcal{A}$ proving the identity of the voter $\mathcal{V}_i$.

If the Voter Authorizer service can validate all identity tokens and the voter is eligible, i.e. $\mathcal{V}_i \in \boldsymbol{\mathcal{V}}$, then it will authorize the use of the public key $Y_i$ for the voter $\mathcal{V}_i$. This is done by the Voter Authorizer $\mathcal{A}$ interacting with the Digital Ballot Box $\mathcal{D}$ in the protocol \ref{fig: write item on the bulletin board protocol} $\mathtt{WriteOnBoard}(\mathcal{A}, m_\mathrm{vs}, c_\mathrm{vs}, p_\mathrm{vs})$ to write a voter session item $b_\mathrm{vs}$ on the bulletin board as the $k^\mathrm{th}$ item, according to the rules specified in \cref{app: bulletin board item types}, where $m_\mathrm{vs} =$ "voter session", the parent $p_\mathrm{vs}$ is the address of the latest configuration item and the content $c_\mathrm{vs}$ consists of the voter identifier, the public key $Y_i$, and the authentication fingerprint which is the hash value of all identity tokens received form the voter. 

The Voter Authorizer returns to the voter the voter session item $b_\mathrm{vs}$ and its receipt $\rho_k$, as received from the Digital Ballot Box. From this point on, the voter can interact directly with the Digital Ballot Box as the identity $\mathcal{V}_i$.

The Voter Authorizer Service $\mathcal{A}$ stores a link between the voter identity $\mathcal{V}_i$ and all related identity tokens for the purpose of the private auditing process in the post-election phase as described in \cref{sec: private auditing process}. This link is stored privately by the Voter Authorizer service due to the fact that the identity tokens likely contain personal information that must not be disclosed on the public bulletin board.

\clearpage
\begin{landscape}
\begin{figure}[ht]
    \centering
    \begin{tikzpicture}[framed,
            node distance=0,
            % every node/.style={draw},
            ]{
        
        % Actors
        \node[title_3, text width=191, above, anchor=north east] (v) {
            \textbf{Voter $\mathcal{V}_i$}};
        \node[title_3, text width=187, right=of v] (va) {
            \textbf{Voter Authorizer $\mathcal{A}$}};
        \node[title_3, text width=141, right=of va] (ip) {
            \textbf{Identity Provider $\mathcal{I}_j$}};
        
        % internal knowledge
        \node[internal_3, text width=191, below=of v] (v_ik) {
            internal knowledge: $Y_\mathcal{A}$, $Y_\mathcal{D}$, $\{ Y_{\mathcal{I}_1}, ..., Y_{\mathcal{I}_{n_\mathrm{i}}} \}$
            };
        \node[internal_3, text width=187, below=of va] (va_ik) {
            internal knowledge: $x_\mathcal{A}$, $\boldsymbol{\mathcal{V}}$, $\{ Y_{\mathcal{I}_1}, ..., Y_{\mathcal{I}_{n_\mathrm{i}}} \}$
            };
        \node[internal_3, text width=141, below=of ip] (ip_ik) {
            internal knowledge: $x_{\mathcal{I}_j}$
            };
        
        % All content
        \node[below=of v_ik] (top){};
        \node[arrow_v2, towards_right, between={v.center}{ip.center}] at (top.south -| v.center) (a_1) {
            authenticate as $\mathcal{V}_i$
            };
        \node[process_v2, keep_middle, after_arrow] at (a_1.south -| ip.center) (ip_1) {
            $\sigma_{\mathrm{id}, j} \gets \mathsf{Sign}(x_{\mathcal{I}_j}; \mathcal{V}_i)$
            };
        \node[arrow_v2, towards_left, between={v.center}{ip.center}] at (ip_1.south -| ip.center) (a_2) {
            $\sigma_{\mathrm{id}, j}$
            };
        \node[process_v2, keep_left, after_arrow] at (a_2.south -| v.west) (v_1) {
            verify that $\mathsf{SigVer}(Y_{\mathcal{I}_j}, \sigma_{\mathrm{id}, j}; \mathcal{V}_i)$
            };
        \node[banner_v2, keep_left, after_arrow, between={v.west}{ip.east}] at (v_1.south -| v.west) (b_1) {
            when successfully authenticated with all $\mathcal{I}_j \in \boldsymbol{\mathcal{I}}$ and received $\{ \sigma_{\mathrm{id}, 1}, ..., \sigma_{\mathrm{id}, n_\mathrm{i}} \}$
            };
        \node[process_v2, keep_middle, after_arrow] at (b_1.south -| v.center) (v_2) {
            $(x_i, Y_i) \gets \mathsf{KeyGen}()$
            };
        \node[arrow_v2, towards_right, between={v.center}{va.center}] at (v_2.south -| v.center) (a_3) {
            $Y_i, \{ \sigma_{\mathrm{id}, 1}, ..., \sigma_{\mathrm{id}, n_\mathrm{i}} \}$
            };
        \node[process_v2, keep_middle, after_arrow] at (a_3.south -| va.center) (va_1) {
            verify that $\mathcal{V}_i \in \boldsymbol{\mathcal{V}}$ and $\mathsf{SigVer}(Y_{\mathcal{I}_j}, \sigma_{\mathrm{id}, j}; \mathcal{V}_i)$, with $j \in \{1, ..., n_\mathrm{i}\}$ then: \\ [5pt]
            $\phi \gets \mathcal{H}(\sigma_{\mathrm{id}, 1} || ... || \sigma_{\mathrm{id}, n_\mathrm{i}})$, $c_\mathrm{vs} \gets (\mathcal{V}_i, Y_i, \phi)$ \\
            $m_\mathrm{vs} \gets$ "voter session", $p_\mathrm{vs} \gets$ the address of the latest config item \\ [5pt]
            \scriptsize perform protocol \ref{fig: write item on the bulletin board protocol} to write $b_\mathrm{vs}$ as the $k^\mathrm{th}$ item \\ [-2pt]
            \scriptsize $(b_\mathrm{vs}, h_{k-1}, \rho_k) \gets \mathtt{WriteOnBoard}(\mathcal{A}, m_\mathrm{vs}, c_\mathrm{vs}, p_\mathrm{vs})$ \\ [5pt]
            internally store the tuple for auditing: $(\mathcal{V}_i, \phi, \{ \sigma_{\mathrm{id}, 1}, ..., \sigma_{\mathrm{id}, n_\mathrm{i}} \})$
            };
        \node[arrow_v2, towards_left, between={v.center}{va.center}, anchor=north west] at (va_1.south -| v.center) (a_4) {
            $b_\mathrm{vs} = (m_\mathrm{vs}, c_\mathrm{vs}, \mathcal{A}, \sigma_\mathcal{A}, t_k, p_\mathrm{vs}, h_k)$, $\rho_k$, $h_{k-1}$
            };
        \node[process_v2, keep_left, after_arrow] at (a_4.south -| v.west) (v_3) {
            verify that $b_\mathrm{vs}$ is valid according to validations of protocol \ref{fig: write item on the bulletin board protocol} \\
            and $c_\mathrm{vs} = (\mathcal{V}_i, Y_i, \mathcal{H}(\sigma_{\mathrm{id}, 1} || ... || \sigma_{\mathrm{id}, n_\mathrm{i}}))$
            };
        
        % Arrows and lines
        \draw[dashed] (v.south west)--(ip.south east);    
        \draw[dashed] (v_ik.west |- va_ik.south)--(ip_ik.east |- va_ik.south);
        
        \draw[dotted] (b_1.north west)--(b_1.north east);
        \draw[dotted] (b_1.south west)--(b_1.south east);
        
        \draw[dotted] (v_ik.south)--(v_1.north -| v.center);
        \draw[dotted] (v_1.south -| v.center)--(b_1.north -| v.center);
        \draw[dotted] (b_1.south -| v.center)--(v_2.north -| v.center);
        \draw[dotted] (v_2.south -| v.center)--(v_3.north -| v.center);

        \draw[dotted] (a_3.south -| va.center)--(va_1.north -| va.center);
        \draw[dotted] (va_1.south -| va.center)--(a_4.south -| va.center);

        \draw[dotted] (a_1.south -| ip.center)--(ip_1.north -| ip.center);
        \draw[dotted] (ip_1.south -| ip.center)--(a_2.south -| ip.center);
        }
    \end{tikzpicture}
    \caption{Voter authentication protocol}
    \label{fig: voter authentication protocol}
\end{figure}
\end{landscape}


\clearpage
\subsubsection{Mapping vote options on the Elliptic Curve} \label{sec: mapping vote options on the elliptic curve}
An expressed vote (a vote in plain text) must be able to be converted, deterministically, into an elliptic curve point in order to be used in our cryptographic protocols. Additionally, a point from the elliptic curve must be able to be turned back to a plain text vote, if the point has been constructed from a plain text. Depending on the election type (referendum, simple election, multiple choice election, STV election), the plain text vote can be constructed in different ways, for example a simple string, or an array of integers or even a complex data structure. Regardless of the vote encoding rules, the plin text vote is converted into its byte representation $\boldsymbol{b} \in \mathbb{B}^*$.

Next, $\boldsymbol{b}$ is converted into an elliptic curve point $M \gets \mathsf{Bytes2Point}(\boldsymbol{b})$ (\cref{alg: bytes to point}), which can be used further in the encryption mechanism described in \cref{sec: vote cryptogram generation process}. Thus, point $M$ is the representation of voter's vote choices in cryptographic form.

Recovering the byte array $\boldsymbol{b}$ from $M$ can be done by $\boldsymbol{b} \gets \mathsf{Point2Bytes}(M)$ (\cref{alg: point to bytes}), which can further be decoded into a plain text vote, depending on the vote encoding rules.


\subsubsection{Vote cryptogram generation process} \label{sec: vote cryptogram generation process}
During the vote cryptogram generation process, the voting application collaborates with the digital ballot box $\mathcal{D}$ for generating the cryptogram $e$ that represents the encryption of the vote $M$. This process results in the fact that neither the voter $\mathcal{V}_i$ nor the digital ballot box $\mathcal{D}$ will be in possession of whole randomizer value $r$ used in the generation process of cryptogram $e$ (recall from \cref{app: elgamal cryptosystem} that $e = \mathsf{Enc}(Y_\mathrm{enc}, M; r)$). That is achieved by both the voter and the digital ballot box building up the randomizer but none of them knowing its entire value. It is important for the voter to not know this value in order to not be able to produce cryptographic evidence of the way he voted (as in \cref{app: proving the content of a cryptogram}), thus achieving \textit{receipt freeness}. The entire process cosists of commiting to the encryption randomizers which is presented in \cref{fig: encryption commitments submission protocol} and submitting the encrypted ballot as presented in \cref{fig: encrypted ballot submission protocol}.

\begin{figure}[ht]
    \centering
    \begin{tikzpicture}[framed,
            node distance=0,
            % every node/.style={draw}
            ]{
            
        % Actors
        \node[title, above, anchor=north east] (v) {
            \textbf{Voter $\mathcal{V}_i$}};
        \node[title, right=of v] (dbb) {
            \textbf{Digital Ballot Box $\mathcal{D}$}};
        
        % internal knowledge
        \node[internal, below=of v] (v_ik) {
            internal knowledge: $x_i$, $Y_\mathcal{D}$, $b_\mathrm{vs}$
            };
        \node[internal, below=of dbb] (dbb_ik) {
            internal knowledge: $x_\mathcal{D}$, $Y_\mathrm{enc}$, \\
            $\boldsymbol{b} = \{ b_1, ..., b_{k-1} \}$
            };
        
        %All content
        \node[process_v2, keep_left, first] at (dbb_ik.south -| v_ik.west) (v_1) {
            $r_\mathrm{v} \in_\mathrm{R} \mathbb{Z}_q$, $s_\mathrm{v} \in_\mathrm{R} \mathbb{Z}_q$ \\
            $c_\mathrm{vec} \gets \mathsf{Com}(r_\mathrm{v}, s_\mathrm{v})$, $p_\mathrm{vec} \gets$ the address of $b_\mathrm{vs}$ \\
            $m_\mathrm{vec} \gets$ "voter encryption commitment"
            };
        \node[banner_v2, keep_left, after_arrow, between={v.west}{dbb.east}] at (v_1.south west) (b_1) {
            perform protocol \ref{fig: write item on the bulletin board protocol} to write $b_\mathrm{vec}$ as the $k^\mathrm{th}$ item \\
            $(b_\mathrm{vec}, h_{k-1}, \rho_k) \gets \mathtt{WriteOnBoard}(\mathcal{V}_i, m_\mathrm{vec}, c_\mathrm{vec}, p_\mathrm{vec})$
            };
        \node[process_v2, keep_right, after_arrow] at (b_1.south east) (dbb_1) {
            $r_\mathrm{d} \in_\mathrm{R} \mathbb{Z}_q$, $s_\mathrm{d} \in_\mathrm{R} \mathbb{Z}_q$ \\
            $c_\mathrm{sec} \gets \mathsf{Com}(r_\mathrm{d}, s_\mathrm{d})$, $p_\mathrm{sec} \gets$ the address of $b_\mathrm{vec}$ \\
            $m_\mathrm{sec} \gets$ "server encryption commitment" \\ [7pt]
            \scriptsize perform protocol \ref{fig: write item on the bulletin board protocol} to write $b_\mathrm{sec}$ as the $(k+1)^\mathrm{th}$ item \\ [-2pt]
            \scriptsize $(b_\mathrm{sec}, h_k, \rho_{k+1}) \gets \mathtt{WriteOnBoard}(\mathcal{D}, m_\mathrm{sec}, c_\mathrm{sec}, p_\mathrm{sec})$ \\ [7pt]
            $e_\mathrm{d} \gets \mathsf{Enc}(Y_\mathrm{enc}, \mathcal{O}; r_\mathrm{d})$
            };
        \node[arrow_v2, towards_left, between={v.center}{dbb.center}, anchor=north west] at (dbb_1.south -| v.center) (a_1) {
            $(b_\mathrm{vec}, h_{k-1}, \rho_k)$, $(b_\mathrm{sec}, h_k, \rho_{k+1})$, $e_\mathrm{d}$
            };
        \node[process_v2, keep_left, after_arrow] at (a_1.south -| v.west) (v_2) {
            verify that $b_\mathrm{vec}$ and $b_\mathrm{sec}$ are valid \\
            according to validations of protocol \ref{fig: write item on the bulletin board protocol}
            };
        
        % Arrows and lines
        \draw[dashed] (v.south west)--(dbb.south east);    
        \draw[dashed] (dbb_ik.south -| v.west)--(dbb_ik.south east);

        \draw[dotted] (b_1.north west)--(b_1.north east);
        \draw[dotted] (b_1.south west)--(b_1.south east);

        \draw[dotted] (v_1.south -| v.center)--(b_1.north -| v.center);
        \draw[dotted] (b_1.south -| v.center)--(v_2.north -| v.center);

        \draw[dotted] (b_1.south -| dbb.center)--(dbb_1.north -| dbb.center);
        \draw[dotted] (dbb_1.south -| a_1.east)--(a_1.south east);
        }
    \end{tikzpicture}
    \caption{Encryption commitments submission protocol}
    \label{fig: encryption commitments submission protocol}
\end{figure}

\begin{figure}[ht]
    \centering
    \begin{tikzpicture}[framed,
            node distance=0,
            % every node/.style={draw}
            ]{
            
        % Actors
        \node[title, above, anchor=north east] (v) {
            \textbf{Voter $\mathcal{V}_i$}};
        \node[title, right=of v] (dbb) {
            \textbf{Digital Ballot Box $\mathcal{D}$}};
        
        % internal knowledge
        \node[internal, below=of v] (v_ik) {
            internal knowledge: $x_i$, $Y_\mathcal{D}$, $Y_\mathrm{enc}$, \\
            $b_\mathrm{vec}$, $b_\mathrm{sec}$, $M$, $r_\mathrm{v}$, $e_\mathrm{d}$
            };
        \node[internal, below=of dbb] (dbb_ik) {
            internal knowledge: $x_\mathcal{D}$, $Y_\mathrm{enc}$, $r_\mathrm{d}$, \\
            $\boldsymbol{b'} = \{ b_1, ..., b_{k'-1} \}$
            };
        
        %All content
        \node[process_v2, first, keep_left] at (v_ik.south west) (v_1) {
            $e_\mathrm{v} \gets \mathsf{Enc}(Y_\mathrm{enc}, M; r_\mathrm{v})$ \\
            $e \gets \mathsf{HomAdd}(e_\mathrm{d}, e_\mathrm{v})$ \\
            $PK \gets \mathsf{DLProve}(r_\mathrm{v}, \{ G \})$ \\
            $c_\mathrm{bc} \gets e$, $p_\mathrm{bc} \gets$ address of $b_\mathrm{sec}$ \\
            $m_\mathrm{bc} \gets$ "ballot cryptograms"
            };
        \node[arrow_v2, towards_right, between={v.center}{dbb.center}] at (v_1.south -| v.center) (a_1) {
            $e = (R, C)$, $PK$
            };
        \node[process_v2, keep_right, after_arrow] at (a_1.south -| dbb.east) (dbb_1) {
            verify that $\mathsf{DLVer}(PK, \{ G \}; \{ R - [r_\mathrm{d}]G \})$
            };
        \node[banner_v2, keep_left, after_arrow, between={v.west}{dbb.east}] at (dbb_1.south -| v.west) (b_1) {
            perform protocol \ref{fig: write item on the bulletin board protocol} to write $b_\mathrm{bc}$ as the ${k'}^\mathrm{th}$ item \\
            $(b_\mathrm{bc}, h_{k'-1}, \rho_{k'}) \gets \mathtt{WriteOnBoard}(\mathcal{V}_i, m_\mathrm{bc}, c_\mathrm{bc}, p_\mathrm{bc})$
            };
        \node[process_v2, keep_right, after_arrow] at (b_1.south east) (dbb_2) {
            $c_\mathrm{vts} \gets \perp$, $p_\mathrm{vts} \gets$ the address of $b_\mathrm{bc}$ \\
            $m_\mathrm{vts} \gets$ "verification track start" \\ [7pt]
            \scriptsize perform protocol \ref{fig: write item on the bulletin board protocol} to write $b_\mathrm{vts}$ as the first \\ [-2pt]
            \scriptsize item on the hidden track introduced by $b_\mathrm{bc}$ \\ [-2pt]
            \scriptsize $(b_\mathrm{vts}, h_{k'}, \rho) \gets \mathtt{WriteOnBoard}(\mathcal{D}, m_\mathrm{vts}, c_\mathrm{vts}, p_\mathrm{vts})$
            };
        \node[arrow_v2, towards_left, between={v.center}{dbb.center}] at (dbb_2.south -| dbb.center) (a_2) {
            $(b_\mathrm{bc}, h_{k'-1}, \rho_{k'})$, $(b_\mathrm{vts}, h_{k'}, \rho)$
            };
        \node[process_v2, keep_left, after_arrow] at (a_2.south -| v.west) (v_2) {
            verify that $b_\mathrm{bc}$ and $b_\mathrm{vts}$ are valid \\
            according to validations of protocol \ref{fig: write item on the bulletin board protocol}
            };
        
        % Arrows and lines
        \draw[dashed] (v.south west)--(dbb.south east);    
        \draw[dashed] (dbb_ik.south -| v.west)--(dbb_ik.south east);

        \draw[dotted] (b_1.north west)--(b_1.north east);
        \draw[dotted] (b_1.south west)--(b_1.south east);

        \draw[dotted] (v_1.south -| v.center)--(b_1.north -| v.center);
        \draw[dotted] (b_1.south -| v.center)--(v_2.north -| v.center);

        \draw[dotted] (a_1.south east)--(dbb_1.north -| dbb.center);
        \draw[dotted] (dbb_1.south -| dbb.center)--(b_1.north -| dbb.center);
        \draw[dotted] (b_1.south -| dbb.center)--(dbb_2.north -| dbb.center);
        \draw[dotted] (dbb_2.south -| dbb.center)--(a_2.south east);
        }
    \end{tikzpicture}
    \caption{Encrypted ballot submission protocol}
    \label{fig: encrypted ballot submission protocol}
\end{figure}

The generation process begins by the voting application generating its encryption randomizer $r_\mathrm{v} \in_\mathrm{R} \mathbb{Z}_q$ and computing a commitment to it $c_\mathrm{v} \gets \mathsf{Com}(r_\mathrm{v}, s_\mathrm{v})$, where $s_\mathrm{v} \in_\mathrm{R} \mathbb{Z}_q$. Then, the voting application interacts with the digital ballot box in the protocol \ref{fig: write item on the bulletin board protocol} $\mathtt{WriteOnBoard}(\mathcal{V}_i, m_\mathrm{vec}, c_\mathrm{vec}, p_\mathrm{vec})$ in order to append the vote encryption commitment item $b_\mathrm{vec}$ on the board, where $m_\mathrm{vec} =$ "voter encryption commitment", the content $c_\mathrm{vec}$ consists of the commitment $c_\mathrm{v}$ and the parrent $p_\mathrm{vec}$ is the address of the voter session item as received in \cref{sec: voter authentication procedure}. Note that before appending the new item, the bulletin board consists of items $\boldsymbol{b} = \{ b_1, ..., b_{k-1} \}$, therefore $b_\mathrm{vec}$ becoming the $k^\mathrm{th}$ item.

After publishing the voter encryption commitment item on the bulletin board, the digital ballot box imediately geterates its own set of encryption randomizer $r_\mathrm{d} \in_\mathrm{R} \mathbb{Z}_q$ and commitment $c_\mathrm{d} \gets \mathsf{Com}(r_\mathrm{d}, s_\mathrm{d})$, where $s_\mathrm{d} \in_\mathrm{R} \mathbb{Z}_q$. Then, it self writes a server encryption commitment item $b_\mathrm{sec}$ on the board by running protocol \ref{fig: write item on the bulletin board protocol} $\mathtt{WriteOnBoard}(\mathcal{D}, m_\mathrm{sec}, c_\mathrm{sec}, p_\mathrm{sec})$, where $m_\mathrm{sec} =$ "server encryption commitment", the content $c_\mathrm{sec}$ consists of its commitment $c_\mathrm{d}$ and the parrent $p_\mathrm{sec}$ is the address of the voter encryption commitment item $b_\mathrm{vec}$.

Next, the digital ballot box returns to the voting application both items $b_\mathrm{vec}$ and $b_\mathrm{sec}$ together with their respective receipts and previous board hash values, according to the protocol \ref{fig: write item on the bulletin board protocol} described in \cref{sec: writing on the bulletin board} and the empty cryptogram $e_\mathrm{d}$, i.e. an encryption of the neutral point $\mathcal{O}$ using the encryption randomizer $r_\mathrm{d}$. The voting application performs the validation of the board items $b_\mathrm{vec}$ and $b_\mathrm{sec}$ according to the protocol \ref{fig: write item on the bulletin board protocol} and continues if sucessful.

After both parties have published their encryption commitment items, as presented in \cref{fig: encrypted ballot submission protocol}, the voting application encrypts the voter's encoded vote $M$ (as constructed in \cref{sec: mapping vote options on the elliptic curve}) by computing $e_\mathrm{v} \gets \mathsf{Enc}(Y_\mathrm{enc}, M; r_\mathrm{v})$ (\cref{alg: enc}). This gets further combined with the empty cryptogram received form the digital ballot box, to produce the voter's final ballot cryptogram $e \gets \mathsf{HomAdd}(e_\mathrm{v}, e_\mathrm{d})$ (\cref{alg: hom add}). The voting application also computes a proof of correct encryption $PK \gets \mathsf{DLProve}(r_\mathrm{v})$ (\cref{alg: dl prove}) that confirms the fact that the empty cryptogram $e_\mathrm{d}$ has been used in computation of the final ballot cryptogram $e$.

Finally, the voting application interacts with the digital ballot box in the protocol $\mathtt{WriteOnBoard}(\mathcal{V}_i, m_\mathrm{bc}, c_\mathrm{bc}, p_\mathrm{bc})$ (protocol \ref{fig: write item on the bulletin board protocol}) in order to append the ballot cryptogram item $b_\mathrm{bc}$ on the board, where $m_\mathrm{bc} =$ "ballot cryptograms", the content $c_\mathrm{bc}$ consists of the cryptogram $e$ and the parrent $p_\mathrm{bc}$ is the address of the server encryption commitment item $b_\mathrm{sec}$. Note that this time, the bulletin board consists of items $\boldsymbol{b'} = \{ b_1, ..., b_{k'-1} \}$, where $k' \geq k$ as more items could have been appended by other voters in between protocols from \cref{fig: encryption commitments submission protocol} and \cref{fig: encrypted ballot submission protocol}, therefore $b_\mathrm{bc}$ becoming the ${k'}^\mathrm{th}$ item.

Additionally, the voting application submits the proof of correct encryption $PK$ to the digitla ballot box, which performs protocol \ref{fig: write item on the bulletin board protocol} if $\mathsf{DLVer}(PK, \{ G \}, \{ R - [r_\mathrm{d}]G \})$ (\cref{alg: dl ver}) succeeds, where the content of the item $c_\mathrm{bc} = e = (R, C)$.

After publishing the ballot cryptograms item on the bulletin board, the digital ballot box imediately self writes a verification track start item $b_\mathrm{vts}$ on the hidden track of the bulletin board by running $\mathtt{WriteOnBoard}(\mathcal{D}, m_\mathrm{vts}, c_\mathrm{vts}, p_\mathrm{vts})$ (protocol \ref{fig: write item on the bulletin board protocol}), where $m_\mathrm{vts} =$ "verification track start", the content $c_\mathrm{sec}$ is empty and the parrent $p_\mathrm{vts}$ is the address of the ballot cryptogram item $b_\mathrm{bc}$. 

Next, the digital ballot box returns to the voting application both items $b_\mathrm{bc}$ and $b_\mathrm{vts}$ together with their respective receipts and previous board hash values, according to the protocol \ref{fig: write item on the bulletin board protocol} described in \cref{sec: writing on the bulletin board}. The voting application performs the validation of the two board items according to the protocol \ref{fig: write item on the bulletin board protocol}.

Note that, the cryptogram $e$ is actually equivalent to $\mathsf{Enc}(Y_\mathrm{enc}, M; r)$, where $r = r_\mathrm{v} + r_\mathrm{d}$. Both the voter and the digital ballot box know part of the randomizer value, $r_\mathrm{v}$ and $r_\mathrm{d}$ respectively, but neither of them knows the combined value $r$.


\color{orange}
\subsubsection{Not used - to be moved or removed}
\begin{enumerate}
    \item The \textit{Voter} $\mathcal{V}_i$ asks the \textit{Digital Ballot Box} $\mathcal{D}$ for the latest election configuration items which presents the means of voter authentication and all contest options.
    
    \item At this point, the \textit{Voter} $\mathcal{V}_i$ has two options:
    \begin{itemize}
        \item If he has enough trust in the voting application, the \textit{Voter} $\mathcal{V}_i$ can mark the vote cryptogram $e$ as "cast" on the bulletin board. Then, the process continues on to \Cref{itm: cast vote}.
        
        \item Otherwise, the \textit{Voter} $\mathcal{V}_i$ has the option to challenge the vote cryptogram and verify that it actually contains the vote $m$, process described in \Cref{sec: challenging a vote cryptogram}. During this process, the vote cryptogram $e$is marked as "spoiled" on the bulletin board. After challenging the vote cryptogram, the \textit{Voter} has to regenerate a digital ballot (return to previous point). The process of challenging a vote cryptogram can be repeated as many times as needed until the \textit{Voter} gains enough trust in the voting application and can continue in the process to \Cref{itm: cast vote}.
    \end{itemize}
    
    \item \label{itm: cast vote} The \textit{Voter} $\mathcal{V}_i$ marks the vote cryptogram as "cast" on the bulletin board by posting a \textit{cast request} item to the \textit{Digital Ballot Box} $\mathcal{D}$ as described in \Cref{sec: vote confirmation receipt}. The process is illustrated in \Cref{fig: protocol for casting a submitted ballot}.
    
    \item The last response from the \textit{Digital Ballot Box} $\mathcal{D}$ that the \textit{Voter} $\mathcal{V}_i$ receives which certifies the acceptance of the \textit{cast request} item on the bulletin board is regarded as a vote receipt. The \textit{Voter} should store this receipt, as it can be used at any time to verify that the vote is registered by the \textit{Digital Ballot Box} as described in \Cref{sec: vote is registered as cast}.
\end{enumerate}
\color{black}








\clearpage
\subsubsection{Challenging a Vote Cryptogram} \label{sec: challenging a vote cryptogram}
After encrypting a ballot, a the \textit{Voter} $\mathcal{V}_i$ can choose whether to test or submit it. To perform the testing process of an encrypted ballot, the voter needs to interact with the \textit{External Verifier} $\mathcal{X}$ that will perform all the testing operations on behalf of the voter, according to the data delivered by the \textit{Digital Ballot Box} $\mathcal{D}$. At the end of the testing process, the voter will be presented with the vote choice(s) that were encoded in the encrypted ballot. When doing the testing procedure, the encrypted ballot that is being tested gets spoiled. Therefore, the voter needs to redo the vote cryptogram generation process from \cref{sec: vote cryptogram generation process} to get a new encrypted ballot, which the voter has to choose again whether to test or to submit. This process can be repeated until the voter trusts the legitimacy of the next encrypted ballot generated by the voting application. The protocol is inspired from \cite{Benaloh06}.

The \textit{Voter} $\mathcal{V}_i$ holds the ancestry $\alpha_\mathrm{vts} = \alpha_\mathrm{cfg} \cup \{ b_\mathrm{vos}, b_\mathrm{vec}, b_\mathrm{sec}, b_\mathrm{bac}, b_\mathrm{vts} \}$, where $\alpha_\mathrm{cfg}$ is the list of board items that define the election configuration, $b_\mathrm{vos}$ is the \textit{voter session} item, $b_\mathrm{vec}$ is the \textit{voter encryption commitment} item, $b_\mathrm{sec}$ is the \textit{server encryption commitment} item, $b_\mathrm{bac}$ is the \textit{ballot cryptograms} item and $b_\mathrm{vts}$ is the \textit{verification track start} item, each with their respective hash addresses $h_\mathrm{vos}$, $h_\mathrm{vec}$, $h_\mathrm{sec}$, $h_\mathrm{bac}$ and $h_\mathrm{vts}$. The \textit{Digital Ballot Box} $\mathcal{D}$ is in possession of the entire bulletin board $\lambda$ that includes the ancestry of the verification track start item $\alpha_\mathrm{vts}$. The ancestry $\alpha_\mathrm{vts}$ has been created during the vote cryptogram generation process (\cref{sec: vote cryptogram generation process}). The \textit{External Verifier} $\mathcal{X}$ starts with no internal knowledge.

The first part of the protocol (\cref{fig: protocol for challenging a vote cryptogram part 1a} and \cref{fig: protocol for challenging a vote cryptogram part 1b}) establish a trusted connection amongst the \textit{Voter} and the \textit{External Verifier} over the bulletin board. The \textit{Voter} inputs to the \textit{External Verifier} the address of the \textit{verification track start} item $h_\mathrm{vts}$, which then asks the \textit{Digital Ballot Box} for the ancestry of the item at that address. The \textit{Digital Ballot Box} delivers the ancestry $\alpha_\mathrm{vts}$, which the \textit{External Verifier} validates by $\mathsf{AncestryVer}(\alpha_\mathrm{vts})$. If valid, the \textit{External Verifier} notifies the \textit{Voter} that the ballot was successfully found.

Then, \textit{Voter} $\mathcal{V}_i$ interacts with the \textit{Digital Ballot Box} to append a \textit{spoil request} item $b_\mathrm{spr} = (\text{"spoil request"}, \perp, \mathcal{V}_i, \sigma_\mathrm{spr}, t_\mathrm{spr}, h_\mathrm{bac}, h_\mathrm{spr})$ (according to the rules from \cref{sec: bulletin board event types}), where $\sigma_\mathrm{spr}$ is the voter's signature on the concatenation of the text "spoil request" and the address of the \textit{ballot cryptograms} item $h_\mathrm{bac}$, and $h_\mathrm{spr}$ is the address of the new \textit{spoil request} item. The symbol $\perp$ denotes that the \textit{spoil request} item has no content. The voter is provided with the item $b_\mathrm{spr}$ and the receipt $\rho_\mathrm{spr}$ which is a signature of the \textit{Digital Ballot Box} on the concatenation of the voter's signature $\sigma_\mathrm{spr}$ and the address of the item $h_\mathrm{spr}$. \textit{Voter} verifies the integrity of the new board item and of the receipt.

In \cref{fig: protocol for challenging a vote cryptogram part 1b} is presented that the item $b_\mathrm{spr}$ and the receipt $\rho_\mathrm{spr}$ are delivered also to the \textit{External Verifier}, which checks that the \textit{spoil request} item is consistent with the ancestry of the \textit{ballot cryptograms} item $\mathsf{AncestryVer}(\alpha_\mathrm{bac} \cup \{ b_\mathrm{spr} \})$, where $\alpha_\mathrm{vts} = \alpha_\mathrm{bac} \cup \{ b_\mathrm{vts} \}$. If validation succeeds, \textit{External Verifier} generates its own key pair $(x_\mathcal{X}, Y_\mathcal{X}) \gets \mathsf{KeyGen}()$ and interacts with the \textit{Digital Ballot Box} to append the \textit{verifier} item $b_\mathrm{ver} = (\text{"verifier"}, Y_\mathcal{X}, \mathcal{X}, \sigma_\mathrm{ver}, t_\mathrm{ver}, h_\mathrm{spr}, h_\mathrm{ver})$ (according to the rules from \cref{sec: bulletin board event types}), where $\sigma_\mathrm{ver}$ is the external verifier's signature on the concatenation of the text "verifier", its public key $Y_\mathcal{X}$ and the address of the \textit{spoil request} item $h_\mathrm{spr}$. The value $h_\mathrm{ver}$ is the address of the new \textit{verifier} item. \textit{Digital Ballot Box} returns to the \textit{External Verifier} the item $b_\mathrm{ver}$ and the receipt $\rho_\mathrm{ver}$ which is a signature of the \textit{Digital Ballot Box} on the concatenation of the external verifier's signature $\sigma_\mathrm{ver}$ and the address of the item $h_\mathrm{ver}$. \textit{External Verifier} checks the integrity of the new board item and of the receipt. If they validate, \textit{External Verifier} presents to the \textit{Voter} the address of the \textit{verifier} item $h_\mathrm{ver}$.

\textit{Digital Ballot Box} also notifies the \textit{Voter} about the \textit{verifier} item $b_\mathrm{ver}$ and the receipt $\rho_\mathrm{ver}$. The voter checks that the \textit{verifier} item is consistent with the ancestry of the \textit{spoil request} item $\mathsf{AncestryVer}(\alpha_\mathrm{spr} \cup b_\mathrm{ver})$ and verifies the integrity of the receipt. If both validate, then the \textit{Voter} compares the address of $b_\mathrm{ver}$ (i.e. $h_\mathrm{ver}$) with the address received from the \textit{External Verifier}. If they match, the \textit{Voter} and the \textit{External Verifier} successfully established a trusted connection and can continue the protocol.

In the last part of the protocol (\cref{fig: protocol for challenging a vote cryptogram part 2a} and \cref{fig: protocol for challenging a vote cryptogram part 2b}) both the \textit{Voter} and \textit{Digital Ballot Box} deliver to the \textit{External Verifier}, in a secure manner, their randomizer values. The \textit{External Verifier} checks that all the values are correct and use them to decrypt the voter's ballot cryptograms and present the vote choices to the \textit{Voter} for assessment.

In \cref{fig: protocol for challenging a vote cryptogram part 2a}, the \textit{Voter} interacts with the \textit{Digital Ballot Box} to append a \textit{voter commitment opening} item $b_\mathrm{vco}$ = ("voter commitment opening", $d_\mathrm{vco}$, $\mathcal{V}_i$, $\sigma_\mathrm{vco}$, $t_\mathrm{vco}$, $h_\mathrm{ver}$, $h_\mathrm{vco}$) (according to the rules from \cref{sec: bulletin board event types}), where $\sigma_\mathrm{vco}$ is the voter's signature on the concatenation of the text "voter commitment opening", the content of the item $d_\mathrm{vco}$ and the address of the \textit{verifier} item $h_\mathrm{ver}$. The value $h_\mathrm{vco}$ is the address of the new \textit{voter commitment opening} item. The content of the item $d_\mathrm{vco}$ is the encryption of the values $r_\mathrm{v}$ and $s_\mathrm{v}$ with a symmetric key $k_\mathrm{v}$ derived from a Diffie-Hellmann key exchange between the \textit{Voter's} and the \textit{External Verifier's} public key infrastructure (i.e. $k_\mathrm{v} = \mathsf{DHKDF}(x_i, Y_\mathcal{X}) = \mathsf{DHKDF}(x_\mathcal{X}, Y_i)$, where $(x_i, Y_i)$ is the \textit{Voter's} key pair and $(x_\mathcal{X}, Y_\mathcal{X})$ is the \textit{External Verifier's} key pair). 

\textit{Digital Ballot Box} also appends a \textit{server commitment opening} item $b_\mathrm{sco}$ = ("server commitment opening", $d_\mathrm{sco}$, $\mathcal{D}$, $\sigma_\mathrm{sco}$, $t_\mathrm{sco}$, $h_\mathrm{vco}$, $h_\mathrm{sco}$), where $\sigma_\mathrm{sco}$ is the signature of the \textit{Digital Ballot Box} on the item and $h_\mathrm{sco}$ is the address of the new \textit{server commitment opening} item. The content of the item $d_\mathrm{vco}$ is the encryption of the values $r_\mathrm{d}$ and $s_\mathrm{d}$ with a symmetric key $k_\mathrm{d}$ derived from a Diffie-Hellmann key exchange between the \textit{Digital Ballot Box} and the \textit{External Verifier's} public key infrastructure (i.e. $k_\mathrm{d} = \mathsf{DHKDF}(x_\mathcal{D}, Y_\mathcal{X}) = \mathsf{DHKDF}(x_\mathcal{X}, Y_\mathcal{D})$, where $(x_\mathcal{D}, Y_\mathcal{D})$ is the key pair of \textit{Digital Ballot Box} and $(x_\mathcal{X}, Y_\mathcal{X})$ is the \textit{External Verifier's} key pair). 

Note that $r_\mathrm{v}$, $s_\mathrm{v}$ and $r_\mathrm{d}$, $s_\mathrm{d}$ respectively, are the commitment values generated during the vote cryptogram generation process described in \cref{sec: vote cryptogram generation process}.

\textit{Digital Ballot Box} returns to the \textit{Voter} both items $b_\mathrm{vco}$ and $b_\mathrm{sco}$ and the receipt $\rho_\mathrm{sco}$ which is a signature of \textit{Digital Ballot Box} on the concatenation of the voter's signature $\sigma_\mathrm{vco}$ and the address of the \textit{server commitment opening} item $h_\mathrm{sco}$. \textit{Voter} checks the integrity of the new board items and of the receipt.

In \cref{fig: protocol for challenging a vote cryptogram part 2b} is presented that the items $b_\mathrm{vco}$ and $b_\mathrm{sco}$ and the receipt $\rho_\mathrm{sco}$ are delivered also to the \textit{External Verifier}, which checks that the two new items are consistent with the ancestry of the \textit{verifier} item $\mathsf{AncestryVer}(\alpha_\mathrm{ver} \cup \{ b_\mathrm{vco}, b_\mathrm{sco} \})$. If validation succeeds, \textit{External Verifier} decrypts the commitment opening values $(r_\mathrm{v}, s_\mathrm{v}) \gets \mathsf{TxtDec}(d_\mathrm{vco}, k_\mathrm{v})$ from the content of the \textit{voter commitment opening} item $b_\mathrm{vco}$ and the values $(r_\mathrm{d}, s_\mathrm{d}) \gets \mathsf{TxtDec}(d_\mathrm{sco}, k_\mathrm{d})$ from the content of the \textit{server commitment opening} item $b_\mathrm{sco}$, respectively. Note that the symmetric keys are derived from the Diffie-Hellmann key exchange between the respective parties, i.e. $k_\mathrm{v} = \mathsf{DHKDF}(x_\mathcal{X}, Y_i)$ for the shared key between the \textit{Voter} and the \textit{External Verifier} and $k_\mathrm{d} = \mathsf{DHKDF}(x_\mathcal{X}, Y_\mathcal{D})$ for the shared key between the \textit{Digital Ballot Box} and the \textit{External Verifier}.

Then, the \textit{External Verifier} verifies both the voter and the server commitments ($c_\mathrm{v}$ and $c_\mathrm{d}$) from the \textit{voter encryption commitment} and the \textit{server encryption commitment} items ($b_\mathrm{vec}$ and $b_\mathrm{sec}$) received in the ancestry of the \textit{verification track start} item $\alpha_\mathrm{vts} = \alpha_\mathrm{cfg} \cup \{ b_\mathrm{vos}, b_\mathrm{vec}, b_\mathrm{sec}, b_\mathrm{bac}, b_\mathrm{vts} \}$. If both commitment are valid, i.e. $\mathsf{ComVer}(c_\mathrm{v}, r_\mathrm{v}, s_\mathrm{v})$ and $\mathsf{ComVer}(c_\mathrm{d}, r_\mathrm{d}, s_\mathrm{d})$, then the \textit{External Verifier} can decrypt the ballot and present it to the \textit{Voter}.

To perform the decryption, the \textit{External Verifier} tweaks the cryptogram by replacing the randomizer of the tuple with the encryption key $e' \gets (Y_\mathrm{enc}, C)$, where the original cryptogram $e = (R, C)$ is found in the content of the \textit{ballot cryptograms} item $b_\mathrm{bac}$. As decryption key, it uses the sum of the voter's and the server's randomizers $r' \gets r_\mathrm{v} + r_\mathrm{d}$. The resulting vote $V' \gets \mathsf{Dec}(r', e')$ is delivered to the \textit{Voter}.

The \textit{Voter} verifies that the received vote $V'$ matches with the intended selection $V$ from the vote cryptogram generation process (\cref{sec: vote cryptogram generation process}) and acts accordingly. If the vote matches, then the \textit{Voter} is assured that the voting application behaved correctly (i.e. it encrypted a genuine vote). Otherwise, the \textit{Voter} has evidence that the voting application has misbehaved during the process.

\clearpage
\begin{landscape}
\begin{figure}[ht]
    \centering
    \begin{tikzpicture}[framed, node distance=0,
            % every node/.style={draw},
            ]{
        
        % Actors
        \node[title_3, above, anchor=north east] (v) {
            \textbf{Voter $\mathcal{V}_i$}};
        \node[title_3, right=of v] (ev) {
            \textbf{External Verifier $\mathcal{X}$}};
        \node[title_3, right=of ev] (dbb) {
            \textbf{Digital Ballot Box $\mathcal{D}$}};
        
        % internal knowledge
        \node[internal_3, below=of v] (v_ik) {
            internal knowledge: $x_i$, $Y_\mathcal{D}$, $h_\mathrm{bac}$, $h_\mathrm{vts}$, \\
            $\alpha_\mathrm{vts} = \alpha_\mathrm{cfg} \cup \{ b_\mathrm{vos}, b_\mathrm{vec}, b_\mathrm{sec}, b_\mathrm{bac}, b_\mathrm{vts} \}$
            };
        \node[internal_3, below=of ev] (ev_ik) {
            no internal knowledge
            };
        \node[internal_3, below=of dbb.south east, anchor=north east, text width={}] (dbb_ik) {
            internal knowledge: $x_\mathcal{D}$, $\boldsymbol{Y} = \{ Y_1, ..., Y_{n_\mathrm{v}} \}$, \\
            $\alpha_\mathrm{vts} = \alpha_\mathrm{cfg} \cup \{ b_\mathrm{vos}, b_\mathrm{vec}, b_\mathrm{sec}, b_\mathrm{bac}, b_\mathrm{vts} \}$, $\lambda = \{ b_1, ..., b_{n_\mathrm{b}} \}$
            };
        
        % All content
        \node[new_arrow, towards_right, between={v.center}{ev.center}] at (v_ik.south -| v.center) (a_1) {
            $h_\mathrm{vts}$
            };
        \node[new_arrow, towards_right, between={ev.center}{dbb.center}] at (a_1.south -| ev.center) (a_2) {
            $h_\mathrm{vts}$
            };
        \node[new_arrow, towards_left, between={ev.center}{dbb.center}] at (a_2.south -| dbb.center) (a_3) {
            $\alpha_\mathrm{vts} = \alpha_\mathrm{bac} \cup \{ b_\mathrm{vts} \}$
            };
        \node[process_3, keep_middle, after_arrow] at (a_3.south -| ev.center) (ev_1) {
            verify that $\mathsf{AncestryVer}(\alpha_\mathrm{vts})$
            };
        \node[new_arrow, towards_left, between={v.center}{ev.center}] at (ev_1.south -| ev.center) (a_4) {
            "ballot found"
            };
        \node[process_3, keep_middle, after_arrow] at (a_4.south -| v.center) (v_1) {
            $m_\mathrm{spr} \gets $ "spoil request" \\
            $\sigma_\mathrm{spr} \gets \mathsf{Sign}(x_i; m_\mathrm{spr} || \perp || h_\mathrm{bac})$
            };
        \node[new_arrow, towards_right, between={v.center}{dbb.center}] at (v_1.south -| v.center) (a_5) {
            $i$, $m_\mathrm{spr}$, $\sigma_\mathrm{spr}$, $h_\mathrm{bac}$
            };
        \node[process_3, keep_right, after_arrow] at (a_5.south -| dbb.east) (dbb_1) {
            verify that $\mathsf{SigVer} (Y_i, \sigma_\mathrm{spr}; m_\mathrm{spr} || \perp || h_\mathrm{bac})$ then: \\ [4pt]
            $t_\mathrm{spr} \gets$ current time, \\
            $h_\mathrm{spr} \gets \mathcal{H}(m_\mathrm{spr} || \perp || h_\mathrm{bac} || h_{n_\mathrm{b}} || t_\mathrm{spr})$ \\
            $b_\mathrm{spr} \gets (m_\mathrm{spr}, \perp, \mathcal{V}_i, \sigma_\mathrm{spr}, t_\mathrm{spr}, h_\mathrm{bac}, h_\mathrm{spr})$, \\
            $\lambda \gets \lambda \cup \{ b_\mathrm{spr} \}$ \\ 
            $\rho_\mathrm{spr} \gets \mathsf{Sign}(x_\mathcal{D}; \sigma_\mathrm{spr} || h_\mathrm{spr})$
            };
        \node[new_arrow, towards_left, between={v.center}{dbb.center}] at (dbb_1.south -| dbb.center) (a_6) {
            $b_\mathrm{spr} = (m_\mathrm{spr}, \perp, \mathcal{V}_i, \sigma_\mathrm{spr}, t_\mathrm{spr}, h_\mathrm{bac}, h_\mathrm{spr})$, $\rho_\mathrm{spr}$
            };
        \node[process_3, keep_left, after_arrow] at (a_6.south -| v.west) (v_2) {
            verify that $h_\mathrm{spr} = \mathcal{H}(m || \perp || h_\mathrm{bac} || h_{n_\mathrm{b}} || t_\mathrm{spr})$ \\
            and $\mathsf{SigVer} (Y_\mathcal{D}, \rho_\mathrm{spr}; \sigma_\mathrm{spr} || h_\mathrm{spr})$
            };
        \node[below=of v_2] (bottom){};
        
        % Arrows and lines
        \draw[dashed] (v.south west)--(dbb.south east);    
        \draw[dashed] (v_ik.south west)--(dbb_ik.south east);
        
        \draw[dotted] (v_ik.south -| v.center)--(v_1.north -| v.center);
        \draw[dotted] (v_1.south -| v.center)--(v_2.north -| v.center);
        \draw[dotted] (v_2.south -| v.center)--(bottom.south -| v.center);
        
        \draw[dotted] (a_1.south -| ev.center)--(ev_1.north -| ev.center);
        \draw[dotted] (ev_1.south -| ev.center)--(bottom.south -| ev.center);
        
        \draw[dotted] (dbb_ik.south -| dbb.center)--(dbb_1.north -| dbb.center);
        \draw[dotted] (dbb_1.south -| dbb.center)--(bottom.south -| dbb.center);
        }
    \end{tikzpicture}
    \caption{Protocol for challenging a vote cryptogram (part 1a)}
    \label{fig: protocol for challenging a vote cryptogram part 1a}
\end{figure}

\clearpage
\begin{figure}[ht]
    \centering
    \begin{tikzpicture}[framed, node distance=0,
            % every node/.style={draw},
            ]{
        
        % Actors
        \node[title_3, above, anchor=north east] (v) {
            \textbf{Voter $\mathcal{V}_i$}};
        \node[title_3, right=of v] (ev) {
            \textbf{External Verifier $\mathcal{X}$}};
        \node[title_3, right=of ev] (dbb) {
            \textbf{Digital Ballot Box $\mathcal{D}$}};
        
        % internal knowledge
        \node[internal_3, below=of v.south west, anchor=north west, text width={}] (v_ik) {
            internal knowledge: $x_i$, $Y_\mathcal{D}$, $h_\mathrm{bac}$, $h_\mathrm{vts}$, \\
            $\alpha_\mathrm{spr} = \alpha_\mathrm{cfg} \cup \{ b_\mathrm{vos}, b_\mathrm{vec}, b_\mathrm{sec}, b_\mathrm{bac}, b_\mathrm{spr} \}$, \\ $b_\mathrm{vts}$
            };
        \node[internal_3, below=of ev] (ev_ik) {
            internal knowledge: $Y_\mathcal{D}$, \\
            $\alpha_\mathrm{vts} = \alpha_\mathrm{bac} \cup \{ b_\mathrm{vts} \}$
            };
        \node[internal_3, below=of dbb.south east, anchor=north east, text width={}] (dbb_ik) {
            internal knowledge: $x_\mathcal{D}$, $\boldsymbol{Y} = \{ Y_1, ..., Y_{n_\mathrm{v}} \}$, \\
            $\alpha_\mathrm{spr} = \alpha_\mathrm{cfg} \cup \{ b_\mathrm{vos}, b_\mathrm{vec}, b_\mathrm{sec}, b_\mathrm{bac}, b_\mathrm{spr} \}$, \\ $b_\mathrm{vts}$, $\lambda = \{ b_1, ..., b_{n_\mathrm{b}} \}$
            };
        
        % All content
        \node[below=of dbb_ik] (top){};
        \node[new_arrow, towards_left, between={ev.center}{dbb.center}] at (top.south -| dbb.center) (a_7) {
            \phantom{$b_\mathrm{spr} = (\mathcal{V}_i)$}
            };
        \node[align=center, font=\small, anchor=north west] at (top.south -| ev.center) (a_7_content) {
            $b_\mathrm{spr} = (m_\mathrm{spr}, \perp, \mathcal{V}_i, \sigma_\mathrm{spr}, t_\mathrm{spr}, h_\mathrm{bac}, h_\mathrm{spr})$, $\rho_\mathrm{spr}$
            };
        \node[process_3, keep_middle, after_arrow] at (a_7.south -| ev.center) (ev_2) {
            verify that $\mathsf{AncestryVer}(\alpha_\mathrm{bac} \cup \{ b_\mathrm{spr} \})$ \\
            and $\mathsf{SigVer} (Y_\mathcal{D}, \rho_\mathrm{spr}; \sigma_\mathrm{spr} || h_\mathrm{spr})$ then: \\ [4pt]
            $(x_\mathcal{X}, Y_\mathcal{X}) \gets \mathsf{KeyGen}()$, $m_\mathrm{v} \gets "verifier"$ \\
            $\sigma_\mathrm{ver} \gets \mathsf{Sign}(x_\mathcal{X}; m_\mathrm{v} || Y_\mathcal{X} || h_\mathrm{spr})$
            };
        \node[new_arrow, towards_right, between={ev.center}{dbb.center}] at (ev_2.south -| ev.center) (a_8) {
            $h_\mathrm{vts}$, $m_\mathrm{v}$, $Y_\mathcal{X}$, $\sigma_\mathrm{ver}$
            };
        \node[process_3, keep_right, after_arrow] at (a_8.south -| dbb.east) (dbb_2) {
            verify that $\mathsf{SigVer} (Y_\mathcal{X}, \sigma_\mathrm{ver}; m_\mathrm{v} || Y_\mathcal{X} || h_\mathrm{spr})$ then: \\ [4pt]
            $t_\mathrm{ver} \gets$ current time, $h_\mathrm{ver} \gets \mathcal{H}(m_\mathrm{v} || Y_\mathcal{X} || h_\mathrm{spr} || h_\mathrm{vts} || t_\mathrm{ver})$ \\
            $b_\mathrm{ver} \gets (m_\mathrm{v}, Y_\mathcal{X}, \mathcal{X}, \sigma_\mathrm{ver}, t_\mathrm{ver}, h_\mathrm{spr}, h_\mathrm{ver})$ \\
            $\lambda \gets \lambda \cup \{ b_\mathrm{ver} \}$, $\rho_\mathrm{ver} \gets \mathsf{Sign}(x_\mathcal{D}; \sigma_\mathrm{ver} || h_\mathrm{ver})$
            };
        \node[new_arrow, towards_left, between={v.center}{dbb.center}] at (dbb_2.south -| dbb.center) (a_9) {
            $b_\mathrm{ver} = (m_\mathrm{v}, Y_\mathcal{X}, \mathcal{X}, \sigma_\mathrm{ver}, t_\mathrm{ver}, h_\mathrm{spr}, h_\mathrm{ver})$, $\rho_\mathrm{ver}$
            };
        \node[process_3, keep_left, after_arrow] at (a_9.south -| v.west) (v_3) {
            verify that $\mathsf{AncestryVer}(\alpha_\mathrm{spr} \cup \{ b_\mathrm{ver} \})$ \\
            and $\mathsf{SigVer} (Y_\mathcal{D}, \rho_\mathrm{ver}; \sigma_\mathrm{ver} || h_\mathrm{ver})$
            };
        \node[new_arrow, towards_left, between={ev.center}{dbb.center}] at (a_9.south -| dbb.center) (a_10) {
            \phantom{$b_\mathrm{ver} = (Y_\mathcal{X})$}
            };
        \node[align=center, font=\small, anchor=north west] at (a_9.south -| ev.center) (a_10_content) {
            $b_\mathrm{ver} = (m_\mathrm{v}, Y_\mathcal{X}, \mathcal{X}, \sigma_\mathrm{ver}, t_\mathrm{ver}, h_\mathrm{spr}, h_\mathrm{ver})$, $\rho_\mathrm{ver}$
            };
        \node[process_3, keep_middle, after_arrow] at (a_10.south -| ev.center) (ev_3) {
            verify that $h_\mathrm{ver} = \mathcal{H}(m_\mathrm{v} || Y_\mathcal{X} || h_\mathrm{spr} || h_\mathrm{vts} || t_\mathrm{ver})$ \\
            and $\mathsf{SigVer} (Y_\mathcal{D}, \rho_\mathrm{ver}; \sigma_\mathrm{ver} || h_\mathrm{ver})$
            };
        \node[new_arrow, towards_left, between={v.center}{ev.center}] at (ev_3.south -| ev.center) (a_11) {
            $h'_\mathrm{ver} = h_\mathrm{ver}$
            };
        \node[process_3, keep_middle, after_arrow] at (a_11.south -| v.center) (v_4) {
            verify that $h_\mathrm{ver} = h'_\mathrm{ver}$
            };
        \node[below=of v_4] (bottom){};
        
        % Arrows and lines
        \draw[dashed] (v.south west)--(dbb.south east);    
        \draw[dashed] (top.north -| v.west)--(top.north -| dbb.east);
        
        \draw[dotted] (top.north -| v.center)--(v_3.north -| v.center);
        \draw[dotted] (v_3.south -| v.center)--(v_4.north -| v.center);
        \draw[dotted] (v_4.south -| v.center)--(bottom.south -| v.center);
        
        \draw[dotted] (top.north -| ev.center)--(ev_2.north -| ev.center);
        \draw[dotted] (ev_2.south -| ev.center)--(ev_3.north -| ev.center);
        \draw[dotted] (ev_3.south -| ev.center)--(bottom.south -| ev.center);
        
        \draw[dotted] (top.north -| dbb.center)--(dbb_2.north -| dbb.center);
        \draw[dotted] (dbb_2.south -| dbb.center)--(bottom.south -| dbb.center);
        }
    \end{tikzpicture}
    \caption{Protocol for challenging a vote cryptogram (part 1b)}
    \label{fig: protocol for challenging a vote cryptogram part 1b}
\end{figure}

\clearpage
\begin{figure}[ht]
    \centering
    \begin{tikzpicture}[framed, node distance=0,
            % every node/.style={draw},
            ]{
        
        % Actors
        \node[title_3, above, anchor=north east] (v) {
            \textbf{Voter $\mathcal{V}_i$}};
        \node[title_3, right=of v] (ev) {
            \textbf{External Verifier $\mathcal{X}$}};
        \node[title_3, right=of ev] (dbb) {
            \textbf{Digital Ballot Box $\mathcal{D}$}};
        
        % internal knowledge
        \node[internal_3, below=of v] (v_ik) {
            internal knowledge: $x_i$, $Y_\mathcal{D}$, $Y_\mathcal{X}$, $h_\mathrm{bac}$, \\
            $h_\mathrm{vts}$, $h_\mathrm{ver}$, $\alpha_\mathrm{ver} = \alpha_\mathrm{spr} \cup \{ b_\mathrm{ver} \}$, $b_\mathrm{ver}$, \\
            $r_\mathrm{v}$, $s_\mathrm{v}$
            };
        \node[internal_3, below=of ev] (ev_ik) {
            internal knowledge: $x_\mathcal{X}$, $Y_\mathcal{D}$ \\
            $\alpha_\mathrm{ver} = \alpha_\mathrm{bac} \cup \{ b_\mathrm{spr}, b_\mathrm{ver} \}$, $b_\mathrm{vts}$
            };
        \node[internal_3, below=of dbb.south east, anchor=north east, text width={}] (dbb_ik) {
            internal knowledge: $x_\mathcal{D}$, $\boldsymbol{Y} = \{ Y_1, ..., Y_{n_\mathrm{v}} \}$, $Y_\mathcal{X}$, \\
            $\alpha_\mathrm{ver} = \alpha_\mathrm{spr} \cup \{ b_\mathrm{ver} \}$, $b_\mathrm{vts}$, $\lambda = \{ b_1, ..., b_{n_\mathrm{b}} \}$
            };
        
        % All content
        \node[below=of v_ik] (top){};
        \node[process_3, keep_left] at (top.south -| v.west) (v_1) {
            $k_\mathrm{v} \gets \mathsf{DHKDF}(x_i, Y_\mathcal{X})$ \\
            $d_\mathrm{vco} \gets \mathsf{TxtEnc}(r_\mathrm{v} || s_\mathrm{v}, k_\mathrm{v})$ \\
            $m_\mathrm{vco} \gets$ "voter encryption commitment opening" \\
            $\sigma_\mathrm{vco} \gets \mathsf{Sign}(x_i; m_\mathrm{vco} || d_\mathrm{vco} || h_\mathrm{ver})$
            };
        \node[new_arrow, towards_right, between={v.center}{dbb.center}] at (v_1.south -| v.center) (a_1) {
            $i$, $m_\mathrm{vco}$, $d_\mathrm{vco}$, $\sigma_\mathrm{vco}$, $h_\mathrm{ver}$
            };
        \node[process_3, keep_right, after_arrow] at (a_1.south -| dbb.east) (dbb_1) {
            verify that $\mathsf{SigVer} (Y_i, \sigma_\mathrm{vco}; m_\mathrm{vco} || d_\mathrm{vco} || h_\mathrm{ver})$ then: \\ [4pt]
            $t_\mathrm{vco} \gets$ current time, $h_\mathrm{vco} \gets \mathcal{H}(m_\mathrm{vco} || d_\mathrm{vco} || h_\mathrm{ver} || h_\mathrm{ver} || t_\mathrm{vco})$ \\
            $b_\mathrm{vco} \gets (m_\mathrm{vco}, d_\mathrm{vco}, \mathcal{V}_i, \sigma_\mathrm{vco}, t_\mathrm{vco}, h_\mathrm{ver}, h_\mathrm{vco})$ \\
            $k_\mathrm{d} \gets \mathsf{DHKDF}(x_\mathcal{D}, Y_\mathcal{X})$, $d_\mathrm{sco} \gets \mathsf{TxtEnc}(r_\mathrm{d} || s_\mathrm{d}, k_\mathrm{d})$ \\
            $m_\mathrm{sco} \gets$ "server encryption commitment opening" \\
            $t_\mathrm{sco} \gets$ current time, $h_\mathrm{sco} \gets \mathcal{H}(m_\mathrm{sco} || d_\mathrm{vco} || h_\mathrm{vco} || h_\mathrm{vco} || t_\mathrm{sco})$ \\
            $b_\mathrm{sco} \gets (m_\mathrm{sco}, d_\mathrm{sco}, \mathcal{D}, \sigma_\mathrm{sco}, t_\mathrm{sco}, h_\mathrm{vco}, h_\mathrm{sco})$ \\
            $\lambda \gets \lambda \cup \{ b_\mathrm{vco}, b_\mathrm{sco} \}$, $\rho_\mathrm{sco} \gets \mathsf{Sign}(x_\mathcal{D}; \sigma_\mathrm{vco} || h_\mathrm{sco})$
            };
        \node[new_arrow, towards_left, between={v.center}{dbb.center}] at (dbb_1.south -| dbb.center) (a_2) {
            $b_\mathrm{vco} = (m_\mathrm{vco}, d_\mathrm{vco}, \mathcal{V}_i, \sigma_\mathrm{vco}, t_\mathrm{vco}, h_\mathrm{ver}, h_\mathrm{vco})$, \\
            $b_\mathrm{sco} = (m_\mathrm{sco}, d_\mathrm{sco}, \mathcal{D}, \sigma_\mathrm{sco}, t_\mathrm{sco}, h_\mathrm{vco}, h_\mathrm{sco})$, $\rho_\mathrm{sco}$
            };
        \node[process_3, keep_left, after_arrow] at (a_2.south -| v.west) (v_2) {
            verify that $\mathsf{SigVer} (Y_\mathcal{D}, \rho_\mathrm{sco}; \sigma_\mathrm{vco} || h_\mathrm{sco})$, \\
            $h_\mathrm{vco} = \mathcal{H}(m_\mathrm{vco} || d_\mathrm{vco} || h_\mathrm{ver} || h_\mathrm{ver} || t_\mathrm{vco})$ and \\
            $h_\mathrm{sco} = \mathcal{H}(m_\mathrm{sco} || d_\mathrm{sco} || h_\mathrm{vco} || h_\mathrm{vco} || t_\mathrm{sco})$
            };
        \node[below=of v_2] (bottom){};
        
        % Arrows and lines
        \draw[dashed] (v.south west)--(dbb.south east);    
        \draw[dashed] (top.north -| v.west)--(top.north -| dbb.east);
        
        \draw[dotted] (top.north -| v.center)--(v_1.north -| v.center);
        \draw[dotted] (v_1.south -| v.center)--(v_2.north -| v.center);
        \draw[dotted] (v_2.south -| v.center)--(bottom.south -| v.center);
        
        \draw[dotted] (top.north -| ev.center)--(bottom.south -| ev.center);
        
        \draw[dotted] (top.north -| dbb.center)--(dbb_1.north -| dbb.center);
        \draw[dotted] (dbb_1.south -| dbb.center)--(bottom.south -| dbb.center);
        }
    \end{tikzpicture}
    \caption{Protocol for challenging a vote cryptogram (part 2a)}
    \label{fig: protocol for challenging a vote cryptogram part 2a}
\end{figure}

\clearpage
\begin{figure}[ht]
    \centering
    \begin{tikzpicture}[framed, node distance=0,
            % every node/.style={draw},
            ]{
        
        % Actors
        \node[title_3, above, anchor=north east] (v) {
            \textbf{Voter $\mathcal{V}_i$}};
        \node[title_3, right=of v] (ev) {
            \textbf{External Verifier $\mathcal{X}$}};
        \node[title_3, right=of ev] (dbb) {
            \textbf{Digital Ballot Box $\mathcal{D}$}};
        
        % internal knowledge
        \node[internal_3, below=of v] (v_ik) {
            internal knowledge: $x_i$, $Y_\mathcal{D}$, $Y_\mathcal{X}$, $h_\mathrm{bac}$, \\
            $h_\mathrm{vts}$, $h_\mathrm{ver}$, $\alpha_\mathrm{sco} = \alpha_\mathrm{ver} \cup \{ b_\mathrm{vco}, b_\mathrm{sco} \}$, $b_\mathrm{ver}$, \\
            $r_\mathrm{v}$, $s_\mathrm{v}$, $V$
            };
        \node[internal_3, below=of ev] (ev_ik) {
            internal knowledge: $x_\mathcal{X}$, $Y_i$, $Y_\mathcal{D}$,  \\
            $\alpha_\mathrm{ver}$, $b_\mathrm{vts}$, $c_\mathrm{v}$, $c_\mathrm{d}$, $Y_\mathrm{enc}$, $e = (R, C)$
            };
        \node[internal_3, below=of dbb.south east, anchor=north east, text width={}] (dbb_ik) {
            internal knowledge: $x_\mathcal{D}$, $\boldsymbol{Y} = \{ Y_1, ..., Y_{n_\mathrm{v}} \}$, $Y_\mathcal{X}$, \\
            $\alpha_\mathrm{sco} = \alpha_\mathrm{ver} \cup \{ b_\mathrm{vco}, b_\mathrm{sco} \}$, $\rho_\mathrm{sco}$, $b_\mathrm{vts}$, \\
            $\lambda = \{ b_1, ..., b_{n_\mathrm{b}} \}$, $r_\mathrm{d}$, $s_\mathrm{d}$
            };
        
        % All content
        \node[below=of v_ik] (top){};
        \node[new_arrow, towards_left, between={ev.center}{dbb.center}] at (top.south -| dbb.center) (a_1) {
            \phantom{$b_\mathrm{vco} = (\mathcal{V}_i)$} \\
            \phantom{$b_\mathrm{sco} = (\mathcal{D})$, $\rho_\mathrm{sco}$}
            };
        \node[align=center, font=\small, anchor=north west] at (top.south -| ev.center) (a_1_content) {
            $b_\mathrm{vco} = (m_\mathrm{vco}, d_\mathrm{vco}, \mathcal{V}_i, \sigma_\mathrm{vco}, t_\mathrm{vco}, h_\mathrm{ver}, h_\mathrm{vco})$, \\
            $b_\mathrm{sco} = (m_\mathrm{sco}, d_\mathrm{sco}, \mathcal{D}, \sigma_\mathrm{sco}, t_\mathrm{sco}, h_\mathrm{vco}, h_\mathrm{sco})$, $\rho_\mathrm{sco}$
            };
        \node[process_3, keep_middle, after_arrow] at (a_1.south -| ev.center) (ev_1) {
            verify that $\mathsf{AncestryVer}(\alpha_\mathrm{ver} \cup \{ b_\mathrm{vco}, b_\mathrm{sco} \})$ \\
            and $\mathsf{SigVer} (Y_\mathcal{D}, \rho_\mathrm{sco}; \sigma_\mathrm{vco} || h_\mathrm{sco})$ then: \\ [4pt]
            $k_\mathrm{v} \gets \mathsf{DHKDF}(x_\mathcal{X}, Y_i)$, $(r_\mathrm{v}, s_\mathrm{v}) \gets \mathsf{TxtDec}(d_\mathrm{vco}, k_\mathrm{v})$ \\
            $k_\mathrm{d} \gets \mathsf{DHKDF}(x_\mathcal{X}, Y_\mathcal{D})$, $(r_\mathrm{d}, s_\mathrm{d}) \gets \mathsf{TxtDec}(d_\mathrm{sco}, k_\mathrm{d})$ \\ [4pt]
            verify that $\mathsf{ComVer}(c_\mathrm{v}, r_\mathrm{v}, s_\mathrm{v})$
            and $\mathsf{ComVer}(c_\mathrm{d}, r_\mathrm{d}, s_\mathrm{d})$ then: \\ [4pt]
            $r' \gets r_\mathrm{v} + r_\mathrm{d}$, $e' \gets (Y_\mathrm{enc}, C)$, $V' \gets \mathsf{Dec}(r', e')$
            };
        \node[new_arrow, towards_left, between={v.center}{ev.center}] at (ev_1.south -| ev.center) (a_2) {
            $V'$
            };
        \node[process_3, keep_middle, after_arrow] at (a_2.south -| v.center) (v_1) {
            verify that $V = V'$
            };
        \node[below=of v_1] (bottom){};
        
        % Arrows and lines
        \draw[dashed] (v.south west)--(dbb.south east);    
        \draw[dashed] (v_ik.south west)--(dbb_ik.south east);
        
        \draw[dotted] (top.north -| v.center)--(v_1.north -| v.center);
        \draw[dotted] (v_1.south -| v.center)--(bottom.south -| v.center);
        
        \draw[dotted] (top.north -| ev.center)--(ev_1.north -| ev.center);
        \draw[dotted] (ev_1.south -| ev.center)--(a_2.south -| ev.center);
        
        \draw[dotted] (top.north -| dbb.center)--(bottom.south -| dbb.center);
        }
    \end{tikzpicture}
    \caption{Protocol for challenging a vote cryptogram (part 2b)}
    \label{fig: protocol for challenging a vote cryptogram part 2b}
\end{figure}
\clearpage
\end{landscape}


\clearpage
\subsubsection{Vote Confirmation Receipt} \label{sec: vote confirmation receipt}
Once the vote cryptogram $e_i$ has been registered, the \textit{Voter} $\mathcal{V}_i$ receives back from the \textit{Digital Ballot Box} $\mathcal{D}$ a confirmation receipt in form of a Schnorr signature (\Cref{alg: sign}) \( \rho_i = \mathbf{Sign}_{x_\mathrm{sign}} (\sigma_i || h_{\mathrm{b},i}) \), together with the new board hash value $h_{\mathrm{b},i}$, the previous board hash value $h_{\mathrm{b},i-1}$ and the registration time stamp $t_{\mathrm{r},i}$. Recall from \Cref{sec: pre-election phase} that $x_\mathrm{sign}$ is the \textit{Digital Ballot Box} $\mathcal{D}$'s signing key.

The receipt certifies that the vote cryptogram has been registered on the bulletin board at exactly version $h_{\mathrm{b},i}$. The \textit{Voter} $\mathcal{V}_i$ can use this receipt to verify, at any time, that her vote is included on the board, and that the history of the board has not changed by validating the hash value. 

If the previous board hash value $h_{\mathrm{b},i-1}$ does not match with the acknowledged hash $h_{\mathrm{a},i}$, that means a "race" situation (described in the previous section) has happened and the current \textit{Voter} $\mathcal{V}_i$ has lost the race, i.e. while the current \textit{Voter} $\mathcal{V}_i$ was preparing her vote submission, another \textit{Voter} $\mathcal{V}_i$ has successfully managed to post another vote submission. This is a perfectly valid scenario that can occur in busy situations.

Note that, if a \textit{Voter} $\mathcal{V}_i$ has a confirmation receipt that does not correspond with the current state of the bulletin board, that immediately reveals an attempt to break the integrity of the bulletin board and should be reported to the election authorities.

\begin{figure}[ht]
    \centering
    \begin{tikzpicture}[framed, node distance=0,
            % every node/.style={draw}
            ]{
            
        % Actors
        \node[title, above, anchor=north east] (v) {
            \textbf{Voter} $\mathcal{V}_i$};
        \node[title, right=of v] (dbb) {
            \textbf{Digital Ballot Box} $\mathcal{D}$};
        
        % internal knowledge
        \node[block, below=of v] (v1) {
            /* internal knowledge: \\
            \phantom{/}* $x_i$, $Y_\mathcal{D}$, $p$ */
            };
        \node[block, below=of dbb] (dbb1) {
            /* internal knowledge: \\
            \phantom{/}* $x_\mathcal{D}$, $\boldsymbol{Y} = \{Y_1, ..., Y_{n_\mathrm{v}}\}$, \\
            \phantom{/}* $\boldsymbol{h} = \{h_1, ... h_{n_\mathrm{b}}\}$ */
            };
        \node[block, anchor=south west] at (dbb1.south -| v1.west) (v1-corner) {};
        
        %All content
        \node[block, below=of v1-corner.south west, anchor=north west] (v2) {
            $m \gets$ "cast" \\
            $\sigma_i \gets \mathsf{Sign}(x_i; m || p)$
            };
        \node[arrow, below=of v2.south, anchor=north west] (a1) {
            $i$, $\sigma_i$, $m$, $p$
            };
        \node[block, anchor=north west] at (a1.south -| dbb1.west) (dbb2) {
            verify that $m$ and $p$ are compatible \\
            and $\mathsf{SigVer} (Y_i, \sigma_i; m || p)$ then: \\ [9pt]
            $t_{n_\mathrm{b}+1} \gets$ current timestamp \\
            $h_{n_\mathrm{b}+1} \gets \mathcal{H}(m || p || h_{n_\mathrm{b}} || t_{n_\mathrm{b}+1})$ \\ [9pt]
            /* store on the Bulletin Board as a \\
            \phantom{/}* new item the tuple: \\
            \phantom{/}* $(m, \perp, \mathcal{V}_i, \sigma_i, t_{n_\mathrm{b}+1}, p, h_{n_\mathrm{b}+1})$ */ \\ [9pt]
            $\rho \gets \mathsf{Sign}(x_\mathcal{D}; \sigma_i || h_{n_\mathrm{b}+1})$
            };
        \node[arrow, anchor=north west] at (a1.west |- dbb2.south) (a2) {
            $\rho$, $t_{n_\mathrm{b}+1}$, $h_{n_\mathrm{b}+1}$, $h_{n_\mathrm{b}}$
            };
        \node[block, anchor=north west] at (a2.south -| v2.west) (v3) {
            verify that \\
            $h_{n_\mathrm{b}+1} = \mathcal{H}(m || p || h_{n_\mathrm{b}} || t_{n_\mathrm{b}+1})$ \\
            and $\mathsf{SigVer} (Y_\mathcal{D}, \rho; \sigma_i || h_{n_\mathrm{b}+1})$
            };
        
        % Arrows and lines
        \draw[dashed] (v.south west)--(dbb.south east);    
        \draw[dashed] (v1-corner.south west)--(dbb1.south east);
        \draw[->] (a1.south west)--(a1.south east);
        \draw[->] (a2.south east)--(a2.south west);
        }
    \end{tikzpicture}
    \caption{Protocol for casting a submitted ballot}
    \label{fig: protocol for casting a submitted ballot}
\end{figure}

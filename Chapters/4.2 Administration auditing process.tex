\subsection{Administration auditing process} \label{sec: administration auditing process}
This section describes the auditing steps that are available only to election officials because they are based on data that is not publicly available. These auditing processes verify the activity of specific components of the election system. The administration auditing processes give confidence to the election officials that the election is run correctly. Therefore, the result is trustworthy.


\subsubsection{Eligibility verifiability}
This auditing process verifies that only eligible voters have submitted ballots to the bulletin board, i.e., verifying that all voter session items have been authorized by the voter authorizer based on successful voter authentication. This process can be done continuously throughout the election phase or as a final auditing step at the end of the election phase but before a result is computed.

Formally, the auditing starts by providing all eligible voter identities $\boldsymbol{\mathcal{V}} = \{ \mathcal{V}_1, ..., \mathcal{V}_{n_\mathrm{v}} \}$, the public key of the voter authorizer $Y_\mathcal{A}$ and the list of voter session items from the bulletin board $\{ b_{\mathrm{vs}; 1}, ..., b_{\mathrm{vs}; n_\mathrm{vs}} \}$, where $n_\mathrm{vs}$ is the total number of voter session items. Recall from \cref{sec: public bulletin board} that each item has the following structure $b_{vs; i} = (m_i, c_i, \mathcal{A}, \sigma_i, t_i, p_i, h'_i, h_i)$, with $i \in \{ 1, ..., n_\mathrm{vs} \}$.

The auditor checks that the voter authorizer has signed each item by running $\mathsf{SigVer} (Y_\mathcal{A}; \sigma_i, m_i || c_i || p_i)$ (\cref{alg: sig ver}) and its content relates to an eligible voter, i.e., $c_i = (\mathcal{V}_i, Y_i, H_i)$, with $\mathcal{V}_i \in \boldsymbol{\mathcal{V}}$. Then, the auditor verifies that all voter session items have been authorized after successful authentication, which is achieved differently, depending on the voter authentication mode used (see \cref{sec: voter authentication modes}).

When \textbf{credential-based} voter authentication mode is enabled, the auditor has also been provided with all voter authentication public keys $\{ Y_{\mathrm{auth}; 1}, ..., Y_{\mathrm{auth}; n_\mathrm{v}} \}$ for each of the voters in $\boldsymbol{\mathcal{V}}$. The voter authorizer provides all proofs of credentials $\{ PK_1, ..., PK_{n_\mathrm{vs}} \}$ associated with each voter session item from the bulletin board. The auditor checks that $H_i = \mathcal{H} (PK_i)$ and $\mathsf{DLVer} (PK_i, \{ G \}, \{ Y_{\mathrm{auth}; i} \})$ (\cref{alg: dl ver}), where $H_i$ is the authentication fingerprint from the content of the voter session item $b_{\mathrm{vs}; i}$ and $Y_{\mathrm{auth}; i}$ is the authentication public key of voter $\mathcal{V}_i$. In case one of the validations fails, that discovers an attempt of the voter authorizer to create a fraudulent voter session.

Recall from \cref{sec: voter credential distribution process} that the proof of credentials $PK_i$ is initiated by credentials generated based on a minimum of 80 bits of entropy. Being so low on entropy, the voter authentication public keys and proofs of credentials are not publicly disclosed to prevent a brute-force attack. Therefore, they are auditable only by the election officers.

When \textbf{identity-based} voter authentication mode is used, the auditor is provided instead, with the certificates of all identity providers $\boldsymbol{\mathcal{I}} = \{ \mathcal{I}_1, ..., \mathcal{I}_{n_\mathrm{i}} \}$ including their public keys $\{ Y_{\mathcal{I}_1}, ..., Y_{\mathcal{I}_{n_\mathrm{i}}} \}$ and identities for all voters in $\boldsymbol{\mathcal{V}}$ mapped to each identity provider in $\boldsymbol{\mathcal{I}}$. When being audited, the voter authorizer has to provide all identity tokens $\sigma_{\mathrm{id}; i, j}$ generated by each identity provider $\mathcal{I}_j \in \boldsymbol{\mathcal{I}}$ used to create the voter session item $b_{\mathrm{vs}; i}$. The auditor checks that the identity tokens are associated with the voter session item $H_i = \mathcal{H} (\sigma_{\mathrm{id}; i, 1} || ... || \sigma_{\mathrm{id}; i, n_{\mathrm{i}}})$, where $H_i$ is the authentication fingerprint from the item content. Also, it checks the validity of the identity tokens by $\mathsf{SigVer} (Y_{\mathcal{I}_j}, \sigma_{\mathrm{id}; i, j}, \mathcal{V}_i)$ (\cref{alg: sig ver}) and whether they are associated with an eligible voter, i.e., $\mathcal{V}_i \in \boldsymbol{\mathcal{V}}$. In case any of the validations fail, that discovers an attempt of the voter authorizer to create a fraudulent voter session.

Voter identities used for third-party identity providers are considered personal data and cannot be publicly disclosed on the bulletin board. Therefore, this auditing step is available only to election officers.

\subsection{Voter-specific verifications} \label{sec: voter-specific verifications}
During the voting process, voters can verify two aspects of their vote: that it is cast as intended and registered as cast. These verification steps help voters gain confidence that the election system behaves correctly, at least at processing their vote.


\subsubsection{Vote is cast as intended}
In order to verify that the vote is cast as intended, the voter needs to verify that the encryption mechanism of the voting application behaved correctly, i.e. the cryptogram contains the correct vote. After generating the vote cryptogram, the voter can choose whether to submit his vote or to challenge the encryption process. If the voter chooses to challenge the vote cryptogram, the system will print on the screen the value of the cryptogram $e = (R, C)$, the encryption key $Y_\mathrm{enc}$, and the randomizer $r$ used in the encryption. Note that \( r = r_0 + r_1 \), where $r_0$ is the randomizer generated by the server and $r_1$ is generated by the voting application. That means the server has to collaborate in this process for providing $r_0$.

Now the voter can use a secondary device to decrypt the content of the cryptogram by applying \( M \gets \mathsf{Dec} (e', r) \) (\cref{alg: dec}), where $e' = (Y_\mathrm{enc}, C)$. If the vote $M$ corresponds to the correct value that the voter intended to cast, then the voter gains confidence that the voting application behaves correctly.

If the voter chooses to challenge the vote cryptogram, then the cryptogram is invalidated because the value $r_0$ has been exposed. After challenging the vote cryptogram, the voter has to recast his vote by generating another vote cryptogram, which again, he has the option to challenge or submit.

 
\subsubsection{Vote is registered as cast} \label{sec: vote is registered as cast}
After posting a vote submission (vote cryptogram $e$ and the voter signature $\sigma$), the voter receives a receipt $\rho$ that certifies that his vote submission has been registered on the bulletin board at position $h_\mathrm{b}$. The receipt can be validated by checking \( \mathsf{SigVer} (Y_\mathrm{sig}, \rho; \sigma || h_\mathrm{b}) \) (\cref{alg: sig ver}).

Anytime during the election, the voter can check his receipt against the bulletin board, which responds with the appropriate vote submission, thus the voter gains confidence that his vote is registered as cast.

\subsection{Voter-specific verifications} \label{sec: voter-specific verifications}
During the voting process, voters can verify two aspects of their vote: that it is cast as intended and registered as cast. These verification steps help voters gain confidence that the election system behaves correctly, at least at processing their vote.


\subsubsection{Vote is cast as intended}
At the end of the vote cryptogram generation process (\cref{sec: vote cryptogram generation process}), the voter is presented with cryptogram $e = (R, C)$, which is the encryption of vote $V$ with the encryption key $Y_\mathrm{enc}$ and randomizer $r = r_\mathrm{v} + r_\mathrm{d}$, where $r_\mathrm{v}$ is known by the voting application and $r_\mathrm{d}$ is known only by the digital ballot box. Hence, it is the voting application and the digital ballot box that collectively perform the encryption of the voter's vote.

Because the vote is encrypted, the voter cannot tell whether cryptogram $e$ actually represents an encryption of vote $V$ or not. Therefore, to get convinced that the voting application and the digital ballot box behaved correctly during the vote cryptogram generation process, the voter can perform a challenge of the vote cryptogram, as presented in \cref{sec: challenging a vote cryptogram} to verify the activity of the voting application and digital ballot box.

If the voter chooses to challenge the cryptogram, a second device is used to perform all the cryptographic validations on behalf of the voter. Both randomizers $r_\mathrm{v}$ and $r_\mathrm{d}$ are sent securely from the voting application and the digital ballot box, respectively, to the external verifier application that runs on the secondary device. The verification application uses them to unpack the encrypted ballot and present the vote choices to the voter. The fully detailed process is shown in \cref{sec: challenging a vote cryptogram}.

If the vote corresponds to the voter's intended choices, then the voter gains confidence that the voting application behaved correctly.

If the vote does not correspond to the voter's intention, the auditing process provides evidence that the encrypted ballot has not been cast as intended. Note that, in this case, there is no distinction between the election system maliciously changing the voter's vote behind the scenes or the voter accidentally mischoosing the vote options.

After the voter has successfully audited the vote, it gets invalidated because the randomizer value $r_\mathrm{v} + r_\mathrm{d}$ has been exposed. Now, the voter has to regenerate a vote cryptogram (as presented in \cref{sec: vote cryptogram generation process}). The voter has, again, the option to challenge or submit. Voters should challenge again until they have enough confidence in the election system to cast their vote as intended.

 
\subsubsection{Vote is registered as cast} \label{sec: vote is registered as cast}
After posting a vote submission (vote cryptogram $e$ and the voter signature $\sigma$), the voter receives a receipt $\rho$ that certifies that his vote submission has been registered on the bulletin board at position $h_\mathrm{b}$. The receipt can be validated by checking \( \mathsf{SigVer} (Y_\mathrm{sig}, \rho; \sigma || h_\mathrm{b}) \) (\cref{alg: sig ver}).

Anytime during the election, the voter can check his receipt against the bulletin board, which responds with the appropriate vote submission, thus the voter gains confidence that his vote is registered as cast.

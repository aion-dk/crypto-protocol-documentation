\subsection{Communication channels} \label{sec: communication channels}
In AVX there will broadly speaking be three different types of channels which data is transferred through, secure, unsecure and authentic. What type of channel is used depends on the encryption available/possible for the data being sent.


\subsubsection{Unsecure channels}
An unsecure channel is used when the data is already safely encrypted on the application layer. 

This means that using a secure channel would add an additional layer of security but is not necessary as the data is already sufficiently protected. 

An example is when transferring the encrypted ballots then a unsecure channel can be used as the ballots are sufficiently protected from any outsider interference or manipulation.


\subsubsection{Secure channels}
A secure channel is used when the data is transferred is unencrypted and would be readable if an outsider were to gain access to the data. The channel must therefore be protected from any unwanted readers.

An example is when the election administrator initializes a new bulletin board on the \textit{Digital Ballot Box} $\mathcal{D}$ as there is no cryptographic keys established at this point between the two parties as this is their first communication with each other.


\subsubsection{Authentic channels}
An authentic channel is used when the data is transferred is public but still needs to be tamper proof. 

An example is when transferring public keys, these are by their name public, but needs to be received unchanged or they wont work.

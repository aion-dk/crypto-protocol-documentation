\subsection{Assumptions} \label{sec: assumptions}
Our election protocol assumes the following limitations about a potential attacker. The first category describes assumptions necessary for the election system to work correctly. The second category describes the assumptions required to fulfill the security properties described in \cref{sec: election properties}.

\paragraph{Assumptions related to the well-functioning of the election protocol:}
\begin{itemize}
    \item An attacker's computation power is assumed to be polynomially bound. 
    \item The elliptic curve discrete logarithm problem is assumed to be infeasible to break, as described in \cref{app: elliptic curve discrete logarithm problem}.
    \item All the secure communication channels are assumed to be private and tamper-resistant. The full list of secure communication channels used in the protocol consists of the following:
    \begin{itemize}
        \item generally, users are assumed to interact genuinely with their devices,
        \item the voting application assumes to receive correct inputs from voters,
        \item voters are not observed while interacting with their devices,
        \item the trustee application is assumed to correctly and secretly receive trustee inputs, including the share of the decryption key in the post-election phase \cref{sec: post-election phase},
        \item election officials can interact genuinely and privately with a browser to access the election administration service and voter authorizer,
        \item election officials can interact genuinely and privately with the auditing scripts during the administration auditing process (\cref{sec: administration auditing process}),
        \item credential authorities are assumed to secretly and correctly receive voter contact information from an election official during the voter credential distribution process (\cref{sec: voter credential distribution process}),
        \item credential authorities are assumed to secretly and correctly distribute voter credentials to the voters during the voter credential distribution process (\cref{sec: voter credential distribution process}),
        \item the voting application is assumed to genuinely and privately interact with all third-party identity providers.
    \end{itemize}
    \item All the authentic communication channels are assumed to be tamper-resistant. The full list of authentic communication channels used in the protocol consists of the following:
    \begin{itemize}
        \item the channel used between the election administration service and all the other services that need to get their public key authorized for their role, as described in \cref{sec: election configuration}. 
    \end{itemize}
    % \item The election administration service is assumed to be trustworthy, i.e., it secretly stores its private key $x_\mathcal{E}$ and acts according to the protocol only when triggered by an authorized election official.
    % \item The digital ballot box is assumed to be trustworthy, i.e., it secretly stores its private key $x_\mathcal{D}$ and acts according to the protocol.
\end{itemize}

\paragraph{Assumptions related to security properties:}
\begin{itemize}
    \item For achieving the eligibility property, we assume:
    \begin{itemize}
        \item when voter authorization mode is \textbf{credential-based}, there is at least one honest credential authority that generates and distributes correct voter credentials,
        \item when voter authorization mode is \textbf{identity-based}, there is at least one honest third-party identity provider that generates genuine identity tokens on successful voter authentication,
        \item the administration auditing process (\cref{sec: administration auditing process}) is trustworthy, i.e., an honest election official runs genuine auditing tools against real election data.
    \end{itemize}
    \item For achieving the privacy and anonymity properties, we assume:
    \begin{itemize}
        \item there are no more than $t$ malicious trustees or compromised trustee applications, where $t$ is the decryption threshold configured during the threshold ceremony (\cref{sec: threshold ceremony}),
        \item the voting application does not leak voter secret information.
    \end{itemize}
    \item For achieving verifiability, we assume:
    \begin{itemize}
        \item A voter uses at least one honest device, e.g., either the voting application device or the external verifier device,
        \item There are multiple external verifier deployments, out of which at least one is considered trustworthy by the voter.
    \end{itemize}
    \item For preserving the integrity, we assume that the integrity audit (\cref{sec: integrity of the bulletin board}) is trustworthy, i.e., an honest election official runs genuine auditing tools against real election data.
\end{itemize}

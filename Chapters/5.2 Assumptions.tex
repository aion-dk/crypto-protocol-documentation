\subsection{Assumptions} \label{sec: assumptions}
The threats against a system can be derived from the built-in trust-dependent assumptions. The following will list the processes in various phases that exhibit trust-dependent assumptions.


\subsubsection{Pre-election phase} 

\begin{enumerate}
\item It is assumed that the IT administrator is installing genuine code in the hosting environment
\item It is assumed that the IT administrator is creating the appropriate (trustworthy) users
\item It is assumed that users are securely connected to the Election Admin Application
\item It is assumed that the Election Administrator generates the eligible voters (credential-based mode)
\item It is assumed that the Election Administrator registers the public keys of the Voter Authorizer and the Identity Provider(s)
\item It is assumed that the Election Administrator uploads the eligible voter identities to the Voter Authorizer
\item It is assumed that the Election Administrator uploads all the public keys of the identity providers. All via secure channels.
\item It is assumed that the identity provider only generates identity tokens after successful authentication
\item It is assumed that the Election Administrator is sending the correct personal information (used for generating access codes) to the Credentials Authorities via secure channel
\item It is assumed that the Credentials Authorities will deliver the correct private key to voters via secure channel
\item It is assumed that the Election Administrator assigns the correct trustees
\item It is assumed that the Election System sends the trustee access token via secure channel
\item It is assumed that the Trustees publishes their public keys to the system via secure channel (https)
\item It is assumed that the Trustees create shares of private key (decryption key?) by sending secrets to each other encrypted by the public key of the others
\item It is assumed that the Election System will spawn a new Bulletin Board with the information gathered from previous steps via a secure channel
\item It is assumed that auditors will review published (on the Bulletin Board) changes to the election configuration
\end{enumerate}


\subsubsection{Election phase} 

\textbf{Voting}

\begin{enumerate}
\item It is assumed that the voter retain the private key for the entire election (credential-based mode)
\item It is assumed that a secure channel is used to transmit messages between Voting Application and Voter Authorizer (identity-based mode)
\item It is assumed that Identity Providers only release identity tokens on successful authentication (following OpenID Connect Protocol)
\item It is assumed that voters cast votes as intended (not coerced or part in deanonymization attack)
\item It is assumed that received cryptograms are empty
\item It is assumed that the Benaloh Challenge is conducted on a secondary device running an (uncompromised?) external verifier
\item It is assumed that the Voter is either trusting the cryptogram or is performing the Benaloh challenge
\item It is assumed that the Digital Ballot Box will only reveal the randomizer value to the External Verifier when the Voter decides to proceed with the Benaloh challenge.
\item It is assumed that the External Verifier will unpack the cryptogram by decrypting the incoming randomizer values from the Voter and the Digital Ballot Box
\item It is assumed that the Voter inputs the correct Verifier Tokens?
\item It is assumed that the Digital Ballot Box does not brute force the randomizer value from the Voter (possible?)
\end{enumerate}

\textbf{Signature Verification}

To be finished …

\textbf{Ballot Extraction}

To be finished …


\subsubsection{Post-Election Phase} 
To be finished …

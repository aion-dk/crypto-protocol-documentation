\subsection{Pre-election phase} \label{sec: pre-election phase}
During the \textit{pre-election phase} human election officers use the Election Administrator service to configure and setup a new election. That consists of the following steps:
\begin{itemize}
    \item initiate a new bulletin board as described in \cref{sec: digital ballot box initialization},
    \item define election level configuration, including election title, contest titles, candidates and other services required in the election, as decribed in \cref{sec: election configuration},
    \item define eligible voters and configure voter authorization mode as decribed in \cref{sec: voter authorization configuration},
    \item facilitate the threshold ceremony as described in \cref{sec: threshold ceremony},
    \item configure the election phase by setting up voting rounds as in \cref{sec: voting rounds}.
\end{itemize}


\subsubsection{Digital Ballot Box initialization} \label{sec: digital ballot box initialization}
An election officer decides the elliptic curve domain parameters $(p, a, b, G, q, h)$ and, according to that, the election administrator service generates a new key pair $(x_\mathcal{E}, Y_\mathcal{E}) \gets \mathsf{KeyGen}()$ (\cref{alg: key gen}), where $x_\mathcal{E}$ is its signing key and will be kept secret throughout the election, and $Y_\mathcal{E}$ is its public signature verification key. Then, the election administrator requests the digital ballot box to initialize a new bulletin board with the initial election meta-data configuration (including the elliptic curve domain parameters and the public key $Y_\mathcal{E}$). A predefined set of elliptic curve domain parameters is listed in \cref{app: supported elliptic curves}.

On this request, the digital ballot box generates a new key pair $(x_\mathcal{D}, Y_\mathcal{D}) \gets \mathsf{KeyGen}()$ (\cref{alg: key gen}), where $x_\mathcal{D}$ is its signing key and will be kept secret throughout the election period, and $Y_\mathcal{D}$ is its public signature verification key. Then, it spawns a new bulletin board by generating a genesis item as the first item of the board $(b_1, \rho_1) \gets \mathtt{WriteOnBoard}(\mathcal{D}, m_1, c_1, p_1)$ (protocol \ref{pro: write on board}), where $m_1 = $ "genesis", $p_1 = \varnothing$ and the content $c_1$ is constructed according to the rules specified in \cref{app: bulletin board item types}. Next, the digital ballot box returns to the election administrator with the freshly created genesis item $b_1$. From this point on, the election administrator service and the digital ballot box represent indentities $\mathcal{E}$ and $\mathcal{D}$ respectivly on the bulletin board.


\subsubsection{Election configuration} \label{sec: election configuration}
Once a bulletin board exists, the election administrator $\mathcal{E}$ can write all the configuration items on the bulletin board by following $\mathtt{WriteOnBoard}(\mathcal{E}, m_i, c_i, p_i)$ (protocol \ref{pro: write on board}). All items are computed and published one by one, based on the rules defined in \cref{app: bulletin board item types}. All items are signed with the election administrator signing key $x_\mathcal{E}$ and they reference as a parent the address of the previous configuration item, i.e. $p_i = h_{i-1}$. The items that make up the initial configuration are:
\begin{itemize}
    \item the election configuration item,
    \item contest configuration items for each contest and
    \item actor configuration items for each other service that needs to interact with the digiutal ballot box, e.g. the voter authorizer or the ballot adjudicator.
\end{itemize}

For each contest, the election officer has to configure a contest identifier, marking rules, result rules and a list of candidates represented by $\boldsymbol{m} = \{ m_1, ..., m_{n_\mathrm{c}} \}$, where $n_\mathrm{c}$ is the number of candidates.

For each actor, the Election Administrator service interacts with the other service, e.g. the voter authorizer, to generate its own key pair $(x_\mathcal{A}, Y_\mathcal{A}) \gets \mathsf{KeyGen}()$ (\cref{alg: key gen}). Value $x_\mathcal{A}$ is the voter authorizer signing key and will be kept secret throughout the election period while $Y_\mathcal{A}$ is its public signature verification key and it is shared with the election administrator. Next, the election administrator assignes the role of \textit{voter authorizer} to the public key $Y_\mathcal{A}$. Once the actor configuration item containing the public key $Y_\mathcal{A}$ is published on the bulletin board, the voter authorizer becomes indentity $\mathcal{A}$ and can interact with the digital ballot box.


\subsubsection{Voter authorization configuration} \label{sec: voter authorization configuration}
An election officer interacts with the Voter Authorizer service to configure the voter authentication mode, which can be either \textbf{credential-based} or \textbf{identity-based}. In order to define the list of eligible voters, the election officer has to perform some extra configuration depending on the voter authentication mode selected previously.

When \textbf{credential-based} voter authentication mode is enabled, the election officer sets up a set of credentials authorities $\boldsymbol{\mathcal{C}} = \{ \mathcal{C}_1, ..., \mathcal{C}_{n_\mathrm{c}} \}$ used for distributing voter credentials, each authority $\mathcal{C}_i \in \boldsymbol{\mathcal{C}}$ being specified to use a particular communication channel for distributing voter credentials, e.g. e-mail, post or SMS. Then the election officer provides the list of eligible voters $\boldsymbol{\mathcal{V}} = \{ \mathcal{V}_1, ..., \mathcal{V}_{n_\mathrm{v}} \}$, where each voter is defined by a unique identifier and a list of contact information for each communication channel used by the credentials authorities.

Next, credentials authorities and the voter authorizer perform the voter credential distribution process (describe in \cref{sec: voter credential distribution process}) and setup the public authentication key of each voter.

Finally, the voter authorizer writes the voter authorization configuration item on the bulletin board by performing $\mathtt{WriteOnBoard}(\mathcal{A}, m_i, c_i, p_i)$ (protocol \ref{pro: write on board}) based on the rules defined in \cref{app: bulletin board item types}. The item is signed by the voter authorizer signing key $x_\mathcal{A}$ and it contains the voter authentication mode.

When \textbf{identity-based} voter authentication mode is enabled, the election officer selects a list of third party identity providers $\boldsymbol{\mathcal{I}} = \{ \mathcal{I}_1, ..., \mathcal{I}_{n_\mathrm{i}} \}$ used for authenticating voters during the election phase. Then the election officer provides the list of eligible voters $\boldsymbol{\mathcal{V}} = \{ \mathcal{V}_1, ..., \mathcal{V}_{n_\mathrm{v}} \}$, where each voter is defined by a unique identifier and a list of identities supported by each of the identity providers.

Finally, the voter authorizer writes the voter authorization configuration item on the bulletin board by following $\mathtt{WriteOnBoard}(\mathcal{A}, m_i, c_i, p_i)$ (protocol \ref{pro: write on board}) based on the rules defined in \cref{app: bulletin board item types}. The item is signed by the voter authorizer signing key $x_\mathcal{A}$ and it contains the voter authentication mode and the list of identity providers $\mathcal{I}_j$, each defined by their public key $Y_{\mathcal{I}_j}$ and a URL, with $j \in \{ 1, ..., n_\mathrm{i} \}$.


\subsubsection{Voter credential distribution process} \label{sec: voter credential distribution process}
This process is only applicable if the vote authentication mode is \textbf{credential-based}, as described in \cref{sec: credential-based mode}.

Each credentials authority $\mathcal{C}_j \in \boldsymbol{\mathcal{C}}$, receives a list of voters consisting of contact details for each voter $\boldsymbol{a} = \{ a_1, ..., a_{n_\mathrm{v}} \}$ in form of e-mail addresses or postal addresses or phone numbers, depending on the credentials authority's communication channel. The credentials authority generates random credentials $c_{i, j} \in_\mathrm{R} \mathbb{B}^\ell$ for each voter, with $i \in \{ 1, ..., n_\mathrm{v} \}$. The credentials authority distributes the credential $c_{i, j}$ to that specific voter $\mathcal{V}_i$ (using voter's contact detail $a_i$) and appends the corresponding public authentication key $Y_{\mathrm{auth}; i, j}$ in the list of voters next to the specific voter $\mathcal{V}_i$, where $(x_{\mathrm{auth}; i, j}, Y_{\mathrm{auth}; i, j}) \gets \mathsf{Pass2Key}(c_{i, j})$ (\cref{alg: pass to key}).

Credential generation can be done in form of a random string of characters or numbers or a combination of them. In any case, it has to be bound by the level of entropy $\ell$. It is recommended that credentials are based on at least 80 bits of entropy, i.e. $\ell \geq 80$. That corresponds to a 14 characters alphanumeric code that has to be sent to the voter and inputted in the voting application.

All credentials authorities $\mathcal{C}_j \in \boldsymbol{\mathcal{C}}$ return to the voter authorizer the lists with voters contact details and public authentication keys $(a_i, Y_{\mathrm{auth}; i,j})$. The voter authorizer combines all keys received from all credentials authorities for each voter to form the voter's public authentication key $Y_{\mathrm{auth}; i} = \sum_{j=1}^{n_\mathrm{c}} Y_{\mathrm{auth}; i, j}$.

For authenticating to the voter authorizer, the voter $\mathcal{V}_i \in \boldsymbol{\mathcal{V}}$ has to input in the voting application all credentials $\{ c_{i, 1}, ..., c_{i, n_\mathrm{c}} \}$ received via different channels from all credentials authorities. The voting application will derive keys from each credential $(x_{\mathrm{auth}; i, j}, Y_{\mathrm{auth}; i, j}) \gets \mathsf{Pass2Key}(c_{i, j})$ (\cref{alg: pass to key}) and aggregate all of them to form the voter's private authentication key $x_{\mathrm{auth}; i} = \sum_{j=1}^{n_\mathrm{c}} x_{\mathrm{auth}; i, j} \pmod q$. The private authentication key is used to compute a proof of credentials $PK_\mathrm{auth}$, as described in \cref{sec: voter authorization procedure}, which is used to authenticate the voter.


\subsubsection{Threshold ceremony} \label{sec: threshold ceremony}
An election officer interacts with the election administrator service to define the lists of trustees $\boldsymbol{\mathcal{T}} = \{ \mathcal{T}_1, ..., \mathcal{T}_{n_\mathrm{t}} \}$. Then, the election administrator coordinates the threshold ceremony during which all trustees $\mathcal{T}_i \in \boldsymbol{\mathcal{T}}$ participate in the protocol from \cref{fig: threshold ceremony} described in \cref{app: threshold cryptosystem} in order to generate the election encryption key $Y_\mathrm{enc}$ and each trustee's share of the decryption key $sx_i$. The election officer sets the threshold value $t$, such that any $t$ out of the $n_\mathrm{t}$ trustees can perform decryption.

At the end of the ceremony, the election administrator writes the threshold configuration item on the bulletin board by following $\mathtt{WriteOnBoard}(\mathcal{E}, m_i, c_i, p_i)$ (protocol \ref{pro: write on board}) based on the rules defined in \cref{app: bulletin board item types}. The item contains:
\begin{itemize}
    \item election encryption key $Y_\mathrm{enc}$,
    \item threshold setup $t$-out-of-$n_\mathrm{t}$,
    \item public keys of each trustee $Y_{\mathcal{T}_i}$, with $i \in \{1, ..., n_\mathrm{t}\}$,
    \item public polynomial coefficients of each trustee $P_{\mathcal{T}_i,j}$, with $j \in \{1, ..., t-1 \}$.
\end{itemize}


\subsubsection{Voting rounds} \label{sec: voting rounds}
An election officer interacts with the election administrator service to define when the election phase is taking place, by setting a start date and an end date. Then, the election administrator $\mathcal{E}$ writes a voting round item on the bulletin board by following $\mathtt{WriteOnBoard}(\mathcal{E}, m_i, c_i, p_i)$ (protocol \ref{pro: write on board}) based on the rules defined in \cref{app: bulletin board item types}, specifying the start and end date and enabling all contest identifiers.

For a regular election, a single voting round is sufficient, but multiple voting rounds can be configured to start at different times and different contests could be enabled in each voting round.

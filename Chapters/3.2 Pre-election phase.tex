\subsection{Pre-election phase} \label{sec: pre-election phase}
During the \textit{pre-election phase} the \textit{Election Administrator} $\mathcal{E}$ has to configure the following parameters needed to set up an election:
\begin{itemize}
    \item the election start date and end date,
    \item the question of the contests, marking types and result rules,
    \item the possible answers (vote options) in all contests \( \boldsymbol{m} = \{m_1, ..., m_{n_\mathrm{c}}\} \), where $n_\mathrm{c}$ is the number of vote options, each representing a candidate name,
    \item the voter authentication method (\textbf{pre-election} or \textbf{on-demand}), which require some further configuration accordingly:
    \begin{itemize}
        \item \textbf{(pre-election)} the list of printing authorities \( \boldsymbol{\mathcal{P}} = \{\mathcal{P}_1, ..., \mathcal{P}_{n_\mathrm{p}}\} \) used for distributing voter credentials, each printing authority $\mathcal{P}_i$ being specified to use a particular communication channel for distributing voter credentials, e.g. e-mail, post or SMS
        \item \textbf{(on-demand)} the \textit{Voter Authorization Service} $\mathcal{A}$ which will generate its signing key pair \( (x_\mathcal{A}, Y_\mathcal{A}) \leftarrow \mathsf{KeyGen}() \) (\Cref{alg: key gen}), where $x_\mathcal{A}$ is the signing key which will be kept secret throughout the election period, and $Y_\mathcal{A}$ is the public signature verification key which will be shared.
        \item \textbf{(on-demand)} established connection with the \textit{Identity Providers} $\boldsymbol{\mathcal{I}} = \{ \mathcal{I}_1, ..., \mathcal{I}_{n_\mathrm{i}}\}$, get their public keys \{ \( Y_{\mathcal{I}_1}, ..., Y_{\mathcal{I}_{n_\mathrm{i}}} \) \} and exchange them with the \textit{Voter Authorization Service} $\mathcal{A}$. Each \textit{Identity Provider} $\mathcal{I}_i$ is supposed to use a unique kind of identity as a means of voter identification (e.g. email, phone number, social security number or a specific internal account). 
    \end{itemize}
    \item the list of eligible voters \( \boldsymbol{\mathcal{V}} = \{\mathcal{V}_1, ..., \mathcal{V}_{n_\mathrm{v}}\} \), that is defined differently, depending on the voter registration mode:
    \begin{itemize}
        \item \textbf{(pre-election)} Each voter $\mathcal{V}_i$ is defined by contact information for each of the communication channels that is used in the voter credential distribution process (\Cref{sec: voter credential distribution process}), e.g. an e-mail address, a postal address or a phone number.
        \item \textbf{(pre-election)} All printing authorities participate in the \textit{voter credential distribution process} (described in \Cref{sec: voter credential distribution process}) to distribute voter credentials. Public signature verification keys are computed for all voters $\{Y_1, ..., Y_{n_\mathrm{v}}\}$ and shared with the \textit{Election Administrator}.
        \item \textbf{(on-demand)} Each voter $\mathcal{V}_i$ is defined by unique identities that are used by all \textit{Identity Providers} $\boldsymbol{\mathcal{I}}$ and then verified by the \textit{Voter Authorization Service} $\mathcal{A}$.
        \item \textbf{(on-demand)} the \textit{Voter Authorization Service} $\mathcal{A}$ is loaded with the list of eligible voters $\boldsymbol{\mathcal{V}}$.
    \end{itemize}
    \item the lists of trustees \( \boldsymbol{\mathcal{T}} = \{\mathcal{T}_1, ..., \mathcal{T}_{n_\mathrm{t}}\} \) and the list of mix nodes \( \boldsymbol{\mathcal{M}} = \{\mathcal{M}_1, ..., \mathcal{M}_{n_\mathrm{m}}\} \). All trustees participate in the \textit{threshold ceremony}, described in \Cref{sec: threshold ceremony} in order to generate the election encryption key $Y_\mathrm{enc}$.
\end{itemize}

When all configuration is in completed, the \textit{Election Administrator} $\mathcal{E}$ initiates a new \textit{Digital Ballot Box} $\mathcal{D}$ and settles the initial configuration on the bulletin board. The process is described in \Cref{sec: digital ballot box initialization}. 


\subsubsection{Threshold Ceremony} \label{sec: threshold ceremony}
During the threshold ceremony, all \textit{Trustees} $\{\mathcal{T}_1, ..., \mathcal{T}_{n_\mathrm{t}}\}$, coordinated by the \textit{Election Administrator} $\mathcal{E}$, participate in the protocol from \Cref{fig: threshold ceremony} described in \Cref{app: elgamal threshold cryptosystem} in order to generate the election encryption key $Y_\mathrm{enc}$ and each \textit{Trustee's} share of the decryption key $sx_i$. The \textit{Election Administrator} sets the threshold value $t$, such that any $t$ out of the $n_\mathrm{t}$ trustees can perform decryption.

As a result of the threshold ceremony, for verification purpose, the \textit{Election Administrator} exports the following values:
\begin{itemize}
    \item the election encryption key $Y_\mathrm{enc}$,
    \item the threshold setup $t$-out-of-$n_\mathrm{t}$,
    \item the public keys of each trustee $Y_{\mathcal{T}_i}$, with $i \in \{1, ..., n_\mathrm{t}\}$,
    \item the public coefficient sets of each trustee $\{P_{i,1}, ..., P_{i,t-1}\}$.
\end{itemize}


\subsubsection{Voter Credential Distribution Process} \label{sec: voter credential distribution process}
This process is only applicable if the vote registration mode is \textbf{pre-election}, as described in \Cref{sec: pre-election mode}.

Each \textit{Printing Authority} \( \mathcal{P}_j \in \boldsymbol{\mathcal{P}}\), receives a list of voters consisting of contact details for each voter \( \boldsymbol{a} = \{a_1, ..., a_{n_\mathrm{v}}\} \) in form of e-mail addresses or postal addresses or phone numbers, depending on the printing authority's communication channel. The \textit{Printing Authority} generates a random key pair for each of them \( (x_{i, j}, Y_{i, j}) \leftarrow \mathsf{KeyGen}() \) (\Cref{alg: key gen}), with \( j \in \{ 1, ..., n_\mathrm{p} \} \) and \( i \in \{ 1, ..., n_\mathrm{v} \} \). The \textit{Printing Authority} distributes the secret key $x_{i, j}$ to that specific voter $\mathcal{V}_i$ (using voter's contact detail $a_i$) and appends the corresponding public key $Y_{i, j}$ in the list of voters next to the specific voter $\mathcal{V}_i$.

All \textit{Printing Authorities} return to the \textit{Election Administrator} the lists with voters contact details and public keys $(a_i, Y_{i,j})$. The \textit{Election Administrator} combines all public keys received from all \textit{Printing Authorities} for each \textit{Voter} to form the \textit{Voter's} public signature verification key \( Y_i = \sum_{j=1}^{n_\mathrm{p}} Y_{i, j} \).

For authenticating to the voting system, the \textit{Voter} \( \mathcal{V}_i \in \boldsymbol{\mathcal{V}} \) has to input in the browser all secret keys \( \{x_{i, 1}, ..., x_{i, n_\mathrm{p}}\} \) received via different channels from all \textit{Printing Authorities}. The browser will combine all of them to form the \textit{Voter's} signing key \( x_i = \sum_{j=1}^{n_\mathrm{p}} x_{i, j} \pmod q \) which authorizes the voter.


\subsubsection{Digital Ballot Box Initialization} \label{sec: digital ballot box initialization}
The \textit{Election Administrator} $\mathcal{E}$ requests the \textit{Digital Ballot Box} $\mathcal{D}$ to initialize a new bulletin board with the initial election meta-data configuration (including the voter authorization mode).

On this request, the \textit{Digital Ballot Box} generates a new key pair $(x_\mathcal{D}, Y_\mathcal{D}) \gets \mathsf{KeyGen}()$ (\Cref{alg: key gen}), where $x_\mathcal{D}$ is the signing key and will be kept secret throughout the election period, and $Y_\mathcal{D}$ is the public signature verification key. Then, it spawns a new bulletin board by generating a new \textit{genesis} item (following the protocol in \Cref{fig: protocol for writing an item on the bulletin board}) according to the rules specified in \Cref{sec: bulletin board event types}. Next, the \textit{Digital Ballot Box} returns to the \textit{Election Administrator} with the freshly created item.

Next, the \textit{Election Administrator} generates and writes all the configuration items on the bulletin board by following protocol in \Cref{fig: protocol for writing an item on the bulletin board}. All items are computed and posted one by one, based on the rules defined in \Cref{sec: bulletin board event types}. The items that make up the initial configuration are:
\begin{itemize}
    \item the election configuration item,
    \item a contest configuration item for each contest,
    \item the threshold configuration item,
    \item the voters configuration item, in case voter authorization mode is \textbf{pre-election},
    \item the voter authorization configuration item, in case voter authorization mode is \textbf{on-demand}.
\end{itemize}

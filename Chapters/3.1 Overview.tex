\subsection{Overview}

%\begin{figure}[H]
%    \centering
%    \includesvg{Pictures/}
%    \caption{}
%    \label{fig:}
%\end{figure}

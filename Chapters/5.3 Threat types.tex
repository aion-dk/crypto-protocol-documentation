\subsection{Threat types} 
The follow STRIDE threat types\cite{STRIDE} will be mapped against the trust-dependent assumptions:

% \begin{table}[H]
%     \begin{tabularx}{\textwidth}{|>{\hsize=.6\hsize}X|>{\hsize=.7\hsize}X|>{\hsize=1.7\hsize}X|} \hline
%         \textbf{Threat}        & \textbf{Desired Property} & \textbf{Description} \\ \hline
%         Spoofing               & Authenticity              & Spoofing is a situation in which a person or program successfully identifies as another by falsifying data, to gain an illegitimate advantage \\ \hline
%         Tampering              & Integrity                 & Tampering can refer to many forms of sabotage but the term is often used to mean intentional modification of products in a way that would make them harmful to the consumer \\ \hline
%         Repudiation            & Non-repudiation          & Non-repudiation refers to a situation where a statement's author cannot successfully dispute its authorship or the validity of an associated contract \\ \hline
%         Information disclosure & Confidentiality           & Information disclosure is a security violation, in which sensitive, protected or confidential data is copied, transmitted, viewed, stolen or used by an individual unauthorized to do so \\ \hline
%         Denial of Service      & Availability              & A denial-of-service attack is a cyber-attack in which the perpetrator seeks to make a machine or network resource unavailable \\ \hline
%         Elevation of Privilege & Authorization             & Privilege escalation is the act of exploiting a bug, a design flaw, or a configuration oversight o gain elevated access to resources that are normally protected from an application or user \\ \hline
%     \end{tabularx}
% \end{table}

\begin{table}[H]
\begin{tabular}{|l|l|l|}
\hline
\textbf{Threat}        & \textbf{Desired Property} & \textbf{Description}                                                                                                                                                                                                                                \\ \hline
Spoofing               & Authenticity              & \begin{tabular}[c]{@{}l@{}}Spoofing is a situation in which a person \\ or program successfully identifies as \\ another by falsifying data, to gain an \\ illegitimate advantage\end{tabular}                                                      \\ \hline
Tampering              & Integrity                 & \begin{tabular}[c]{@{}l@{}}Tampering can refer to many forms of \\ sabotage but the term is often used to \\ mean intentional modification of \\ products in a way that would make \\ them harmful to the consumer\end{tabular}                     \\ \hline
Repudiation            & Non-repudiability         & \begin{tabular}[c]{@{}l@{}}Non-repudiation refers to a situation \\ where a statement's author cannot \\ successfully dispute its authorship or \\ the validity of an associated contract\end{tabular}                                              \\ \hline
Information disclosure & Confidentiality           & \begin{tabular}[c]{@{}l@{}}Information disclosure is a security \\ violation, in which sensitive, protected \\ or confidential data is copied, \\ transmitted, viewed, stolen or used by \\ an individual unauthorized to do so\end{tabular}        \\ \hline
Denial of Service      & Availability              & \begin{tabular}[c]{@{}l@{}}A denial-of-service attack is a \\ cyber-attack in which the perpetrator \\ seeks to make a machine or network \\ resource unavailable\end{tabular}                                                                      \\ \hline
Elevation of Privilege & Authorization             & \begin{tabular}[c]{@{}l@{}}Privilege escalation is the act of \\ exploiting a bug, a design flaw, \\ or a configuration oversight o gain \\ elevated access to resources that are \\ normally protected from an \\ application or user\end{tabular} \\ \hline
\end{tabular}
\end{table}

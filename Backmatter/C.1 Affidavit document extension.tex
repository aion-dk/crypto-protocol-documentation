\subsection{Affidavit document extension} \label{app: affidavit document extension}


\subsubsection{General description}
The affidavit document extension is an optional feature that introduces an extra authentication factor for voters in the form of a document hand signed by the voter. The affidavit is encrypted and submitted together with the digital ballot. Affidavits are manually checked by election officers, which can either accept or reject a signed affidavit. The affidavit verification process can be done continuously throughout the election phase or as a bulk process right before the result is computed.

During the post-election phase, ballots get included in the tally only if they have been marked as accepted after the affidavit verification process.

The affidavit document comes in the form of a PDF document. It gets encrypted by the voting application and stored privately by the digital ballot box. During the affidavit verification process, the affidavit is decrypted by the election administrator service and presented to an election officer for assessment.

A fingerprint of the encrypted affidavit is included in the ballot submission, signed by the voting application, and published on the bulletin board for trackability and integrity reasons. The encrypted affidavit is not publicly accessible on the bulletin board for privacy concerns. 


\subsubsection{Election protocol modification}
In the pre-election phase, the election administrator $\mathcal{E}$ generates another key pair that is used for encrypting/decrypting the affidavit documents $(x_\mathrm{aff}, Y_\mathrm{aff}) \gets \mathsf{KeyGen}()$ (\cref{alg: key gen}), where $x_\mathrm{aff}$ is the private key used for decryption and will be kept secret by the election administrator, and $Y_\mathrm{aff}$ is the public key that voters will use to encrypt affidavits. The public key $Y_\mathrm{aff}$ is included in the election configuration item. Thus it is publicly accessible on the bulletin board.

During the election phase, a voter $\mathcal{V}_i$ must provide the affidavit document to the voting application. Before publishing the cast request item, as described in \cref{sec: vote confirmation receipt}, the voting application computes the encryption of the affidavit document $ea \gets \mathsf{SymEnc} (k_\mathrm{aff}, a)$ (\cref{alg: sym enc}), where $a \in \mathbb{B}^*$ is the byte representation of the affidavit document. The symmetric key $k_\mathrm{aff}$ is derived from the Diffie-Hellman key exchange mechanism $k_\mathrm{aff} \gets \mathsf{DerSymKey} (x_i, Y_\mathrm{aff})$ (\cref{alg: der sym key}), where $x_i$ is the voter's private key (as described in \cref{sec: voter authorization procedure}) and $Y_\mathrm{aff}$ is the affidavit public encryption key that is extracted from the election configuration item.

A fingerprint of the encrypted affidavit (i.e., $\mathcal{H} (ea)$) is included in the content of the cast request item. The encrypted affidavit is submitted alongside the cast request item to the digital ballot box, which validates the request and appends the item on the bulletin board if the fingerprint matches the encrypted affidavit. Note that the digital ballot box cannot read the contents of the affidavit document, as it does not know the affidavit decryption key $x_\mathrm{aff}$.

To perform the affidavit verification process (\cref{fig: affidavit verification process}), an election officer interacts with the election administrator service, which gets the encrypted affidavit $ea$ of the voter $\mathcal{V}_i$, together with the ancestry of items $\alpha_\mathrm{cr} = \alpha_\mathrm{cnf} \cup \{ b_\mathrm{vs}, b_\mathrm{vec}, b_\mathrm{sec}, b_\mathrm{bc}, b_\mathrm{cr} \}$ from the digital ballot box. The election administrator validates the ancestry by $\mathsf{AncestryVer} (\alpha_\mathrm{cr}, \varnothing)$ (\cref{alg: ancestry ver}), and checks that $c_\mathrm{cr} = \mathcal{H} (ea)$, where $c_\mathrm{cr}$ is the content of the cast request item $b_\mathrm{cr}$. If all validations succeed, it decrypts the affidavit by $a \gets \mathsf{SymDec} (k_\mathrm{aff}, ea)$ (\cref{alg: sym dec}). The symmetric key is derived by $k_\mathrm{aff} \gets \mathsf{DerSymKey} (x_\mathrm{aff}, Y_i)$, where the public key $Y_i$ is defined in the voter session item $b_\mathrm{vs}$. Value $a$ is decoded as a PDF file and rendered to the election officer for assessment.

The election officer decides whether to accept or reject the affidavit. If accepted, the election administration service $\mathcal{E}$ interacts with the digital ballot box to append a ballot accepted item $b_\mathrm{ba}$ on the bulletin board by running protocol \ref{pro: write on board} $\mathtt{WriteOnBoard}(\mathcal{E}, m_\mathrm{ba}, c_\mathrm{ba}, p_\mathrm{ba})$, where $m_\mathrm{ba} = $ "ballot accepted", the content is empty, and the parent is the address of the cast request item $b_\mathrm{cr}$. If rejected, the same interaction happens to append a ballot rejected item $b_\mathrm{br}$ by $\mathtt{WriteOnBoard}(\mathcal{E}, m_\mathrm{br}, c_\mathrm{br}, p_\mathrm{br})$, where $m_\mathrm{br} = $ "ballot rejected", the content $c_\mathrm{br}$ contains the rejection reason, and the parent is the address of the cast request item $b_\mathrm{cr}$.

\begin{figure}[ht]
    \centering
    \begin{tikzpicture}[framed,
            node distance=0,
            % every node/.style={draw}
            ]{
            
        % Actors
        \node[title, above, anchor=north east] (ea) {
            \textbf{Election Administrator $\mathcal{E}$}};
        \node[title, right=of ea] (dbb) {
            \textbf{Digital Ballot Box $\mathcal{D}$}};
        
        % internal knowledge
        \node[ik] at (ea.south) (ea_ik) {
            internal knowledge: $x_\mathcal{E}$, $x_\mathrm{aff}$
            };
        \node[ik] at (dbb.south) (dbb_ik) {
            internal knowledge: $\boldsymbol{b} = \{ b_1, ..., b_{k-1} \}$, \\
            where $\alpha_\mathrm{cnf}, \{ b_\mathrm{vs}, b_\mathrm{vec}, b_\mathrm{sec}, b_\mathrm{bc}, b_\mathrm{cr} \} \subset \boldsymbol{b}$
            };
        
        %All content
        \node[arrow, towards_left, spaced, between={ea.center}{dbb.center}] at (dbb_ik.south -| ea.center) (a_1) {
            $ea$, $\alpha_\mathrm{cr} = \alpha_\mathrm{cnf} \cup \{ b_\mathrm{vs}, b_\mathrm{vec}, b_\mathrm{sec}, b_\mathrm{bc}, b_\mathrm{cr} \}$
            };
        \node[block, keep_left, spaced] at (a_1.south -| ea.west) (ea_1) {
            $h_\mathrm{cr} \gets $ the address of $b_\mathrm{cr}$, \\
            $c_\mathrm{cr} \gets $ the content of $b_\mathrm{cr}$ \\
            $Y \gets $ the content of $b_\mathrm{vs}$ \\ [7pt]
            verify $\mathsf{AncestryVer}(\alpha_\mathrm{cr}, \varnothing)$ and $c_\mathrm{cr} = \mathcal{H} (ea)$ \\ [7pt]
            $k_\mathrm{aff} \gets \mathsf{DerSymKey} (x_\mathrm{aff}, Y)$ \\
            $a \gets \mathsf{SymDec} (k_\mathrm{aff}, ea)$ \\ [7pt]
            Decide validity of affidavit $a$
            };
        \node[banner, keep_left, spaced, between={ea.west}{dbb.east}] at (ea_1.south -| ea.west) (b_1) {
            $\mathcal{E}$ and $\mathcal{D}$ perform protocol \ref{pro: write on board} to write a ballot accepted $b_\mathrm{ba}$ or \\
            ballot rejected item $b_\mathrm{br}$ as the $k^\mathrm{th}$ item of $\boldsymbol{b}$
            };
        
        % Arrows and lines
        \draw[dashed] (ea.south west)--(dbb.south east);    
        \draw[dashed] (dbb_ik.south -| ea.west)--(dbb_ik.south -| dbb.east);

        \draw[dotted] (b_1.north west)--(b_1.north east);
        \draw[dotted] (b_1.south west)--(b_1.south east);

        \draw[densely dotted] (dbb_ik.south -| ea.center)--(ea_1.north -| ea.center);
        \draw[densely dotted] (ea_1.south -| ea.center)--(b_1.north -| ea.center);

        \draw[densely dotted] (dbb_ik.south -| dbb.center)--(a_1.south east);
        }
    \end{tikzpicture}
    \caption{Affidavit verification process}
    \label{fig: affidavit verification process}
\end{figure}

The affidavit extension also impacts the cleansing procedure (\cref{sec: cleansing procedure}), such that only ballots that have been marked as accepted (i.e., that are followed by a ballot accepted item) will be included in the \textit{initial mixed board}.


\subsubsection{Impact on election properties}
The affidavit extension impacts the eligibility property of the election system. This feature acts as an extra authentication step for voters. The authentication validation is a manual process done by election officers, at a later time than the ballot submission. 

Moreover, the election officials' decisions to accept or reject affidavits cannot be audited. Therefore, the system achieves the eligibility property on an extra assumption that the election officials are trustworthy for the affidavit verification process.

\subsection{Affidavit document extension} \label{app: affidavit document extension}
The affidavit document extension is an optional feature that includes an extra authentication factor in form of an affidavit that is hand signed by the voter. The affidavit accompanies the encrypted digital ballot.


\subsubsection{General description}
The affidavit feature is an extra verifiability check that is implemented upon the election phase when a \textit{Voter} $\mathcal{V}_i$ wants to submit his vote as described in \cref{sec: election phase} (bullet point 3) then an additional step is required by the user which is to provide a signed affidavit to the voting application. During the election phase, after a voter has submitted his vote and an affidavit, the affidavit is manually checked by an \textit{Election Administrator} $\mathcal{E}$ who then either accept or rejects the signature. In the post-election phase only cryptograms that are accompanied by accepted affidavits will pass through the cleansing phase to be part of the mixing and decryption phases as described in \cref{sec: cleansing procedure}.

The file is then encrypted with the encryption key $Y_\mathrm{aff}$ that is written from the election configuration. The file is encrypted using symmetric encryption as described in (ref to symmetric algorithm) where it is also described how $Y_\mathrm{aff}$ is transformed into a symmetric key.

The encrypted affidavit is then sent to the \textit{Digital Ballot Box} $\mathcal{D}$ and stored encrypted privately by the \textit{Digital Ballot Box} $\mathcal{D}$ and a fingerprint of the encrypted affidavit is posted on the bulletin board together with the vote cryptogram. This also means that the affidavit extension modifies the cast-request item described in \cref{sec: writing on the bulletin board} as the item now contains a field with this aforementioned fingerprint.


The affidavit is verified as being genuine, that is generated in a appropriate voting application, by (what parameter?). After being verified the affidavit is unencrypted and the contents are manually accepted or rejected by an \textit{Election Administrator} $\mathcal{E}$.

During the affi davit process an id is generated for the user $\mathcal{V}_{{F}_i}$ which the \textit{Election Administrator} $\mathcal{E}$ then uploads after the affi davit is submitted.


\subsubsection{How it impacts our protocol}
In the pre-election phase the \textit{Election Administrator} $\mathcal{E}$ has to generate an asymmetric key pair for the encryption and decryption of the affidavits. $(x_\mathrm{aff}, Y_{\mathcal{E}_f}) \leftarrow \mathsf{KeyGen}()$ (\cref{alg: key gen}), where $x_{\mathcal{E}_f}$ is the private key and will be kept private by the \textit{Election Administrator} $\mathcal{E}$, and $Y_{\mathcal{E}_f}$ is the public encryption key that the \textit{Voter} $\mathcal{V}$ will use to encrypt his affidavits. The public key is included in the election configuration so a voter can access it and use it to encrypt his affidavit.

The bulletin board has additional information in form of the fingerprint of the encrypted affidavit and the \textit{Digital Ballot Box} $\mathcal{D}$ has an extra item stored. The extra item is either an acceptance item or an rejection item depending on whether or not the affidavit is accepted. The acceptance item contains nothing and the item type in itself is the proof of acceptance. The rejection item contains a rejection reason $m_i$. 

The affidavit extension also impacts the Cleansing Procedure (\cref{sec: cleansing procedure}) as only ballots that has been marked as accepted will be included in the \textit{initial mixed board} list that is going to be mixed and decrypted. An accepted cast request item has an accepted item appended and pointing towards it. The rejected affidavits on the board are documented as discarded on the bulletin board. 


\subsubsection{How it impacts our election properties}
The affidavit extension impacts the eligibility election property as a voter has the additional requirement of the affidavit which has to be approved by an official administrator. This naturally makes the \textit{Election Administrator} $\mathcal{E}$ have additional responsibilities as he verifies voters and the \textit{Election Administrator} $\mathcal{E}$ has to be trusted as the decision on the affidavit cannot be verified.
The election property of verifiability is also affected due to the mentioned problem above that the decision on the affidavit cannot be verified.

This extension adds extra dependencies between

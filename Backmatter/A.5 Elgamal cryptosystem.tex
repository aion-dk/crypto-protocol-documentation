\subsection{Elgamal cryptosystem} \label{app: elgamal cryptosystem}
The \textit{Elgamal cryptosystem} is an asymmetric, randomized encryption scheme, where anybody can encrypt a message using the encryption key resulting in a \textit{cryptogram}, while only the one in possession of the decryption key can extract the message of a cryptogram. The scheme consists of a triple (\textsf{KeyGen}, \textsf{Enc}, \textsf{Dec}) of algorithms that work on the elliptic curve described in \cref{app: elliptic curve discrete logarithm problem}. The scheme is considered secure under the \textit{discrete logarithm assumption}.

An Elgamal key pair is a tuple $(x, Y) \gets \mathsf{KeyGen}()$ (\cref{alg: key gen}), where $x$ is a randomly chosen scalar representing the private decryption key and $Y$ is an elliptic curve point corresponding to the public encryption key.

\begin{algorithm}[ht]
    \DontPrintSemicolon
    \caption{$\mathsf{KeyGen}()$}
    \label{alg: key gen}
    
    $x \in_\mathrm{R} \mathbb{Z}_q$ \;
    $Y \gets [x]G$ \;
    \Return{$(x, Y)$} \tcp*{$(x, Y) \in \mathbb{Z}_q \times \mathbb{P}$}
\end{algorithm}

The encryption algorithm $e = (R, C) \gets \mathsf{Enc} (Y, M; r)$ (\cref{alg: enc}) can be used by anybody in possession of the public encryption key $Y$ to generate a cryptogram on a message $M$, using the randomizer $r$. The cryptogram $e$ can be decrypted back to the original message $M$, only by the one in possession of the private decryption key $x$ in the decryption algorithm $M \gets \mathsf{Dec} (x, e)$ (\cref{alg: dec}). Note that both $\mathsf{Enc}$ and $\mathsf{Dec}$ work on messages that are formatted as elliptic curve points $M \in \mathbb{P}$.

For sake of notation, we define $\mathbb{E} = \mathbb{P} \times \mathbb{P}$ as the set of all possible cryptograms.

\begin{algorithm}[ht]
    \DontPrintSemicolon
    \caption{$\mathsf{Enc} (Y, M; r)$}
    \label{alg: enc}
    \KwData{The encryption key $Y \in \mathbb{P}$}
    \KwMoreData{The message $M \in \mathbb{P}$}
    \KwMoreData{The randomizer $r \in \mathbb{Z}_q$}
    
    $R \gets [r]G$ \;
    $S \gets [r]Y$ \;
    $C \gets S + M$ \;
    $e \gets (R, C)$ \;
    \Return{$e$} \tcp*{$e \in \mathbb{E}$}
\end{algorithm}

\begin{algorithm}[ht]
    \DontPrintSemicolon
    \caption{$\mathsf{Dec} (x, e)$}
    \label{alg: dec}
    \KwData{The decryption key $x \in \mathbb{Z}_q$}
    \KwMoreData{The cryptogram $e = (R, C) \in \mathbb{E}$}
    
    $S \gets [x]R$ \;
    $M \gets C - S$ \;
    \Return{$M$} \tcp*{$M \in \mathbb{P}$}
\end{algorithm}


\subsubsection{Homomorphic Encryption}
Elgamal encryption based on elliptic curve cryptographic primitive is a \textit{homomorphic} encryption scheme concerning point addition. That means the component-wise addition of two cryptograms would result in a new, valid cryptogram that contains the two messages added together.
\[
\mathsf{Enc}(Y, M_1; r_1) + \mathsf{Enc}(Y, M_2; r_2) = \mathsf{Enc}(Y, M_1 + M_2; r_1 + r_2)
\]

The resulting encryption of the homomorphic addition of two cryptograms is $e' = (R', C') \gets \mathsf{HomAdd} (e_1; e_2)$ (\cref{alg: hom add}).

\begin{algorithm}[ht]
    \DontPrintSemicolon
    \caption{$\mathsf{HomAdd}(e_1; e_2)$}
    \label{alg: hom add}
    \KwData{The first cryptogram $e_1 = (R_1, C_1) \in \mathbb{E}$}
    \KwMoreData{The second cryptogram $e_2 = (R_2, C_2) \in \mathbb{E}$}
    
    $R' \gets R_1 + R_2$ \;
    $C' \gets C_1 + C_2$ \;
    $e' \gets (R', C')$\;
    \Return{$e'$} \tcp*{$e' \in \mathbb{E}$}
\end{algorithm}

Following the procedure above, a given cryptogram $e = (R, C)$ can be \textit{re-encrypted} by homomorphically adding it to an \textit{empty cryptogram} (i.e. an encryption of the neutral point $\mathcal{O}$) with randomizer $r' \in_\mathrm{R} \mathbb{Z}_q$. The result is a new, randomly different cryptogram that contains the same message $M$. The process of generating the new cryptogram $e' = (R', C') \gets \mathsf{ReEnc} (Y, e; r')$ is described in \cref{alg: re enc}.

\begin{algorithm}[ht]
    \DontPrintSemicolon
    \caption{$\mathsf{ReEnc} (Y, e; r')$}
    \label{alg: re enc}
    \KwData{The encryption key $Y \in \mathbb{P}$}
    \KwMoreData{The initial cryptogram $e = (R, C) \in \mathbb{E}$}
    \KwMoreData{The new randomizer $r' \in \mathbb{Z}_q$}
    
    $e_2 \gets \mathsf{Enc}(Y, \mathcal{O}; r')$ \tcp*{\cref{alg: enc}}
    $e' \gets \mathsf{HomAdd} (e, e_2)$ \tcp*{\cref{alg: hom add}}
    \Return{$e'$} \tcp*{$e' \in \mathbb{E}$}
\end{algorithm}

Usually, a re-encrypted cryptogram comes with a re-encryption proof to assure that the content of the cryptogram has not been changed. The proof is a non-interactive discrete logarithm equality proof (described in \cref{app: discrete logarithm proofs}) $PK = (K, c, r) \gets \mathsf{DLProve} (r', \{ G, Y \})$ (\cref{alg: dl prove}), while the proof verification algorithm is $\mathsf{DLVer} (PK, \{ G, Y \}; \{ R'-R, C'-C \})$ (\cref{alg: dl ver}).

Naturally, cryptogram addition can be expanded to multiplication to achieve the fact that $\mathsf{Enc}(Y, M; r) + \mathsf{Enc}(Y, M; r) = 2 \cdot \mathsf{Enc}(Y, M; r) = \mathsf{Enc}(Y, [2]M; 2 \cdot r)$. The resulting encryption of the homomorphic multiplication of a cryptogram is $e' = (R', C') \gets \mathsf{HomMul}(e; n)$ (\cref{alg: hom mul}).

\begin{algorithm}[ht]
    \DontPrintSemicolon
    \caption{$\mathsf{HomMul} (e; n)$}
    \label{alg: hom mul}
    \KwData{The initial cryptogram $e = (R, C) \in \mathbb{E}$}
    \KwMoreData{The multiplication factor $n \in \mathbb{Z}$}
    
    $R' \gets [n]R$ \;
    $C' \gets [n]C$ \;
    $e' \gets (R', C')$ \;
    \Return{$e'$} \tcp*{$e' \in \mathbb{E}$}
\end{algorithm}


\subsubsection{Elgamal Threshold Cryptosystem} \label{app: elgamal threshold cryptosystem}
A $t$ out of $n$ threshold cryptosystem is a homomorphic encryption scheme where the decryption key is split among $n$ key holders, called \textit{trustees}. Anybody can encrypt a message using the encryption key. The decryption of a message happens during a process in which at least $t$ trustees have to collaborate in a cryptographic protocol. It is recommended that $t \geq 2/3 \cdot n$. The entire scheme was introduced in \cite{Pedersen91} which is based on mathematical principles of the threshold cryptosystem \cite{Desmedt89, Shamir79}.

The key generation process concludes with the following $(sx_1, ..., sx_n, Y)$, where $Y$ is the public encryption key and each $sx_i$ is a private share of the decryption key, one for each of the $n$ trustees. The process is performed by all trustees while being facilitated by a central entity called \textit{the server}. The entire process is described by the protocol called \textit{the threshold ceremony} illustrated in \cref{fig: threshold ceremony}.

During the \textit{threshold ceremony}, all trustees generate a private-public key pair $(x_i, Y_i) \gets \mathsf{KeyGen}()$ (\cref{alg: key gen}) and publish to the server their public keys. The public encryption key is computed by the sum of the public keys of all trustees $Y = \sum_{i=1}^n Y_i$, while nobody being in the possession of the decryption key $x = \sum_{i=1}^n x_i$, because all $x_i$ are secret. Instead, all trustees work together to distribute $x$ such that any $t$ trustees can find it when necessary. 

Each trustee $\mathcal{T}_i$ generates a polynomial function of degree $t-1$
\[
f_i(z) = x_i + p_{i, 1} \cdot z + ... + p_{i, t-1} \cdot z^{t-1}
\]
and publishes to the server the points $\{ P_{i, 1}, ..., P_{i, t-1} \}$, where each private-public coefficient pair is $(p_{i, k}, P_{i, k}) \gets \mathsf{KeyGen}()$ (\cref{alg: key gen}), with $k \in \{ 1, ..., t-1 \}$.

When all trustees have published their public coefficients, each trustee computes a \textit{partial secret share of the decryption key} for each of the other trustees by $s_{i, j} \gets f_i(j)$, and encrypts them with that specific trustee's public key $c_{i, j} \gets \mathsf{TxtEnc}(Y_j, s_{i, j})$ (\cref{alg: txt enc}). Finally, all trustees publish to the server all encrypted \textit{partial secret shares of the decryption key}.

By encrypting the partial secret shares with different trustee's public key, we make sure that only that specific trustee can read his \textit{partial secret shares of the decryption key}. This procedure is a small deviation from \cite{Pedersen91}, which we introduced in order to simulate a secret communication channel between every two trustees.

Finally, each trustee $\mathcal{T}_i$ downloads from the server their encrypted \textit{partial secret shares} $c_{j, i}$, with $j \in \{ 1, ..., n \}$, decrypts them $s_{j, i} \gets \mathsf{TxtDec}(x_i, c_{j, i})$ (\cref{alg: txt dec}) and validates that they are consistent with the polynomial coefficients of the other trustees $[s_{j, i}]G = Y_j + \sum_{k=1}^{t-1} [i^k] P_{j, k}$. If all \textit{partial secret shares} validate, then trustee $\mathcal{T}_i$ computes his \textit{secret share of the decryption key} by $sx_i \gets \sum_{j=1}^n s_{j, i}$ and stores it privately until needed.

At the end of the \textit{threshold ceremony}, for each trustee $\mathcal{T}_i$, with $i \in \{ 1, ..., n \}$, the \textit{public share of the decryption key} ($sY_i = [sx_i]G$) is publicly computable by the following:
\[
sY_i \gets \sum\limits_{j=1}^{n} (Y_j + \sum\limits_{k=1}^{t-1} [i^k]P_{j, k}).
\]

The encryption algorithm of the threshold cryptosystem is identical to the algorithm described in \cref{app: elgamal cryptosystem}: $e = (R, C) \gets \mathsf{Enc}(Y, M; r)$.

\begin{figure}[ht]
    \centering
    \begin{tikzpicture}[framed, node distance=0,
            spaced/.style={yshift=-6},
            % every node/.style={draw}
            ]{
            
        % Actors
        \node[title, above, anchor=north east] (s) {
            \textbf{Server}};
        \node[title, right=of s] (t) {
            \textbf{Trustee $\mathcal{T}_i$}};
        
        %All content
        \node[block, spaced, keep_left] at (s.south -| s.west) (s_1) {
            $\boldsymbol{Y} \gets \{ \}$, $\boldsymbol{P} \gets \{ \}$, $\boldsymbol{c} \gets \{ \}$
            };
        \node[arrow, towards_right, immediate, between={s.center}{t.center}] at (s_1.south -| s.south) (a_1) {
            \hspace{20pt} invitation
            };
        \node[block, spaced, keep_middle] at (a_1.south -| t.center) (t_1) {
            $(x_i, Y_i) \gets \mathsf{KeyGen}()$
            };
        \node[arrow, towards_left, immediate, between={s.center}{t.center}] at (t_1.south -| s.center) (a_2) {
            $Y_i$ \hspace{20pt}
            };
        \node[block, spaced, keep_middle] at (a_2.south -| s.center) (s_2) {
            $\boldsymbol{Y} \gets \boldsymbol{Y} \cup \{ Y_i \}$
            };
        \node[banner, keep_left, spaced, between={s.west}{t.east}] at (s_2.south -| s.west) (b_1) {
            when all $\mathcal{T}_i$ have published $Y_i$
            };
        \node[block, keep_middle, spaced] at (b_1.south -| s.center) (s_3) {
            set $t \in [\frac{2}{3}n, ..., n]$
            };
        \node[arrow, towards_right, immediate, between={s.center}{t.center}] at (s_3.south -| s.center) (a_3) {
            \hspace{20pt} $t$, $\boldsymbol{Y} = \{ Y_1, ..., Y_n \}$
            };
        \node[block, keep_right, spaced] at (a_3.south -| t.east) (t_2) {
            $(p_{i, k}, P_{i, k}) \gets \mathsf{KeyGen}()$, with $k \in \{ 1, ..., t-1 \}$ \\
            $f_i(a) = x_i + \sum\limits_{k=1}^{t-1} p_{i, k} \cdot a^k \pmod q$ \\
            $s_{i, j} \gets f_i(j)$, with $j \in \{ 1, ..., n \}$ \\
            $c_{i, j} \gets \mathsf{TxtEnc} (Y_j, s_{i, j})$
            };
        \node[arrow, towards_left, immediate, between={s.center}{t.center}] at (t_2.south -| s.center) (a_4) {
            $P_{i, k}$, $c_{i, j}$ \hspace{100pt}
            };
        \node[block, keep_left, spaced] at (a_4.south -| s.west) (s_4) {
            $\boldsymbol{P} \gets \boldsymbol{P} \cup \{ P_{i, k} \}$, with $k \in \{ 1, ..., t-1 \}$ \\
            $\boldsymbol{c} \gets \boldsymbol{c} \cup \{ c_{i, j} \}$, with $j \in \{ 1, ..., n \}$
            };
        \node[banner, keep_left, spaced, between={s.west}{t.east}] at (s_4.south -| s.west) (b_2) {
            when all $\mathcal{T}_i$ have published $P_{i, k}$ and $c_{i, j}$
            };
        \node[arrow, towards_right, between={s.center}{t.center}] at (b_2.south -| s.south) (a_5) {
            $\boldsymbol{P} = \{ P_{1, 1}, ..., P_{n, t-1} \}$, $\{ c_{1, i}, ..., c_{n, i} \}$
            };
        \node[block, keep_right, spaced] at (a_5.south -| t.east) (t_3) {
            $s_{j, i} \gets \mathsf{TxtDec} (x_i, c_{j, i})$, with $j \in \{ 1, ..., n \}$ \\ [7pt]
            verify that $[s_{j, i}]G = Y_j + \sum\limits_{k=1}^{t-1} [i^k]P_{j, k}$ then: \\ [7pt]
            $sx_i \gets \sum\limits_{j=1}^{n} s_{j, i} \pmod q$
            };
        \node[arrow, towards_left, immediate, between={s.center}{t.center}] at (t_3.south -| s.center) (a_6) {
            validation \hspace{80pt}
            };
        \node[banner, keep_left, spaced, between={s.west}{t.east}] at (a_6.south -| s.west) (b_3) {
            when all $\mathcal{T}_i$ have validated
            };
        \node[block, keep_middle, spaced] at (b_3.south -| s.center) (s_5) {
            $Y \gets \sum\limits_{i=1}^{n} Y_i$
            };
        
        % Arrows and lines
        \draw[dashed] (s.south west)--(t.south east);

        \draw[dotted] (b_1.north west)--(b_1.north east);
        \draw[dotted] (b_1.south west)--(b_1.south east);
        \draw[dotted] (b_2.north west)--(b_2.north east);
        \draw[dotted] (b_2.south west)--(b_2.south east);
        \draw[dotted] (b_3.north west)--(b_3.north east);
        \draw[dotted] (b_3.south west)--(b_3.south east);

        \draw[densely dotted] (s_1.south -| s.center)--(s_2.north -| s.center);
        \draw[densely dotted] (s_2.south -| s.center)--(b_1.north -| s.center);
        \draw[densely dotted] (b_1.south -| s.center)--(s_3.north -| s.center);
        \draw[densely dotted] (s_3.south -| s.center)--(s_4.north -| s.center);
        \draw[densely dotted] (s_4.south -| s.center)--(b_2.north -| s.center);
        \draw[densely dotted] (b_2.south -| s.center)--(b_3.north -| s.center);
        \draw[densely dotted] (b_3.south -| s.center)--(s_5.north -| s.center);

        \draw[densely dotted] (a_1.south east)--(t_1.north -| t.center);
        \draw[densely dotted] (t_1.south -| t.center)--(a_2.south east);
        \draw[densely dotted] (a_3.south east)--(t_2.north -| t.center);
        \draw[densely dotted] (t_2.south -| t.center)--(a_4.south east);
        \draw[densely dotted] (a_5.south east)--(t_3.north -| t.center);
        \draw[densely dotted] (t_3.south -| t.center)--(a_6.south east);
        }
    \end{tikzpicture}
    \caption{Threshold ceremony}
    \label{fig: threshold ceremony}
\end{figure}

\clearpage
The decryption protocol of the threshold cryptosystem is inspired from paper \cite{Desmedt89}. At least $t$ trustees are needed to collaborate in the protocol described in \cref{fig: threshold decryption} in order to extract the message $M$ of a cryptogram $e = (R, C)$. We define $\tau \subset \{ 1, ..., n \}$ to be the subset of trustees that do participate in the decryption protocol, with $|\tau| \geq t$. 

Each trustee $\mathcal{T}_i$, with $i \in \tau$, computes a partial decryption $S_i \gets [sx_i]R$ and sends it to the server, where $sx_i$ is trustee's share of the decryption key. The trustee also publishes a proof of correct decryption in form of a non-interactive discrete logarithm zero knowledge proof $PK \gets \mathsf{DLProve} (sx_i, \{ G, R \})$ (\cref{alg: dl prove}).

When receiving a partial decryption from a trustee $\mathcal{T}_i$, the server accepts it if the proof of correct decryption validates by $\mathsf{DLVer} (PK, \{ G, R \}, \{ sY_i, S_i \})$ (\cref{alg: dl ver}), where $sY_i$ is trustee's \textit{public share of the decryption key}. After it received valid, partial decryptions from all trustees $\mathcal{T}_i$, with $i \in \tau$, the server aggregates all partial decryptions together to finalize the decryption and to output the message $M$. The aggregation process from \cite{Desmedt89} is described as follows:

Basically, $M = C - [x]R$, where $x$ is the main decryption key that nobody has. A possible way of computing $[x]R$ is by calculating the \textit{Lagrange Interpolation Polynomial} where each term is a partial decryption $S_i$ received from a trustee $\mathcal{T}_i$ that needs to be multiplied by the \textit{Lagrange Interpolation Polynomial coefficient} which is $\lambda(i) = \prod_{j \in \tau, j \neq i} \frac{-j}{i-j} \pmod q$. Formally, $M \gets C - \sum_{i \in \tau} [\lambda(i)]S_i\text{, with } |\tau| \geq t$. Note that the \textit{Lagrange Interpolation Polynomial} can be computed only when the number of terms is at least the degree of the polynomial, i.e $|\tau| \geq t$.

\begin{figure}[ht]
    \centering
    \begin{tikzpicture}[framed, node distance=0,
            spaced/.style={yshift=-4},
            % every node/.style={draw}
            ]{
            
        % Actors
        \node[title, above, anchor=north east] (s) {
            \textbf{Server}};
        \node[title, right=of s] (t) {
            \textbf{Trustee $\mathcal{T}_i$}};
            
        % internal knowledge
        \node[ik] at (s.south) (s_ik) {
            internal knowledge: $e = (R, C)$, \\
            $\{ sY_1, ..., sY_n \}$
            };
        \node[ik] at (t.south) (t_ik) {
            internal knowledge: $sx_i$
            };
        
        %All content
        \node[block, keep_middle, spaced] at (s_ik.south -| s.center) (s_1) {
            $\tau \gets \{ \}$
            };
        \node[arrow, towards_right, immediate, between={s.center}{t.center}] at (s_1.south -| s.center) (a_1) {
            \hspace{10pt} $e = (R, C)$
            };
        \node[block, keep_right, spaced] at (a_1.south -| t.east) (t_1) {
            $S_i \gets [sx_i]R$ \\
            $PK \gets \mathsf{DLProve} (sx_i, \{ G, R \})$
            };
        \node[arrow, towards_left, immediate, between={s.center}{t.center}] at (t_1.south -| s.center) (a_2) {
            $S_i$, $PK$ \hspace{30pt}
            };
        \node[block, keep_left, spaced] at (a_2.south -| s.west) (s_2) {
            verify $\mathsf{DLVer} (PK, \{ G, R \}, \{ sY_i, S_i \})$ then: \\ [7pt]
            $\tau \gets \tau \cup \{ i \}$
            };
        \node[banner, keep_left, spaced, between={s.west}{t.east}] at (s_2.south -| s.west) (b_1) {
            when enough $\mathcal{T}_i$ have published $S_i$, i.e. $|\tau| \geq t$
            };
        \node[block, keep_left, spaced] at (b_1.south -| s.west) (s_3) {
            $\lambda(i) \gets \prod\limits_{j \in \tau, j \neq i} \frac{-j}{i-j} \pmod q$, with $i \in \tau$ \\
            $M \gets C - \sum_{i \in \tau} [\lambda(i)]S_i$
            };
        
        % Arrows and lines
        \draw[dashed] (s.south west)--(t.south east);
        \draw[dashed] (s_ik.south -| s.west)--(s_ik.south -| t.east);

        \draw[dotted] (b_1.north west)--(b_1.north east);
        \draw[dotted] (b_1.south west)--(b_1.south east);

        \draw[densely dotted] (s_1.south -| s.center)--(s_2.north -| s.center);
        \draw[densely dotted] (s_2.south -| s.center)--(b_1.north -| s.center);
        \draw[densely dotted] (b_1.south -| s.center)--(s_3.north -| s.center);

        \draw[densely dotted] (a_1.south east)--(t_1.north -| t.center);
        \draw[densely dotted] (t_1.south -| t.center)--(a_2.south east);
        }
    \end{tikzpicture}
    \caption{Threshold decryption}
    \label{fig: threshold decryption}
\end{figure}


\subsubsection{Proving the Content of a Cryptogram} \label{app: proving the content of a cryptogram}
Once a cryptogram is generated $e = (R, C) \gets \mathsf{Enc} (Y, M; r)$, only the \textit{sender} (the one who generated the cryptogram) and the \textit{receiver} (the one in possession of the decryption key $x$) know the value of the message $M$. Both of them have the possibility to prove to somebody else (or publicly prove) the content of the cryptogram.

The one who generated the cryptogram can prove to a verifier that the cryptogram $e$ contains message $M$ by engaging in the protocol from \cref{fig: protocol for proving multiple discrete logarithms} in order to prove the knowledge of the randomizer $PK[(r): R = [r]G \wedge (C - M) = [r]Y]$. To generate a publicly verifiable proof, the \textit{sender} can generate a non-interactive proof $PK \gets \mathsf{DLProve} (r, \{ G, Y \})$ (\cref{alg: dl prove}). Any public verifier is convinced that cryptogram $e$ contains message $M$ if the verification algorithm succeeds $\mathsf{DLVer} (PK, \{ G, Y \}; \{ R, C - M \})$ (\cref{alg: dl ver}).

At the same time, the one in possession of the decryption key $x$ can prove the content of the cryptogram $e$ to a verifier by engaging in the same protocol from \cref{fig: protocol for proving multiple discrete logarithms} but this time for proving the knowledge of the decryption key $PK[(x): Y = [x]G \wedge (C - M) = [x]R]$. To generate a publicly verifiable proof, the \textit{receiver} of the cryptogram can generate a non-interactive proof $PK \gets \mathsf{DLProve} (x, \{ G, R \})$ (\cref{alg: dl prove}). Any public verifier is convinced that cryptogram $e$ contains message $M$ if the verification algorithm succeeds $\mathsf{DLVer} (PK, \{ G, R \}; \{ Y, C - M \})$ (\cref{alg: dl ver}).


\subsubsection{Symmetric encryption} \label{app: symmetric encryption}
A special encryption scheme ($\mathsf{SymEnc}$, $\mathsf{SymDec}$) exists in case the message to be encrypted is not an elliptic curve point but instead, an arbitrary length message $m \in \mathbb{B}^*$ e.g. a text message. A difference from Elgamal cryptography is that both encryption and decryption are done based on the same key that needs to be known by both parties (i.e. the sender and the receiver). 

The strategy to convert from a private-public key infrastructure into a symmetric key is to use a \textit{key encapsulation method} based on \textit{Diffie Hellman Key Exchange} as described in \cref{app: diffie hellman key derivation function}. Then, the symmetric key $k$ is used to encrypt the message $m$ using a standard \textit{AES encryption} algorithm \cite{NIST01}, resulting in the encryption $e \gets \mathsf{SymEnc} (k, m)$ (\cref{alg: sym enc}). For decryption, the same symmetric key is derived and then used to decrypt the message $m \gets \mathsf{SymDec} (k, e)$ (\cref{alg: sym dec}).

We use AES algorithms with 256-bit keys in \textit{Galois Counter Mode} with a random 96-bit initialization vector $iv$, no authentication data (i.e. $ad \gets \varnothing$), and a 128-bit tag $t$. Therefore, we define algorithm $\mathsf{AES-GCM-Encrypt}$ that returns the ciphertext $c$ representing the encryption of message $m$ with the key $k$ after the AES-GCM cipher has been initialized with the initialization vector $iv$. Additionally, it returns tag $t$ that authenticates the encryption. We consider as the encryption of message $m$ the tuple $e = (iv, t, c)$.

We also define algorithm $\mathsf{AES-GCM-Decrypt}$ that returns plain text $m$ as the decryption of ciphertext $c$ with key $k$ after initializing the AES-GCM cipher with the initialization vector $iv$. Note that $m$ is returned only if the authentication tag $t$ validates.

\begin{algorithm}[ht]
    \DontPrintSemicolon
    \caption{$\mathsf{SymEnc} (k, m)$}
    \label{alg: sym enc}
    \KwData{The symmetric key $k \in \mathbb{B}^{256}$}
    \KwMoreData{The message $m \in \mathbb{B}^*$}
    
    $iv \in_\mathrm{R} \mathbb{B}^{96}$ \;
    $ad \gets \varnothing$ \;
    $tl \gets 128$ \;
    $(c, t) \gets \mathsf{AES-GCM-Encrypt} (k, m, iv, ad, tl)$ \;
    $e \gets (iv, t, c)$ \;
    \Return{$e$} \tcp*{$e \in \mathbb{B}^{96} \times \mathbb{B}^{128} \times \mathbb{B}^*$}
\end{algorithm}

\begin{algorithm}[ht]
    \DontPrintSemicolon
    \caption{$\mathsf{SymDec} (k, e)$}
    \label{alg: sym dec}
    \KwData{The symmetric key $k \in \mathbb{B}^{256}$}
    \KwMoreData{The encryption $e = (iv, t, c) \in \mathbb{B}^{96} \times \mathbb{B}^{128} \times \mathbb{B}^*$}
    
    $ad \gets \varnothing$ \;
    $m \gets \mathsf{AES-GCM-Decrypt} (k, c, iv, ad, t)$ \;
    \eIf{$m = $ failure}{
        \Return{failure} \tcp*{invalid authentication tag}
        }{
        \Return{$m$} \tcp*{$m \in \mathbb{B}^*$}
        }
\end{algorithm}

\clearpage
\subsection{Zero Knowledge Proofs}
A \textit{zero knowledge proof} (ZKP) is an algorithm by which one party (the \textit{prover}) can prove to another party (the \textit{verifier}) that she knows a secret value $x$, without disclosing any information about $x$. A ZKP can be \textit{interactive}, where the prover and the verifier have to collaborate in a protocol for the verifier to get convinced of the proof. A ZKP can also be \textit{non-interactive}. In this case, the prover alone generates a proof that is publicly verifiable, thus convincing any public verifier of its statement. 

There exists two algorithms \textsf{Prove} for generating a proof and \textsf{Ver} for verifying whether a proof validates. A classic proof has a structure of a triple (commitment, challenge and response). In an interactive zero knowledge protocol, a prover commits to a value, the verifier independently and randomly generates a challenge, the prover computes a response based on the challenge received and the verifier checks that the proof validates. The proof can be turned into a non-interactive one, using the \textit{Fiat-Shamir heuristic} as described in \cite{Fiat87}. The prover computes alone the challenge, in a deterministic manner, based on the commitment, using a hash function.


\subsubsection{Discrete Logarithm Proofs} \label{app: discrete logarithm proofs}
A simple kind of ZKP is the \textit{discrete logarithm proof} that proves knowledge of value $x$, such that $Y = [x]G$, formally 
\[ PK[(x) : Y = [x]G]. \]
The most intuitive application of this could be to prove the possession of the private key associated with a public key.

A bit more complex ZKP is the \textit{discrete logarithm equality proof} that proves that two different elliptic curve points $Y, P \in \mathbb{P}$ have the same elliptic curve discrete logarithm $x \in \mathbb{Z}_q$ in regards to two distinct generators $G, H \in \mathbb{P}$.
\[ PK[(x) : Y = [x]G \wedge P = [x]H]. \]

An optimization in proving the discrete logarithm equality between multiple points in regards to their generators has been described in \cite{Chow10}. Using the optimized algorithm for proving that
\[ PK[(x) : \bigwedge_{i=0}^{n} Y_i = [x]G_i] \]
we generate the proof $PK = (K, c, r)$ by following the protocol described in \cref{fig: protocol for proving multiple discrete logarithms}. The optimization consists in the fact that the commitment $K$ is just one point regardless of the value of $n$.

\begin{figure}[ht]
    \centering
    \begin{tikzpicture}[framed, node distance=0,
            % every node/.style={draw}
            ]{
            
        % Actors
        \node[title, above, anchor=north east] (p) {
            \textbf{Prover}};
        \node[title, right=of p] (v) {
            \textbf{Verifier}};
        
        % internal knowledge
        \node[internal, below=of p] (p_ik) {
            internal knowledge: $x$, \\
            $\boldsymbol{G} = \{G_0, ..., G_n\}$, $\boldsymbol{Y} = \{Y_0, ..., Y_n\}$
            };
        \node[internal, below=of v] (v_ik) {
            internal knowledge:  \\
            $\boldsymbol{G} = \{G_0, ..., G_n\}$, $\boldsymbol{Y} = \{Y_0, ..., Y_n\}$
            };
        
        %All content
        \node[block, spaced, keep_left] at (p_ik.south -| p.west) (p_1) {
            $k \in_\mathrm{R} \mathbb{Z}_q$ \\
            $z_i \gets \mathcal{H} (i || Y_1 || ... || Y_n)$, with $i \in \{ 1, ..., n \}$ \\
            $K \gets [k](G_0 + \sum\limits_{i=1}^n [z_i]G_i)$
            };
        \node[arrow, towards_right, immediate, between={p.center}{v.center}] at (p_1.south -| p.center) (a_1) {
            \hspace{80pt} $K$
            };
        \node[block, spaced, keep_middle] at (a_1.south -| v.center) (v_1) {
            $c \in_\mathrm{R} \mathbb{Z}_q$
            };
        \node[arrow, towards_left, immediate, between={p.center}{v.center}] at (v_1.south -| p.center) (a_2) {
            $c$
            };
        \node[block, spaced, keep_middle] at (a_2.south -| p.center) (p_2) {
            $r \gets k + c \cdot x \pmod q$
            };
        \node[arrow, towards_right, immediate, between={p.center}{v.center}] at (p_2.south -| p.center) (a_3) {
            $r$
            };
        \node[block, spaced, keep_right] at (a_3.south -| v.east) (v_2) {
            $z_i = \mathcal{H} (i || Y_1 || ... || Y_n)$, with $i \in \{ 1, ..., n \}$ \\ [7pt]
            verify that $[r](G_0 + \sum\limits_{i=1}^n [z_i]G_i) = K + [c](Y_0 + \sum\limits_{i=1}^n [z_i]Y_i)$
            };
        
        % Arrows and lines
        \draw[dashed] (p.south west)--(v.south east);    
        \draw[dashed] (p_ik.south west)--(p_ik.south -| v_ik.east);

        \draw[densely dotted] (p_1.south -| p.center)--(p_2.north -| p.center);
        \draw[densely dotted] (p_2.south -| p.center)--(a_3.south west);

        \draw[densely dotted] (a_1.south east)--(v_1.north -| v.center);
        \draw[densely dotted] (v_1.south -| v.center)--(a_2.south east);
        \draw[densely dotted] (a_3.south east)--(v_2.north -| v.center);
        }
    \end{tikzpicture}
    \caption{Protocol for proving multiple discrete logarithms}
    \label{fig: protocol for proving multiple discrete logarithms}
\end{figure}

The proof of multiple discrete logarithms can be turned into a non-interactive one by computing the challenge of the proof based on the commitment, using a hash function. The proof is generated by the algorithm $PK \gets \mathsf{DLProve}(x, \boldsymbol{G})$ (\cref{alg: dl prove}), where $\boldsymbol{G} = \{G_0, ... G_n\}$ is the list of generators.

A public verifier accepts the proof if the algorithm $\mathsf{DLVer}(PK, \boldsymbol{G}; \boldsymbol{Y})$ returns true, where $\boldsymbol{Y} = \{Y_0, ..., Y_n\}$. The verification algorithm is described in \cref{alg: dl ver}.

\begin{algorithm}[ht]
    \DontPrintSemicolon
    \caption{$\mathsf{DLProve} (x, \boldsymbol{G})$}
    \KwData{The private key $x \in \mathbb{Z}_q$}
    \myinput{The list of generators $\boldsymbol{G} = \{G_0, G_1, ..., G_n\} \in \mathbb{P}^{n+1}$}
    
    $k \in_\mathrm{R} \mathbb{Z}_q$ \;
    \For{$i \gets 1$ \KwTo $n$ \KwBy $1$}{
        $z_i \gets \mathcal{H} (i || Y_1 || ... || Y_n)$ \tcp*{$Y_j = [x]G_j$, $j \in \{ 1, ..., n \}$}
    }
    $K \gets [k](G_0 + \sum\limits_{i=1}^n [z_i]G_i)$ \;
    $c \gets \mathcal{H}(G_0 || ... || G_n || K || Y_0 || ... || Y_n)$ \;
    $r \gets k + c \cdot x \pmod q$ \;
    $PK \gets (K, c, r)$ \;
    \Return{$PK$} \tcp*{$PK \in \mathbb{P} \times \mathbb{Z}_q \times \mathbb{Z}_q$}
    
    \label{alg: dl prove}
\end{algorithm}

\begin{algorithm}[ht]
    \DontPrintSemicolon
    \caption{$\mathsf{DLVer}(PK, \boldsymbol{G}; \boldsymbol{Y})$}
    \KwData{The proof $PK = (K, c, r) \in \mathbb{P} \times \mathbb{Z}_q \times \mathbb{Z}_q$}
    \myinput{The list of generators $\boldsymbol{G} = \{G_0, G_1, ..., G_n\} \in \mathbb{P}^{n+1}$}
    \myinput{The list of public keys $\boldsymbol{Y} = \{Y_0, Y_1, ..., Y_n\} \in \mathbb{P}^{n+1}$}
    
    \For{$i \gets 1$ \KwTo $n$ \KwBy $1$}{
        $z_i \gets \mathcal{H} (i || Y_1 || ... || Y_n)$
    }
    \eIf{$c = \mathcal{H}(G_0 || ... || G_n || K || Y_0 || ... || Y_n)$ \\ 
    \KwAnd $[r](G_0 + \sum\limits_{i=1}^n [z_i]G_i) = K + [c](Y_0 + \sum\limits_{i=1}^n [z_i]Y_i)$}{
        $b \gets 1$ \tcp*{proof is valid}
        }{
        $b \gets 0$ \tcp*{proof is invalid}
        }
    \Return{$b$} \tcp*{$b \in \mathbb{B}$}
    
    \label{alg: dl ver}
\end{algorithm}

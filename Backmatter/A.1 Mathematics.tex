\subsection{Mathematics} \label{app: mathematics}


\subsubsection{Group}
In mathematics, a group $\mathcal{G} = (\mathbb{G}, \circ, \mathrm{inv}, e)$ is an algebraic structure consisting of a set $\mathbb{G}$ of elements, a binary operation indicated by symbol $\circ$, a unary operation called $\mathbf{inv}$ and a neutral element $e \in \mathbb{G}$. The following properties must be satisfied by $\mathcal{G}$:

\begin{center}
\begin{tabular}{ l l }
 \textbf{closure} & $x \circ y \in \mathbb{G}$ \\ 
 \textbf{associativity} & $x \circ (y \circ z) = (x \circ y) \circ z$ \\  
 \textbf{identity element} & $x \circ e = e \circ x = x$ \\
 \textbf{inverse element} & $x \circ \mathbf{inv}(x) = e$
\end{tabular}
\end{center}
for all $x, y, z \in \mathbb{G}$.

If $\mathcal{G}$ has a fifth property called \textit{commutativity} (i.e. $x \circ y = y \circ x$), then $\mathcal{G}$ is an \textit{abelian group}. 

Moreover, if $\mathcal{G}$ is a \textit{finite group}, then $\mathbb{G}$ has a finite number of elements and we denote $q = |\mathbb{G}|$ as the order of the group. For example, a finite group would be $(\mathbb{Z}_q, +, -, 0)$, where $\mathbb{Z}_q  = \{0, 1, ..., q-1\}$, the binary operation is addition modulo $q$, the inverse operation is negation and the identity element is 0.

The binary operation can be applied on the same element, namely $x \circ x = [2]x$. We define $[k]x$ as the operation $\circ$ applied $k$ times on the element $x$. 

A finite group $\mathcal{G} = (\mathbb{G}, \circ, \mathrm{inv}, e)$ of order $q$ is called \textit{cyclic group}, if there is a group element $g \in \mathbb{G}$, such that $\mathbb{G} = (g, [2]g, [3]g, ..., [q]g)$. In this case, the element $g$ is called the \textit{generator} of $\mathcal{G}$.


\subsubsection{Finite Field}
A field $\mathcal{F} = (\mathbb{F}, +, \cdot)$ consists of a set $\mathbb{F}$ which is an abelian group in respect to both operations: addition and multiplication. The following properties hold:
\begin{itemize}
    \item $x + y \in \mathbb{F}$ and $x \cdot y \in \mathbb{F}$
    \item $(\mathbb{F}, +, -, 0)$ is an abelian group
    \item $(\mathbb{F}^*, \cdot, ^{-1}, 1)$ is an abelian group
    \item multiplication is distributive over addition: $x \cdot (y + z) = x \cdot y + x \cdot z$
\end{itemize}
for all $x, y, z \in \mathbb{F}$.

A finite field is a filed with a finite number of elements, for example the set of integers modulo $p$, denoted $\mathbb{F}_p$, where $p$ is a prime number.

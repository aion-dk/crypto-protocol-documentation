\clearpage
\subsection{Key derivation} \label{app: key derivation}
Key derivation functions are algorithms that convert one source of randomness and secrecy (such as private keys or passwords) into different formats that can be used in different applications.


\subsubsection{Password-based key derivation function} \label{app: password-based key derivation function}
$\mathsf{PBKDF2}$ is a standard algorithm, described in \cite{RFC8018}, that converts passwords (arbitrary text) into keys that can be used in a cryptographic context. The algorithm takes as arguments a pseudorandom function, a password, a salt, an iteration count, and the desired length of the output key in bytes.

We use $\mathsf{PBKDF2}$ as a building block for $\mathsf{Pass2Key}(m)$ (\cref{alg: pass to key}) that converts password $m$ into a key pair $(x, Y)$. This can be seen as an alternative algorithm to $\mathsf{KeyGen}()$ (\cref{alg: key gen}) that can be used to get a deterministic key pair based on some random seed.

Particularly, no salt is used (i.e., salt is set to $\varnothing$), therefore, only one key pair can be derived from password $m$. The amount of iterations is set to 600.000, according to recommendations from \cite{OWASP}. The password is concatenated with a counter that gets incremented until the output of the key derivation function can be interpreted as a correct private key (i.e., the output bytes are decoded as integer $x$, then checked whether $x \in \mathbb{Z}_q$).

\begin{algorithm}[ht]
    \DontPrintSemicolon
    \caption{$\mathsf{Pass2Key}(m)$}
    \label{alg: pass to key}
    \KwData{A text $m \in \mathbb{B}^*$}
    
    $salt \gets \varnothing$ \;
    $iterations \gets 600.000$ \;
    $\ell \gets \mathsf{ByteLengthOf}(q)$ \tcp*{\cref{alg: bytes length of}}
    $i \gets 0$ \;
    \Repeat{$x < q$}{
        $x \gets \mathsf{PBKDF2}(\mathcal{H}, m || i, salt, iterations, \ell)$ \\
        \If{$x \geq q$}{
            $i \gets i + 1$ \;
            \Continue \;
        }
    }
    $Y \gets [x]G$ \;
    \Return{$(x, Y)$} \tcp*{$(x, Y) \in \mathbb{Z}_q \times \mathbb{P}$}
\end{algorithm}


\clearpage
\subsubsection{Diffie Hellman key derivation function} \label{app: diffie hellman key derivation function}
To use symmetric encryption for encrypting arbitrary amounts of data, a symmetric key needs to be derived from a private-public key environment. Algorithm $\mathsf{DerSymKey}(x_1, Y_2)$ (\cref{alg: der sym key}) deterministically computes a 256-bit symmetric key $k \in \mathbb{B}^{256}$, given a private key and a public key that are not related (i.e., $Y_2 \neq [x_1]G$).

For two entities that have a private-public key infrastructure in place (i.e., entity 1 has key pair $(x_1, Y_1)$ and entity 2 has key pair $(x_2, Y_2)$, where $Y_1 = [x_1]G$ and $Y_2 = [x_2]G$) and that know each other (i.e., entity 1 knows $Y_2$ and entity 2 knows $Y_1$), they can both derive symmetric key $k$ by running $\mathsf{DerSymKey}(x_1, Y_2)$ as entity 1 and $\mathsf{DerSymKey}(x_2, Y_1)$ as entity 2.

The algorithm performs a Diffie Hellman key exchange to reach a shared secret $S \gets [x_1]Y_2 = [x_2]Y_1 = [x_1 + x_2]G$. The resulting value is used as the keying material of a hash-based key derivation function $\mathsf{HKDF}$ (described in \cite{RFC5869}) to convert it into a uniform key $k$. Particularly, no salt and info arguments are used (i.e., salt and info are set to $\varnothing$), therefore only one symmetric key can be derived from two particular key pairs $(x_1, Y_1)$ and $(x_2, Y_2)$.

\begin{algorithm}[ht]
    \DontPrintSemicolon
    \caption{$\mathsf{DerSymKey}(x, Y)$}
    \label{alg: der sym key}
    \KwData{A private key $x \in \mathbb{Z}_q$}
    \KwMoreData{A public key $Y \in \mathbb{P}$}
    
    $salt \gets \varnothing$ \;
    $info \gets \varnothing$ \;
    $length \gets 256$ \;
    $S \gets [x]Y$ \;
    $k \gets \mathsf{HKDF}(S, salt, info, length)$ \;
    \Return{$k$} \tcp*{$k \in \mathbb{B}^{256}$}
\end{algorithm}

\clearpage
\subsection{Hash functions} \label{app: hash functions}
A \textit{cryptographic hash function} is an algorithm used for mapping data of arbitrary size to data of fixed size, also called the \textit{hash value}. We define the hash function $\mathcal{H} : \mathbb{B}^* \gets \mathbb{B}^\ell$, where $\mathbb{B}^\ell$ represents a bit array of length $\ell$. In practice, hash algorithms work on byte arrays instead of bit arrays. Thus, the length of the input or output array is $\ell/8$.

A hash value can be computed for any data, such as a string, a number, or even an object with a complex structure. The hash value would result from the hash function applied to the byte representation of that particular input data. A hash value can be computed for an arbitrary number of inputs simultaneously. In that case, the hash function is applied to the concatenation of all byte representations of each input.

A hash function is known as a \textit{one-way function}, i.e., one can easily verify that some input data maps to a given hash value, but if the input data is unknown, it is infeasible to calculate it given only a hash value. Another property of a cryptographic hash function is \textit{collision resistance}. That means finding two different input data with the same hash values is infeasible.

In our system, we will use the hash function called \textit{SHA-256} that outputs bit arrays of 256 bits in length (32-byte array).
